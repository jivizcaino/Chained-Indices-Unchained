\documentclass[12pt,a4paper]{article}

\usepackage{amssymb}
\usepackage{amsfonts}
\usepackage{graphicx}
\usepackage{amsmath}
\usepackage{caption}
\usepackage{xcolor}
\usepackage{booktabs} % For a better table layout
\usepackage{rotating} % For rotating the table
\usepackage{comment}

\setlength{\parskip}{10pt}

\textheight 23 true cm
\textwidth 15.8 true cm
\oddsidemargin 0 true cm
\evensidemargin 0 true cm
\topmargin -0.8 true cm

\thispagestyle{empty}

\renewcommand{\arraystretch}{1.2}

\renewcommand\baselinestretch{1.35}
\baselineskip=1.35\normalbaselineskip
\footnotesep=1.10\normalbaselineskip

\newtheorem{theorem}{Theorem}
\newtheorem{acknowledgement}{Acknowledgement}
\newtheorem{algorithm}{Algorithm}
\newtheorem{axiom}{Assumption}
\newtheorem{case}{Case}
\newtheorem{claim}{Claim}
\newtheorem{conclusion}{Conclusion}
\newtheorem{condition}{Condition}
\newtheorem{conjecture}{Conjecture}
\newtheorem{corollary}{Corollary}
\newtheorem{criterion}{Criterion}
\newtheorem{definition}{Definition}
\newtheorem{example}{Example}
\newtheorem{exercise}{Exercise}
\newtheorem{lemma}{Lemma}
\newtheorem{notation}{Notation}
\newtheorem{problem}{Problem}
\newtheorem{proposition}{Proposition}
\newtheorem{remark}{Remark}
\newtheorem{solution}{Solution}
\newtheorem{summary}{Summary}

\def\x{x}  % Change here the symbol for investment if you must

\def\equiv{\doteq}  % Change here the symbol for 'equal by definition'


\begin{document}

\title{Chained Indices Unchained: \\ {\Large On the Welfare Foundations of Measuring Income Growth}}

\author{
Omar Licandro%
\footnote{The author expresses gratitude to Ariel Burstein for a highly beneficial discussion held during a visit to UCLA in April 2024.} \\ {\small U. of Leicester and IAE-BSE} 
\and
Juan Ignacio Vizcaino \\ {\small University of Nottingham} 
\vspace{.5cm}}




\date{November 2024}

\maketitle

\begin{abstract}

\noindent  

\medskip\noindent {\sc Keywords}: Quantity indexes

\medskip\noindent {\sc JEL classification numbers}: 

\end{abstract}
 
\thispagestyle{empty}

\newpage

%%%%%%%%%%%%%%%%%%%%%%%%%%%%%%%%%%%%%%%%%


%%%%%%%%%%%%%
\section{Introduction}
%%%%%%%%%%%%%

In the 1990s, the Bureau of Economic Analysis (BEA) transitioned from fixed-base Laspeyres indices to chained Fisher-ideal indices. For Parker and Triplett (1996), such a transition aimed to address two main issues in GDP growth measurement: substitution bias and distortions in historical growth rates caused by base-year revisions.

First, Parker and Triplett (1996) note, “fixed-weighted measures misstate economic growth as one moves further from the base period,” understating growth before the base period and overstating it afterward. This issue became especially prominent in the U.S. in the late 1980's early 1990's as durable goods, particularly computers, saw rapid price declines due to sustained quality improvements. It is important to notice that quality adjustments in measuring price indices were introduced in the mid-1980s. Although these adjustments improved real GDP measurement, they further highlighted the inaccuracy of fixed-base quantity indices requiring more frequent revisions. To address this, the BEA adopted a chained Fisher-ideal index in 1996, which continuously updates weights to better reflect changes in the structure of prices, reducing substitution bias and enhancing GDP stability (Whelan, 2002).

Second, revisions to the base year create further distortions in growth rates over time. The fixed-base methodology required periodic updates to the base year to reduce overstatement in economic growth. As Parker and Triplett (1996) note, updating the fixed weights in comprehensive NIPA revisions systematically reduced real GDP growth in all periods because high-growth commodities, which usually had low prices, were assigned less weight. Although this practice provided more accurate measures in the present, it resulted in downward adjustments in past growth rates with each revision, complicating long-term growth comparisons. The BEA’s adoption of chained Fisher-ideal indices alleviated these issues by updating weights continuously, aligning growth measures more accurately with evolving economic conditions.

This paper examines the economic rational behind this fundamental change in the way real GDP growth is measured. In recent years, important progresses have been made in the application of index number theory to macroeconomics, pointing out the pros and cons of using chained indices to measure economic growth from the perspective of people's welfare.
Duran and Licandro (2024) explore the welfare implications of using chained indices in national accounts. They show that within a dynamic general equilibrium framework, the growth rate of output measured by National Income and Product Accounts (NIPA) is an equivalent variation measure of welfare gains. Their analysis reveals that a Fisher-Shell true quantity index aligns with the Divisia index, which is closely approximated by the chained Fisher Ideal index used in NIPA. This result highlights the importance of chained indices in accurately reflecting welfare changes over time, supporting their application in measuring real GDP growth in national accounts.

Contrarily, Baqaee and Burstein (QJE, 2023) critically assess the use of chained quantity indices, such as the Fisher-ideal index, for measuring GDP growth. They argue that while these indices adjust for changes in relative prices and consumption patterns, they may not accurately reflect welfare changes when preferences are non-homothetic or subject to taste shocks. Specifically, they contend that standard chain-weighted indices can understate welfare-relevant gains for current tastes, leading to discrepancies between real consumption, real GDP, and actual welfare or production levels. Their analysis suggests that in the presence of income effects and taste shocks, chained indices may not reliably measure welfare, highlighting the need for alternative approaches in such contexts.

This note aims to clarify the different approaches developed by Baqaee and Burstein (2023) and Duran and Licandro (2024) for studying the welfare properties of chained Fisher-ideal quantity indices used by national statistics offices to measure GDP growth rates. Both papers are based on the {\it Fisher-Shell principle}, which asserts that when individual preferences are not time-invariant, welfare comparisons over time should be made using a common preference order, referred to here as the {\it reference order}. Additionally, Fisher and Shell (1968) suggest that when comparing the present to the past, the natural reference order is current preferences.
As articulated by Fisher and Shell (1968):

 
\leftskip 20pt
\rightskip 20pt

\noindent{\small “It seems clear that when intertemporal problems are involved, the asymmetry of time makes the question asked assuming today’s tastes more relevant than the equally meaningful question asked assuming yesterday’s tastes.”}

\leftskip 0pt
\rightskip 0pt
\medskip

 Baqaee and Burstein (2023) use the Fisher-Shell principle to explore how much better off a consumer is at any given moment in time compared to any point in the past. 
In a continuous time framework where individuals have general preferences on a given set of consumption goods, they introduce an fequivalent variation measure, which evaluates welare gains between present and past consumption baskets using current preferences as the reference order. %--see their Definition 3. 
We will refer to this as the \textit{Baqaee-Burstein equivalent variation (BBEV)} measure. Their paper shows that a chained quantity index of consumption growth, constructed by integrating Divisia indices over a time interval, aligns with the BBEV measure if and only if preferences are stable and homothetic \footnote{See their Definition 4 and Proposition 1.}%
\footnote{It is interesting to see, that in the Baqaee and Burstein (2023) framework, if preferences are stationary,  the use of the Fisher-Shell principle is inconsequential even in the case of non-homotheticity. }.

The chained index diverges from the BBEV measure if these conditions are not met. Since the Fisher-ideal index approximates a Divisia index well, Baqaee and Burstein (2023) conclude that traditional consumption growth measures in national accounts may not fully reflect true welfare changes unless preferences are homothetic and remain stable.

Duran and Licandro (2024) also apply the Fisher-Shell principle but aim to answer a different question: Is output growth as measured in national accounts welfare-based?
To answer this question, instead of focusing on the aggregation of multiple consumption goods in a world without savings, they concentrate on measuring welfare gains over time within a dynamic general equilibrium framework.
Their main interest is the aggregation of the growth rates of consumption and investment in the context of the national accounts methodology. 

Their approach involves a two-stage application of the Fisher-Shell principle. First, they use an equivalent variation measure to compare contiguous moments using contemporaneous preferences, creating what they term the Fisher-Shell index. The Fisher-Shell index is the limit of a BBEV measure when the time interval converges to zero.
Because of continuity, they do not require homotheticity or stability in preferences to demonstrate that the Fisher-Shell index equals a Divisia index.
Second, to compare any non-contiguous pair of points in time, they chain the corresponding Fisher-Shell indices by integrating them over the relevant interval. Their findings indicate that the chained Divisia index, typically used in national accounts, accurately measures welfare gains even when preferences are unstable and non-homothetic. This means that traditional methods of measuring output growth can still be considered welfare-based under more general conditions than those required by the Baqaee and Burstein equivalent variation measure.

These approaches differ in their assumptions and implications. Baqaee and Burstein use current preferences to compare the current time with the past and argue that the alignment of welfare measures with output growth is contingent on the stability and homothetic nature of preferences. In contrast, Duran and Licandro’s framework accommodates scenarios where preferences are neither stable nor homothetic. They first compare contiguous moments using contemporaneous preferences as the reference order and then chain these welfare gains to evaluate any two points in time.

An important difference between the approaches of Baqaee and Burstein (2023) and Duran and Licandro (2024) concerns the debate in national accounts methodology on the use of fixed-base versus chained indices. Baqaee and Burstein’s approach aligns with the use of fixed-base indices by measuring all welfare gains from the perspective of current preferences and current prices. An important implication is that, from the perspective of National Accounts, a BBEV measure generates continuous revisions over time. In contrast, Duran and Licandro’s approach supports the use of chained indices, which continuously update the reference order to reflect changes in preferences and reference prices. 

The advantage of Duran and Licandro's chained strategy is that it measures welfare gains from the perspective of the corresponding preference order at each point in time. This method does not require continuous revisions, as it adapts dynamically over time. Consequently, their measure is time-invariant, avoiding the need to generate a completely new path of welfare gains as preferences evolve. Furthermore, when comparing two moments in the distant past, Duran and Licandro's approach uses the preferences relevant to the historical period under investigation.

In a world of stable relative prices, fixed-base and chained indices yield similar results. However, in a world with significant trends in relative prices, such as the well-known permanent decline in equipment prices relative to non-durable consumption prices, fixed-base quantity indices suffer from the so-called substitution bias \textcolor{red}{---CHECK THIS STATEMENT---}. This bias typically leads to an underestimation of output growth, necessitating frequent changes of the base year and permanent revisions. Moreover, these revisions are often substantial, making fixed-base indices less appealing, which prompted the Bureau of Economic Analysis (BEA) in the US, and many other statistical offices around the globe, to shift from a fixed-base index, usually a Laspeyres index, to a Fisher-ideal chained index for measuring GDP growth.%
\footnote{In Europe, Eurostat uses a chained Laspeyres index instead of a Fisher-ideal.}

Section \ref{sec:3} applies the methodology suggested by Baqaee and Burstein (2023) to a particular version of the dynamic framework suggested by Duran and Licandro (2024). 
In particular, it applies both methodologies to measure output growth in a dynamic general equilibrium model with embodied technical change, accommodating a permanent decline in the relative price of investment goods. 
The aim of the exercise is to show the different properties of the Fisher-Shell index suggested by Duran and Licandro (2024) and Baqaee and Burstein (2023). 
The main finding is the following. At the balanced growth path of this economy, the Fisher-Shell index proposed by Duran and Licandro (2024) equals a Divisia index, which is time-invariant, consistent with the economy being in a balanced growth path equilibrium.
The BBEV, on the other hand, clearly suffers from the well-known substitution bias problem. Past growth rates tend to be undervalued, declining over time instead of remaining constant, even when the economy is at its balanced growth path. Additionally, as time passes, both the reference order and reference prices change, leading to permanent revisions of the index.

This finding does not invalidate the measure proposed by Baqaee and Burstein (2023); rather, it highlights that chained Divisia indices are also welfare-based. Thus, this note illustrates the broader principle that multiple true quantity indices can be associated with measuring welfare gains in the same context. In light of this result, Proposition 1 in Baqaee and Burstein (2023) can be understood to demonstrate that the chained Divisia index and the Baqaee-Burstein equivalent variation measure, being alternative measures of welfare gains aiming to answer different questions, converge to each other only under homothetic and stable preferences.


%Section II of Baqaee and Burstein (2023) addresses the critical question: “How much better off is a consumer at a given moment compared to any previous point in time?”

%In scenarios where individual preferences change over time, Fisher and Shell (1968) argue that comparisons should be based on a common preference order. We will refer to this preference order as the \textit{reference order}. The question then arises: which reference order should be used when comparing the present with any moment in the past? Fisher and Shell (1968) propose using current preferences as the reference order for making such comparisons. They argue that, “when intertemporal problems are involved, the asymmetry of time makes the question asked assuming today's tastes more relevant than the equally meaningful question asked assuming yesterday's tastes.”
%We will refer to this approach as the \textit{Fisher-Shell principle}, which states that current preferences should be used as a reference order when comparing present to past choices.

%In answering the question posed above, and in line with the Fisher-Shell principle, Baqaee and Burstein (2023) define the change in welfare from comparing the current consumption basket with any past consumption basket as an \textit{equivalent variation evaluated through the lens of current preferences} (see their Definition 3). We will call this the \textit{Baqaee-Burstein equivalent variation (BBEV)} measure.

%More important, in Proposition 1, Baqaee and Burstein (2023) demonstrate that in continuous time, a chained quantity index of output growth between two different points, created by integrating the path of Divisia indices (see their Definition 4), diverges from the BBEV measure unless preferences are stable and homothetic. They conclude that “an immediate consequence of Proposition 1 is the well-known result that real consumption (as measured in National Accounts) is equal to changes in equivalent variation if and only if preferences are homothetic and stable.”

%We should then conclude that the measure of output growth that emerges from  by chaining Divisia indices, the current methodology in the System of National Accounts, is not welfare based? When applied to dynamic general equilibrium closed economies, Duran and Licandro (2024) conclude the opposite. They claim that in this framework the growth rate of output as measured in national accounts is welfare based.
%The principle suggested by Duran and Licandro (2024) is also based on the Fisher-Shell principle, but applies it in two stages. 
%At the first stage, an equivalent variation measure is created using the Fisher-Shell principle to compare any time with the closest one and shows that it is equal to a Divisia index even when preferences are non-homothetic and unstable. The refer to this measure as the Fisher-Shell index. Second, they chain the Fisher-Shell indices when comparing any two different moments in time.

%This note restricts the question formulated by Baqaee and Burstein (2023) to contiguous moments in time. Let us reformulate it as: ``How much better off is a consumer at a given moment compared to the closest previous point in time?" 
%We prove that an equivalent variation evaluated through the lens of current preferences is measurable by a Divisia index even when preferences are unstable and non-homothetic. 

%The question that then emerges is if a chainedAs these welfare gains are expressed in money-metric terms, we chain them to respond to the second question.
%In answering this question, this note reveals an alternate application of the Fisher-Shell principle that intertemporal comparisons must be based on the current preference order.



%As in Baqaee and Burstein (2023), we abstract from many other considerations related to the correct measurement of quantities and prices like product creation and destruction (Hicks 1940; Feenstra 1994; Hausman 1997; Aghion et al. 2019); real consumption does not properly account for changes in the quality of goods (see Syverson 2017); and real consumption fails to properly account for changes in nonmarket components of welfare, like changes in the user cost of durable consumption or leisure and mortality (see Jones and Klenow 2016). [CHECK THIS]


%%%%%%%%%%%%%
\section{Aggregation of Consumption and Investment}
\label{sec:3}
%%%%%%%%%%%%%
\begin{comment}
The paper examines the welfare implications of chained indices by comparing the Baqaee-Burstein equivalent variation (BBEV) measure and a Divisia-based index approach. Using U.S. National Accounts data, the analysis focuses on aggregating consumption and investment within a balanced growth path framework. Consumption is defined as non-durable goods and services, while investment encompasses durable goods, equipment, and intellectual property. Structures are excluded from the analysis due to distinct price dynamics, which simplifies the aggregation process. By applying Fisher-ideal indices, the study provides real measures for consumption and investment and constructs corresponding price deflators.

A key finding is the sustained decline in the relative price of investment over time, with the price falling to about 20\% of its 1947 level by 2023. This trend is significant for chained indices and reveals the limitations of fixed-base measures like BBEV, which tend to undervalue past growth rates due to substitution bias. The Divisia index, by contrast, maintains constant growth rates and remains invariant over time. Consequently, the BBEV measure requires ongoing revisions, while the Divisia index offers a stable alternative for measuring welfare gains in the presence of shifting relative prices.
\end{comment}

In this section, we examine the aggregation of consumption and investment within a balanced growth path (BGP) framework, focusing on the measurement of welfare gains through different indices. Using U.S. National Accounts data, we aggregate non-durable consumption, services, durable goods, equipment, and intellectual property into comprehensive measures of consumption and investment. By applying Fisher-ideal quantity and price indices, we generate real measures of both consumption and investment, excluding residential and non-residential structures due to their distinct price behaviors.

Our findings show a significant and sustained decline in the relative price of investment over time, with the relative price falling to approximately 20\% of its level of 1947 by 2023. This trend highlights the implications of relative price changes on chained indices and underscores the divergence between constant-growth Divisia indices and BBEV measures, particularly in the context of long-term economic growth.

%%%%%%%%%%%%%
\subsection{The Relative Price of Investment and Chained Indices}
%%%%%%%%%%%%%

Based on U.S. National Account data, we define consumption as non-durable consumption and services, and investment as durable goods, equipment, and intellectual property. We use Fisher-ideal quantity and price indices to aggregate the components of consumption and investment into real measures, along with the corresponding price deflators. Nominal GDP is defined as the total private expenditure on consumption and investment. Structures, both residential and non-residential, public expenditure and net exports are excluded due to their distinct price behavior. Including a third type of expenditure in our GDP definition would not fundamentally alter the analysis but would make the argument less straightforward.

Figure~\ref{fig:relative_price} illustrates the decline in the price of investment relative to consumption since 1947, the first year for which official National Accounts data are available in the U.S. Since the 1960s, the relative price of investment has shown a clear downward trend, decreasing at an average annual rate of 2.06\% since the beginning of the sample. By 2023, the relative price of investment goods has fallen to approximately 20\% of its original level. As shown in Figure~\ref{fig:consumption_share}, the share of nominal consumption in our measure of nominal GDP averages around 76\% over the sample period, dipping slightly below 76\% between 1960 and 2004 and rising slightly above this level at the beginning and end of the sample period.

\begin{figure}[h!]
\begin{center}
\includegraphics[width=.6\textwidth]{Figures/relative_price.pdf}
\end{center}
\captionsetup{justification=centerlast} % Centers the caption
\caption{Investment Prices Relative to Consumption Prices }
\label{fig:relative_price}
\begin{center}
\includegraphics[width=.6\textwidth]{Figures/consumption_share.pdf}
\end{center}
\captionsetup{justification=centerlast} % Centers the caption
\caption{Consumption Share \\ \vspace{.3cm}
{\footnotesize BEA data. Consumption aggregates non-durable consumption and services, and  investment aggregates durable consumption, equipment and intellectual property. GDP is consumption plus investment.}}
\label{fig:consumption_share}
\end{figure}

In what follows, we measure real GDP using a Fisher-ideal index based on prices and quantities of the consumption and investment measures defined above. 
The annual growth rate between 1947 and 2023, as measured by a chained Fisher-ideal index, is 3.472\%.
To assess the consistency of the Fisher-ideal index with the Fisher-Shell index, we also compute a Divisia index using past expenditure shares.%
\footnote{Compare the Divisia with current and past shares, and the Tornquist index that uses the simple average. }
The differences between these two indices are negligible, visually indistinguishable in the graphs, as the sum of squared differences is of the order of \(8 \times 10^{-10}\), with the cumulative difference between 1947 and 2023 amounting to approximately 0.5\% of the initial GDP.
Consequently, the chained Fisher-ideal approximates well the chained Fisher-Shell index suggested by Duran~and~Licandro~(2024).

\begin{figure}[h!]
\begin{center}
\includegraphics[width=.6\textwidth]{Figures/real_GDP.pdf}
\end{center}
\captionsetup{justification=centerlast} % Centers the caption
 \caption{ Real GDP \\ \vspace{.3cm}
{\footnotesize BEA data. Real GDP is measured as a chained Fisher-ideal index (solid), a 1947 base-year index (dashed) and a 2023 base-year index (dotted)}}
\label{fig:real_GDP}
\end{figure}

We also compute fixed-base indices using 1947 and 2023 as alternative base years. As shown in Figure~\ref{fig:real_GDP}, the chained Fisher-ideal index falls between the 1947 and 2023 fixed-base indices, all normalized to one in 1947.%
\footnote{Nacho: To be consistent, when we compute a 1947-fixed-base quantity index, we should go back to the aggregation of both consumption and investment, and aggregate their components using a 1947-base index. The same for the 2023-fixed-base index.}%
\footnote{A Fisher-ideal based on the 1947 and 2003 base-year index is very similar to the chained Fisher-ideal.}
By 2023, the difference between these two fixed-base indices is 44.5\%.
In this context, as pointed out by  Parker and Triplett (1996), a Laspeyres index (like the 1947 base-year index)  is systematically higher than a Paasche index (like the 2023 base-year index), the Fisher-ideal lying in the middle.

It is interesting to see that even if the relative price of investment is permaently declining at a large rate, the consumption and investment shares remains strongly stable. The substitution bias argument does not look to work in the data. More interesting, as we show in the Appendix, when income shares are constant over time, using the Cauchy-Schwarz inequality, it is easy to prove that a quantity Laspeyres index is larger than the corresponding quantity Paasche index, the Fisher-ideal index lying in the middle. The distance between them positively depending on the distance in the growth rate of the GDP components. In other words, the raise in the relative price of investment made the Laspeyres fixed-base more inaccurate than before, not because of the substitution bias, but of a Cauchy-Schwarz inequality bias.


%%%%%%%%%%%%%
\subsection{Learning by Doing and Embodied Technical Progress} \label{sec:LBD}
%%%%%%%%%%%%%

%%%%%%%%%%%%%
\subsubsection{Two-Sector Learning-by-Doing Model}
%%%%%%%%%%%%%

To compare the behavior of the base-time  BBEV measure and the chained-Fisher-Shell index, this section studies of the two-sector AK model proposed by Boucekkine~et~al.~(2003), which is a learning-by-doing extension in continuous time of Greenwood et al. (1997).
This model is sufficiently detailed to allow for a quantitative, albeit straightforward, evaluation of the problem under investigation. Additionally, since it shares the standard characteristic of an AK model, that of being at a balanced growth path from the initial time, it facilitates the comparison between the two measurement methods.

Let us assume population is defined on a continuum of measure one, and adopt the consumption good as numeraire. Let us also assume that households are identical, with representative household solving the following problem
\begin{equation}\label{eq:household}
v(k_t;\Theta_t) = \max \int_t^\infty u( c_s) \ \text{e}^{-\rho (s-t)} \text{d} s 
\end{equation}
s.t.
\begin{equation}\label{eq:Intertemp_BC}
\dot k_t = \frac{w_t}{p_t} + (r_t-\delta) k_t - \frac{c_t}{p_t} ,
\end{equation}
where $c_t$ is per capita consumption, $k_t$ is the per capita stock of capital, owned by the household, $\rho > 0$ is the subjective discount rate, and $\delta >0$ the depreciation rate. 
Preferences are constant intertemporal elasticity of substitution (CIES), with $\sigma$ representing the IES.
The vector $\Theta_t = \{p_t,w_t,r_t\}$ is the set of equilibrium prices, where $p_t$ is the price of the capital good, $w_t$ the wage rate and $r_t$ the interest rate.
The first-order conditions of the household problem collapse to the standard Euler equation
\[
\frac{\dot c_t}{c_t} = \sigma\left(r_t -\delta - \rho  + \frac{\dot p_t}{p_t}\right) ,
\]
where $r_t -\delta + \frac{\dot p_t}{p_t}$ is the user cost of capital.

There are two sectors, one producing non-durable goods in quantity $y_t$ and the other producing investment goods in quantity $i_t$. 
A measure one of perfectly competitive firms operates in each sector.
In the non-durable sector, a representative firm produces $y_t$ by means of a Cobb-Douglas technology defined on capital and labour. 
The per capita technology is
\begin{equation}\label{eq:y}
y_t = z_t k_t^\alpha
= c_t + m_t
\end{equation}
where $z_t$ is the state of technology in this sector and $\alpha \in (0,1)$.
Non-durable production is allocated to consumption, $c_t$, and as an input (materials) in the investment sector, $m_t$, both in per capita terms.
In the durable sector, materials are transformed into the investment good at rate $q_t$, s.t.,
$$i_t = q_t m_t.$$
The state of technology in the investment sector is described by $q_t$.
In the following, we will refer to improvements in $q_t$ as embodied technical progress and to improvements in $z_t$ as disembodied.
It is important to notice that, in this economy, $y_t$ measures nominal gross income.
All the exercise that follows consists in transforming it in a measure of real income emerging from people's preferences, ideally measuring gains in welfare.

Both sectors benefit from learning-by-doing (LBD) operating as knowledge spillovers from capital goods production.
Disembodied technical progress in the non-durable sector takes the form $z_t = z k_t^\gamma$.
In the investment sector embodied technical progress take the form $q_t = q k_t^\lambda$.
Parameters verify the following constraints: $z>0$, $q>0$,  $\gamma>0$, $\lambda>0$ and $\gamma+\lambda = 1-\alpha$.
LBD makes the aggregate investment technology to be AK while the aggregate technology producing the non-durable good faces decreasing returns to capital.

%Let us adopt the non-durable good as numeraire, and denote by $p_t$ to the relative price of the durable good. 
Because of linearity in the durable goods technology, at equilibrium,
\begin{equation}\label{eq:p}
p_t = q_t^{-1} =  q^{-1} k_t^{-\lambda}.
\end{equation} 
At equilibrium, the relative price of investment goods permanently declines at the rate of embodied technical progress $\lambda g_k$, where $g_k=\frac{\dot k_t}{k_t}$ is the growth rate of capital.
From the FOC for $k_t$ in the non-durable sector $r_t = \alpha z q $, which is constant thanks to the assumption that $\gamma+\lambda = 1-\alpha$.
Notice that the cost of capital for the non-durable consumption firm is $r_t p_t k_t$. Equilibrium wages are $w_t = (1-\alpha) z k_t^{1 - \lambda }$.

In this framework, an equilibrium is a path $\{c_t,k_t\}$ verifying the Euler equation
\[
\frac{\dot c_t}{c_t} = \sigma \left( 
\alpha z q  - \rho -\delta - \lambda g_{k,t}
\right)
,
\]
and the feasibility condition
% 
\[
\frac{\dot k_t}{k_t} = \big(zq  - \delta \big)  - q k_t^{\lambda-1} c_t .
\]
%The user cost of capital is $r +\delta - \frac{\dot p}{p}$ with, as shown above, $\frac{\dot p}{p} = -\lambda \frac{\dot k}{k}$.

It is easy to show that the economy is at its balanced growth path from the initial period with 
\begin{equation}\label{eq:BGP}
k_t = k_0 \,\text{e}^{g_k  t}
\ \ \ \ \ \text{and}\ \ \ \ 
c_t =   \widehat s_c\, \underbrace{z k_t^{1-\lambda}}_{y_t} = 
\underbrace{\widehat s_c\, z k_0^{1-\lambda}}_{c_0}  \,\text{e}^{g_c  t} ,
\end{equation}
where $\widehat s_c = \frac{zq-\delta - g_k}{zq}$ %\frac{(1-\alpha)zq  + \rho }{zq} $ 
is the share of consumption on total non-durable production $y_t$.
The equilibrium growth rates of capital and consumption, respectively, are
\begin{equation}\label{eq:growth}
g_k = \frac{\sigma\Big( \alpha z q - \delta - \rho \Big)}{1- \lambda + \sigma\lambda} ,
\ \ \ \ \ \text{and}\ \ \ \ 
g_c = (1-\lambda) g_k .
\end{equation}
%The shares of consumption and investment on  non-durable production are
% 
%\begin{equation}\label{eq:shares}
%s_x =  \frac{\alpha z q  - \rho - \delta}{z q - \delta}
%\ \ \ \ \text{and}\ \ \ \
%s_c = 1 - s_x.
%\end{equation}
% 
In the following, let us assume the condition $\alpha z q  - \rho - \delta > 0$ holds, implying that $g_k >0$.

\paragraph{The value function.}
The value function of the representative individual can be explicitly written in the two-sector learning-by-doing model. At equilibrium, the budget constrain in  (\ref{eq:Intertemp_BC}) reads
\[
c_t = w_t + (r - \delta - g_k) p_t k_t .
\]
Then, under $u(c) = \log c$, after substituting the equilibrium consumpton into (\ref{eq:household}),
\[
v(k_t;\Theta_t) = \frac{ \log \big(w_t + (r - \delta - g_k) p_t k_t\big)  }{\rho} + \frac{(1-\lambda)g_k} {\rho^{2}}  .
\]
This implies that, at equilibrium, 
$$
\nu_t = v'_k(k_t;\Theta_t)= \frac{  (r - \delta - g_k) p_t  }{ \rho \big(w_t + (r - \delta - g_k) p_t k_t \big) } .
$$


%%%%%%%%%%%%%
\subsubsection{Using Index Number Theory to Measure GDP Growth}
%%%%%%%%%%%%%


%%%%
\paragraph{Bellman representation of preferences.}
%%%%

Following Duran and Licandro (2024), from the Bellman equation associated to the representative household problem in (\ref{eq:household}), assuming $u(c) = \log c$, we derive the so-called Bellman representation of preferences
\[
w(c,x;\nu_t) = \log c + \nu_t  x ,
\]
where $\nu_t = v'(k_t;\Theta_t)$
is the marginal value of capital at equilibrium, and $x = \dot k$ is net investment. 
In continuous time, the Hamilton-Jacobi-Bellman (HJB) equation transforms a dynamic problem into an infinitesimal sequence of static problems.
In this sense,  $w(c,x;\nu_t)$ represents at time $t$ the same preference order as the one represented in (\ref{eq:household}).
It embodies the utility of current consumption and the utility of all future consumption generated by current net investment.
Even if the instantaneous preferences, $\log c_t$, are time invariant, since the marginal value of capital $\nu_t$ changes over time the Bellman representation is time dependent.
In the following, we will use $\nu_t$ as a shortcut to refer to the time-$t$ Bellman representation of preferences.
It is important to notice that $w(c,x;\nu_t)$ is non-homothetic and not time stable, in the sense of Baqaee and Burstein (2023).

The primal problem faced by the representative household at time $t$ is
\[
\max_{c,x}\  \log c + \nu_t x ,
\ \ \ \ \ \ \ \ \text{s.t.}\ \ \ \ \ \ \ 
c + p_t x = y_t ,
\] 
where  $y_t$ is time-$t$ net income of the representative household at  equilibrium.
The optimal solution is
\begin{equation}\label{eq:c}
c = \frac{p_t}{\nu_t} 
\ \ \ \ \text{and}\ \ \ \ 
x = \frac{y_t}{p_t} - \frac{1}{\nu_t}.
\end{equation}
%provided that $y_t \geq \frac{p_t}{\nu_t}$, otherwise $c = y_t$.
Since the Bellman representation is quasilinear in net investment, %at the interior solution, 
optimal consumption depends not on current income but on the ratio of the relative price of investment to the marginal value of capital. 
Non-consumed income is residually allocated to investment. 
%Since $\nu_t$ decreases exponentially, the contribution of investment converges to zero quite fast, while the contribution of consumption may grow unboundedly. When making this fictitious calculation, the individual does not internalise the constraint that consumption is growing unboundedly while income is bounded by $y_s$.


Substituting the optimal solution (\ref{eq:c}) in the Bellman representation of preferences, we obtain the indirect utility function
\begin{equation}\label{eq:indirect}
u(y_t,p_t;\nu_t) =
\underbrace{\log p_t - \log \nu_t }_{u(c)} + 
\underbrace{\frac{\nu_t   }{p_t}\ y_t- 1}_{\nu_t\, x} ,
\end{equation}
which depends on the utility index $\nu_t$. 
%When  $y_t < \frac{p_t}{\nu_t}$, instead, $u(y_t,p_t;\nu_t) = \log y_t$. Notice that when $y_t$ is approaching $\frac{p_t}{\nu_t}$ from above, $u(y_t,p_t;\nu_t)$ converges to $\log y_t$.
Let the individual at current time $t$  evaluate past utility using current preferences by means of $u(y_s,p_s;\nu_t)$, for $s < t$.
Since income is increasing and the investment price decreasing in the growth process, the individual perception of past consumption utility tends to grow with $t-s$ but the contribution of investment to utility tends to decrease. %Notice that the hypothetical past consumption $\frac{p_s}{\nu_t}$ is bounded by $y_s$. Since over time $p_t$ increases and $y_t$ decreases, there is $\bar s$ defined as $\frac{p_{\bar s}}{\nu_t} = y_{\bar s}$, such that for any time $s < \bar s$, $u(y_s,p_s;\nu_t) =  \log y_{s}$. Moreover, $u(y_s,p_s;\nu_t)$ converges to $ \log y_{s}$ as $s$ converges to $\bar s$ from above.

%\noindent{\bf REVISE FROM HERE}

%What are the effects of an increase in the relative price of investment on the indirect utility function?
%First, an increase in investment prices has a negative income effect on utility through the term $\frac{\nu_t y_t  }{p_t} -1$, which measures the utility of %current income allocated to investment, valued at the marginal value of capital $\nu_t$.
%Second, a higher investment price makes the representative household optimally substitute current consumption by current investment.
%This is the substitution effect of an investment price increase on utility.
%The utility of current consumption is represented by the term $\log p_t- \log \nu_t$. 

%In this framework, the income effect dominates the substitution effect. To show this, take the derivate of the indirect utility function w.r.t. $p_t$, i.e.,
% 
%\[
%\frac{\partial u(y_t,p_t;\nu_t)}{\partial p_t} = \frac{1}{p_t} - \frac{\nu_t y_t  }{p_t^2} = \frac{1}{p_t} \left( 1-\frac{1}{\widehat s_c} \right) < 0 .
%\]
% 
%The last equality results from using equation (\ref{eq:c}) and the equilibrium value of the consumption share in (\ref{eq:shares}).
%An increase in the relative price of investment has the unequivocal effect of reducing welfare. 
%Notice that in this model, a reduction in the price of investment results from an increase in the state of embodied technological knowledge, improving the production possibility frontier. An optimal allocation based on a more efficient technology will then bring more welfare.

%\noindent{\bf UP TO  HERE}


Let us now solve the dual problem
\[
\min_{c,x} c + p_t x ,
\ \ \ \ \ \ \ \ \text{s.t.}\ \ \ \ \ \ \ 
\log c + \nu_t x = u_t .
\]
As for the primal problem, at the interior soluiton $c = \frac{p_r}{\nu_t}$. The associated expenditure function is
\begin{equation}\label{eq:EF}
e(u_t,p_t;\nu_t) = \frac{p_t}{\nu_t} u_t + \frac{p_t}{\nu_t} \big( 1 - \log p_t + \log \nu_t\big).
\end{equation}
%When $u_t <  \log\left(\frac{p_t}{\nu_t}\right)$, instead, $\log(c) = u_t$ --equivalently $x=0$.
%In this case, $e(u_t,p_t;\nu_t)  = \text{e}^{ u_t}$.
%\noindent{\bf REVISE FROM HERE}
%An increase in investment prices requires a larger income to provide the same utility $u_t$, as measured by the first term at the right-hand-side.
%However, substituting out of investment may reduce this cost, as measured by the second term of the right-hand-side.
%Let us compute the first derivate of the expenditure function w.r.t. $p_t$, i.e.,
% 
%\[
%\frac{\partial e(u_t,p_t;\nu_t)}{\partial p_t} = \frac{u}{\nu_t} + \frac{1 - \log p_t + \log \nu_t}{\nu_t} - \frac{1}{\nu_t} 
% = \frac{1}{\nu_t} \left( \frac{1}{\widehat s_c} -1 \right) > 0 .
%\]
% 
%It is easy to see that the income effect of an investment price increase on total expenditure, as expected from the primal problem, dominates. 
%What is the effect of changes in the marginal value of capital $\nu_t$ on both the indirect utility function and the expenditure function? It is easy to see that both are equal to the effect of prices multiplied by minus one.

In summary, in this framework, an improvement in the environment that makes the investment sector productivity permanently increase has the direct effect of making the price of investment goods to decline; this permanent decline in the price of investment goods raises utility and reduces the income needed to generate it. Any correction of past income that takes into account the declining path of investment prices should be larger than observed past income $y_s$.

%\noindent{\bf UP TO  HERE}


\paragraph{Base-time equivalent variation measures.}

Let us now use index number theory to compare income at the current time $t$, with income at any past time $s < t$, controlling for changes in prices.
Since the Bellman representation of preferences $w(c,x;\nu_t)$ is changing over time, as pointed out by Fisher and Shell (1968), a common preference set should be used to make such a comparison.
Following the application of the Fisher-Shell principle by Baqaee and Burstein (2023), we first adopt the current representation $w(c,x;\nu_t)$ as a benchmark for intertemporal comparisons of income. 
We will refer to it as current-base equivalent variation measure. 
We will then study the alternative of adopting past preferences $w(c,x;\nu_s)$ as the benchmark, which we will refer as past-base equivalent variation measure. We will finally, in line with the well-known Fisher-ideal index used to combine Laspeyres and Paasche indices, create a sort Fisher-ideal index by combining the current-base and past-base equivalent variation measures.

In line with Baqaee and Burstein (2023), let us try to answer the question ``how much better off is the representative household in $t$ compared to $s$?” In answering this question, for any $s < t$, we take the perspective of the current representative agent and define the hypothetical income
\begin{equation}\label{eq:HI}
\widehat y_{t,s} = e\Big(u(y_s,p_s;\nu_t) ,p_t;\nu_t\Big)
= \frac{p_t}{\nu_t} 
\left(\log p_s - \log p_t \right)
+  \frac{p_t y_s}{p_s} 
 ,
\end{equation}
where $\widehat y_{t,s} $ is the expenditure at time $s$ that the representative household would needed at current prices $p_t$ to support the utility attainable with income $y_s$ at prices $p_s$ when the indirect utility function and the expenditure function are both evaluated using current preferences $\nu_t$.
In other words, $\widehat y_{t,s}$ is the level of income required at time $s$ to, at current prices $p_t$ and using current preferences $\nu_t$, provide the utility that the current representative household would have got with past income $y_s$ at past prices $p_s$. 
For the representative household at current time $t$, $\widehat y_{t,s}/y_t$ is a money metric measure of the welfare loses of moving back to the past, from current time $t$ to past time $s$.
Its inverse measures, in this particular metrics, how much better off is the representative household in $t$ compared to $s$.%
\footnote{Notice that if past prices were equal to current prices, the hypothetical income would be $y_s$, irrespective of the preference set adopted to evaluate past choices. 
In an economy with time invariant prices, the ratio of past income to current income, $y_s/y_t$, measure them these welfare loses.}
In the following we will refer to the welfare gains as measure by the log of $y_t/\widehat y_{t,s}$, for $s < t$, as the $t$-base Baqaee and Burstein equivalent variation (BBEV) measure.

%The last equality in (\ref{eq:HI}) corresponds to a situation where the hypothetical solutions in the head of the individual are interior, since the individual always exclude the possibility of having negative investments.
%From the perspective of current preferences, the current evaluation of past income, as measured by $y_{t,s}$ at the right-hand-side of equation (\ref{eq:HI}), has two main components, the first is related to consumption and the second to investment. Further the evaluation goes to the past, higher the past price is relative to the current price, and lower past income. High past prices positively affect consumption and consumption utility through the $\log p_s$ term, but reduce investment and investment utility. Low past income adds to the reduction of past investment. For $s$ close to $t$ the investment effect dominates and the hypothetical past income $y_{t,s}$ goes down. But the investment term converges to zero quite fast. Then, there is a point where the raise in consumption utility dominates and the hypothetical past income $y_{t,s}$ may start growing. This process is bounded by the condition that past hypothetical consumption is bounded by $y_s$, meaning that there exists $\bar s$ such that for any $s<\bar s$, $y_{t,s} = y_s$. The boundary is given by the $\bar s$ that solves $y_{\bar s} = \frac{p_{\bar s}}{\nu_t}$.

The Baqaee and Burstein equivalent variation measure, like a Paasche index, compare current to past income at current prices (and preferences). In this sense, we will say that the current time operates as a base time. We can then adopt the other perspective and use (\ref{eq:HI}) to measure welfare gains from the perspective of any past time, in which case, for a given past time $t$, we use (\ref{eq:HI}) to measure welfare gains of moving from $t$ to $s>t$. This alternative index will be like a Laspeyres index.
In the following we will refer to the welfare gains as measure by the log of $\widehat y_{s,t}/y_s$, for $s < t$, as the $s$-base Baqaee and Burstein equivalent variation (BBEV) measure.

Let us then adopt as the Baqaee-Burstein equivalent variation index
\begin{equation}
{\cal P}_{t,s}^{\text{BB}} =\frac{y_t}{\widehat y_{t,s}}
\end{equation}

%Notice also that the logic behind the construction of the hypothetical income $y_{t,s}$ is similar to the logic behind a Paasche index. We build a hypothetical past income valuing past quantities at current prices. The fundamental difference emerging from the use of the Fisher-Shell principle is that we don't use past observed quantities but, given income $y_s$ and prices $p_s$, the optimal quantities emerging from current instead of past preferences.

%When evaluating the time $s$ problem using current prices $p_t$, consumers would like to consume less. The term $\log p_s -\log p_t$ measures then the difference between represents the change in consumption due to 

%In a world of declining investment prices, at current prices, less income would be required at time $s$ to buy investment goods that the income required at past prices. It makes the second term at the right-hand-side of (\ref{eq:HI}) smaller than past income $y_s$. As we move back into the past, the relative price of investment goods is higher and higher, and the amount of investment goods that can be afforded with income $y_s$ is smaller and smaller. 

%The first term at the right-hand-side, indeed, measures the substitution effect. As we move to the past, investment prices are larger and larger. Optimal consumption, as given by equation (\ref{eq:c}), is then larger and larger too, requiring more and more income to afford it. 
%Valuing past investment at the current marginal value of capital $\nu_t$ makes past optimal consumption higher. Consequently, the current representative household has the perception that she was richer in the past that she previously though she was.

%How do income and substitution effects interact in the two-sector LBD economy when comparing current income $d_t$ with the past hypothetical income evaluated at current prices $p_t$ and current preferences $\nu_t$? For that, let us substitute the equilibrium conditions into (\ref{eq:HI}) to get
%\begin{equation}\label{eq:HI2}
%\frac{y_{t,s}}{y_t} = 
%  s_c 
%\left(\log p_s - \log p_t \right)
%+  \frac{p_t}{p_s} \frac{y_s}{y_t}=
%\underbrace{  s_c \lambda g_k (t-s) }_{\text{substitution effect}}+ 
%\underbrace{\ \text{e}^{g_k (s-t)} \ }_{\text{income effect}}.
%\end{equation}

%The permanent decline on investment prices, $g_p = - \lambda g_k$, plus the permanent increase in nominal income, $g_c = (1-\lambda) g_k$, make the income effect to reduce the hypothetical income relative to current income.
%Since investment declines with the raise of investment prices, in the limit it goes to zero, making the income effect of a decline in the investment price negligible.
%The substitution effect makes the hypothetical past income to raise, since the decline in prices makes consumption grow. The substitution effect will dominate since in computing our hypothetical income consumption can grow unboundedly. This is nothing else than the well known {\it substitution bias} problem of fixed based quantity indices.

%Since past investment prices were higher than current ones, as the evidence clearly show, when making the intertemporal comparison, the representative household would like to substitute out of past investment by increasing past consumption. Consequently, the term $\log p_s - \log p_t$ measures the substitution bias. From the optimal condition for consumption (\ref{eq:c}), more and more income is needed to raise consumption.
%The last term, $\frac{\nu_t d_s}{p_s}$ the max possible investment at $s$, resulting of allocating all income to buy investment goods, but valued at the current marginal value of capital $\nu_t$.

%If alternatively, past preferences were used to evaluate the hypothetical past income, the past-base equivalent variation measure would be
 
%\begin{equation}\label{eq:HIL}
%\widetilde y_{t,s} = e\Big(u(d_s,p_s;\nu_s) ,p_t;\nu_s\Big)
%= \frac{p_t}{\nu_s} 
%\left(\log p_s - \log p_t \right)
%+  \frac{p_t d_s}{p_s} 
% ,
%\end{equation}
 
%which, after substituting for the equilibrium solution, becomes
 
%\begin{equation}\label{eq:HI2L}
%\frac{\widetilde y_{t,s}}{d_t} = 
% s_c \frac{\nu_t}{\nu_s}
%\left(\log p_s - \log p_t \right)
%+  \frac{p_t}{p_s} \frac{d_s}{d_t}
%=
%\underbrace{s_c \lambda g_k (t-s)  \text{e}^{g_k (s-t)} }_{\text{substitution effect}} + 
%\underbrace{\ \text{e}^{g_k (s-t)} \ }_{\text{income effect}}.
%\end{equation}
 
%The substitution effect operates now in the opposite direction than in the BBEV measure.

%Since the BBEV measure, based on current preferences, tends to overestimate past income, and the alternative $\widetilde y_{t,s}$  measure, based on past preferences, tends to underestimate it, we will also create, in line with the Fisher-ideal index, a geometric mean of both indices for comparability. We expect that an appropriate weighting of both indices will eliminate the bias, as it does in standard static quantity indices.

\paragraph{Fisher-Shell index.}

Let us follow Duran and Licandro (2024) and first compute the derivative of $\widehat y_{t,s} $ in (\ref{eq:HI}) w.r.t. $s$, which after some simplifications becomes
\begin{equation}\label{eq:derivativeHI}
\dot{\widehat y}_{t,s} = 
\frac{p_t}{p_s} \dot y_s + 
c_t \left( 1  - \frac{p_t}{p_s}\frac{y_s}{c_t}
\right) 
\frac{\dot p_s}{p_s} .
\end{equation}
The instantaneous growth rate at $s$ of the hypothetical income $y_{t,s}$ is
\[
\frac{\dot{\widehat y}_{t,s}}{y_{t,s}} = 
\left(
\log p_s - \log p_t  + \frac{p_t}{p_s}  \frac{y_s}{c_t}
\right)^{-1}
\left(
\frac{p_t}{p_s} \frac{y_s}{c_t} g_{y,s} + 
\left( 1  - \frac{p_t}{p_s}\frac{y_s}{c_t}
\right) 
g_{p,s}
\right) ,
\]
where $\dot{y}_{t,s}$ is the derivative with respect to s, and the instantaneous growth rates of $y_s$ and $p_s$ are $g_{y,s} = \frac{\dot y_s}{y_s}$ and $g_{p,s} = \frac{\dot p_s}{p_s}$, respectively.
Moreover, from the definition of current income
\[
g_{y,s} - s_{x,s} g_{p,s} = s_{c,s} g_{c,s} + s_{x,s} g_{x,s}  ,
\]
where the income shares are $s_{c,s} = \frac{c_s}{y_s}$ and $s_{x,s} = \frac{p_s x_s}{y_s}$, respectively. Substituting it in the previous equation
\[
\frac{\dot{\widehat y}_{t,s}}{y_{t,s}} = 
\left( \frac{p_s}{p_t}  \frac{c_t}{y_s}
\big(\log p_s - \log p_t \big) + 1
\right)^{-1}
\left(
\frac{ g_{d,s}  -
\left( 1- \frac{p_s}{p_t}\frac{c_t}{y_s} \right) g_{p,s}} {g_{d,s} - s_{x,s} g_{p,s}}\right)
\Big(s_{c,s} g_{c,s} + s_{x,s} g_{x,s}\Big)
\]


When we evaluate it at $s=t$, s.t.,
\[
\dot{\widehat{y}}_{t,s}|_{s=t} =
\dot y_t - \underbrace{\left( y_t - \frac{p_t}{\nu_t}  \right)}_{p_t x_t} \frac{\dot p_t}{p_t} 
.
\]
Differentiating  the definition of income w.r.t. time, we get $\frac{\dot y_t}{y_t} - s_{it} \frac{\dot p_t}{p_t} = s_{ct} \frac{\dot c_t}{c_t}  + s_{it} \frac{\dot x_t}{x_t} $, with income shares $s_c + s_i =1$.
To finally define the Fisher-Shell index as
\[
g_t^{\text{FS}} \equiv
\frac{\dot{\widehat{y}}_{t,s}|_{s=t}}{y_t} =
s_{c,t} \frac{\dot c_t}{c_t}  + s_{i,t} \frac{\dot x_t}{x_t} .
\]
As in Durand and Licandro (2024), the Fisher-Shell index is equal to the Divisia index. Morevoer, since the LBD economy is at its balanced growth path from the initial time, all shares and growth rates are time independent. %, determined in equations (\ref{eq:growth}) and  (\ref{eq:shares}).

It is important to notice that the Baqaee and Burstein equivalent variation measure and the Fisher-Shell index are all welfare based. Even if they provide different quantitative measures of real income, they all emerge from a representation of the same preference map. However, they have different properties. In the following section, we study the behaviour of these different measures in the framework of the two-sector AK model under analysis.

%%%%%%%%%%%%%
\subsubsection{Measuring GDP Growth in Practice}
%%%%%%%%%%%%%



\paragraph{Calibration.}

The calibration in Table~\ref{tab:cal} below uses the annual US GDP data published by the Bureau of Economic Analysis (BEA) for the period 1947-2023.
We set $q=1$, without any lose of generality, and $\rho = 0.05$ to match a real interest rate of 5\%.
From the Fernald data set, we set $\alpha = 0.3356$ to the average capital's share of income.%
\footnote{See John Fernald's dataset in https://www.johnfernald.net/TFP.}
For our definitions of consumption, investment and GDP, the calibration of parameters $\lambda$, $z$ and $\delta$ aims at replicating the following moments from the US NIPA data 1947-2023: an average decline rate of the relative price of investment, $g_{p} =  -0.0206$, an average growth rate of GDP per capita, $\widehat g = 0.0247$, and an average consumption share, $\widehat s_c = 0.7608$. Parameter $\gamma$ is given by the contraint $\alpha+\lambda+\gamma=1$.

\begin{table}[ht!]
    \centering
    \caption{Calibrated Values of Parameters and Stationary Moments}
    \begin{tabular}{ccccccc|ccccc}
        \toprule
      \( q \) & \( z \) & \( \rho \) & \( \alpha \) & \( \delta \) & \( \lambda \) & \( \gamma \) & \( g_p \) & \( \widehat{s}_c \) & \( \widehat{g} \) \\
        \midrule
 1 & 0.5185 & 0.05 & 0.3356 & 0.0836 & 0.5097 & 0.1547 & - 0.0206 & 0.7608 & 0.0247 \\
        \bottomrule
    \end{tabular}
    \label{tab:cal}
\end{table}

Since national accounts approximate well a Divisa index and in the model $g_c = (1-\lambda) g_k$, the growth rate of capital $g_k$ is implicitly defined by
$$
\widehat g = \big(1- \lambda \widehat s_c\big) g_k .
$$
From (\ref{eq:p}),  $g_p = - \lambda g_k$.
We can then use the observed growth rate of gross output  per capital $\widehat g$, the gross consumption share $\widehat s_c $ and the decline rate of investment goods prices $g_{p}$ to obtain $g_k=0.0404$ and $\lambda = 0.5097$.
From the definition of the  share of consumption in net income, the productivity scale factor is calibrated at $z = 0.5185$ and the depreciation rate emerging from the equilibrium growth rate of capital in (\ref{eq:growth}) is $\delta = 0.0836$. Learning-by-doing in the non-durable sectors stands for $\gamma = 0.1547$.

Like in Figure~\ref{fig:real_GDP}, Figure~\ref{fig:GDP_LBD} shows the chained Fisher-ideal index, along with the 1947 and 2023 fixed-base indices for GDP per capita, but now using the model’s predicted quantities and prices for consumption and investment instead of the observed ones. Since the model is calibrated to reflect the long-term trends of the U.S. economy, the three measures display behavior closely resembling that seen in the actual data. By 2023, the difference between the two fixed-base indices reaches 43.4\%, nearly matching the 44.5\% observed with NIPA data. All business cycle fluctuations are smoothed out in the model.

\begin{figure}[t!]
\begin{center}
\includegraphics[width=.6\textwidth]{Figures/GDP_LBD.pdf}
\end{center}
\captionsetup{justification=centerlast} % Centers the caption
 \caption{ Real GDP \\ \vspace{.1cm}
{\footnotesize BEA data. Real GDP is measured as a chained Fisher-ideal index (solid), a 1947 base-year index (dashed) and a 2023 base-year index (dotted)}}
\label{fig:GDP_LBD}
\end{figure}


\paragraph{Fisher-Shell vs base-time equivalent variation measures.}


\begin{figure}[t!]
\begin{center}
\includegraphics[width=.6\textwidth]{Figures/BBEV_vs_FS.pdf}
\end{center}
\captionsetup{justification=centerlast} % Centers the caption
\caption{Fisher-Shell vs Fixed-Base BBEV (NDP) \\ \vspace{.1cm}
{\footnotesize Chained Fisher-Shell (solid),  BBEV 1947-fixed-base (dashed) and BBEV 2023-fixed-based  (dotted)}}
\label{fig:BBEV_vs_FS}
\end{figure}

For the above mentioned calibration, Figure \ref{fig:BBEV_vs_FS} represents the evolution of real income in the calibrated LBD economy using the chained Fisher-Shell index, as measured by the Divisia index, and the BBEV measures for the 1947 and 2023 fixed-bases. As in the previous graphs, the x-axis measures time, going from 1947 to 2023, and the y-axis measures the logarithm of the different real income measures --corresponding to NDP in the data. 
The solid line represents the chained Divisia index. It is normalised to one (zero in logs) at year 1947 and it is growing at the constant yearly rate $g = 2.17\%$. It has the property of delivering a time invariant measure, which does not depend on any particular base year, which is constant consistently with the economy being at its balanced growth path.

When measuring the 2023 fixed-base BBEV, the hypothetical interior consumption the representative agent would have liked to have at $s$, using the 2023 Bellman representation, is $\widehat c_{s,2023} = \frac{p_s}{\nu_{2023}}$. Since $\nu_{2023}$ is given, but investment prices were higher in the past, there exist a time $\bar s$ for which $\widehat c_{\bar s,2023}=y_{\bar s}$. For any time $s<\bar s$ the individual knows that $\widehat c_ {s,2023}$ is bounded by $y_{ s}$, with zero investment. 
For any $s<\bar s$, the BBEV measure will be growing at the growth rate of consumption, which is smaller than the growth rate of the economy as measured by the chained Divisia index.
For our calibration, when the individual is at time 2023, the hypothetical consumption reaches its superior boundary at time $\bar s = 2016.23$. For this reason, the BBEV evaluated at current preferences is almost equal to an  index chaining the growth rate of consumption. The 2023 fixed-base BBEV measure is very close to the 2023 fixed-base index in Figure~\ref{fig:GDP_LBD}.

 \begin{figure}[t!]
\begin{center}
\includegraphics[width=.6\textwidth]{Figures/sc_1947.pdf}
\end{center}
\caption{1947-based hypothetical consumption share}
%   \\ \vspace{.3cm}
%{\footnotesize Chained Fisher-Shell (solid),  BBEV 1947-fixed-base (dashed) and BBEV 2023-fixed-based  (dotted)}}

\label{fig:sc_1947}
\end{figure}

Interestingly, when we adopt 1947 as the base-time, meaning we evaluate the equivalent variation measure using the 1947 Bellman representation and prices, the hypothetical consumption share $\widehat c_{s,1947}/y_s$ is always positive, converging to zero as time goes to infinity --see Figure~\ref{fig:sc_1947}. At initial prices and preferences, the individual would like to keep consumption constant, but the consumption share reduces since income is growing. Consequently, as represented in Figure~\ref{fig:BBEV_vs_FS}, the reduction in the investment price induce a huge substitution bias, making the 1947-base BBEV to be 150.7\% larger than the 2023-base BBEV.




%For any time $s<t$, the BBEV measure, when applied to the Bellman representation of preferences, is the $\phi^s_t$ that verifies
%\[
%u(d_t,p_t;\nu_t) =
%u(\text{e}^{\phi^s_t} \, d_s,p_s;\nu_t) .
%\]
%The BBEV condition reads
%\[
%\widetilde d_{t,s} = \text{e}^{\phi^s_t} \, d_s
%= \frac{p_s}{\nu_t} \left(
%\log p_t + \frac{\nu_t d_t}{p_t} - \log p_s
%\right)
% .
%\]
%Taking the first derivative wrt $s$ and evaluating it at $s=t$, the instantaneous growth rate at $t$ of the BBEV is
%$$
%\frac{\dot{\widetilde{d}}_{t,s}|_{s=t}}{d_t} = \left(1-\frac{p_t}{\nu_t d_t} \right) \frac{\dot p_t}{p_t}
%$$


%For any $s < t$ the right-hand-side is negative, requiring that the growth rate implicit in the BBEV measure is smaller than $g_k$. It converges to $g_k$ when $s$ converges to $t$, but farther $s$ is from $t$ larger the distance. This property reminds the famous substitution bias effect on fixed-based quantity indices when, as it is the case in the example, the relative price of investment is permanently declining. When the economy moves ahead over time and the BBEV is computed again and again all growth rates will be revised up, concluding that we were underestimating the growth rates.


%ALTERNATIVE. For any time $s<t$, the BBEV measure, when applied to the Bellman representation of preferences, is the $\phi^s_t$ that verifies
%\[
%u(\text{e}^{\phi^s_t} \,d_t,p_t;\nu_t) =
%u( d_s,p_s;\nu_t) .
%\]
%The BBEV condition reads
%\[
%\widetilde d_{t,s} = \text{e}^{\phi^s_t} \, d_t
%= \frac{p_t}{\nu_t} \left(
%\log p_s + \frac{\nu_t d_s}{p_s} - \log p_t
%\right)
% .
%\]
%Taking the first derivative wrt $s$ and evaluating it at $s=t$, the instantaneous growth rate at $t$ of the BBEV is
%$$
%\frac{\dot{\widetilde{d}}_{t,s}|_{s=t}}{d_t} = \left(1-\frac{p_t}{\nu_t d_t} \right) \frac{\dot p_t}{p_t}
%$$

%%%%%%%%%%%%%
\section{Conclusions}
%%%%%%%%%%%%%


\newpage
\appendix

\section{Laspeyres and Paasche indices}

This appendix shows that, in an economy with constant income shares, the substitution bias does not explain why the Laspeyres index exceeds the Paasche index. Instead, this discrepancy results from the sensitivity of the Paasche index to growth rate differences between consumption and investment. 

In the context of the LBD economy of Section~\ref{sec:LBD} with two goods, consumption and investment, the Laspeyres and Paasche indices between $s$ and $t$, $s<t$, are defined as
\[
{\cal L}_{t,s} = \frac{c_t + p_s x_t}{c_s + p_s x_s}
\ \ \ \ \text{and}\ \ \ \ \
{\cal P}_{t,s} = \frac{c_t + p_t x_t}{c_s + p_t x_s} .
\]
In an economy with constant income shares, at it is the case in the LBD economy, the indices read
\[
{\cal L}_{t,s} = s_c g_c + (1-s_c)g_x
\ \ \ \ \text{and}\ \ \ \ \
{\cal P}_{t,s} = \left(s_c g_c^{-1} + (1-s_c) g_x^{-1} \right)^{-1},
\]
where the time-invariant consumption share $s_c = \frac{c_t}{c_t + p_t c_t }$; $g_c$ and $g_x$ are the time-invariant growth rates between $s$ and $t$ of consumption and investment, respectively. The Laspeyres index measures the weighted average gains from moving forward, being equal to a Divisia index, while the Paasche index measures the inverse of the weighted average loses from moving backward. They are arithmetic and harmonic means, respectively

The ratio
\[
\frac{{\cal L}}{\cal P} = 1  + s_c(1-s_c) \left( \frac{g_c}{g_x}+ \frac{g_x}{g_c} - 2\right)  \geq 1.
\]
Since \( g_c^2 + g_x^2 \geq 2 g_c g_x \), by the Cauchy-Schwarz inequality, we have the well-know result in index number theory that  \( \mathcal{L} \geq \mathcal{P}\), with equality only if \( g_c = g_x \).

Moreover, it is straightforward to show that \( \frac{\mathcal{L}}{\mathcal{P}} \) increases with the distance between \( g_c \) and \( g_x \). For given growth rates, provided \( g_c < g_x \), \( \frac{\mathcal{L}}{\mathcal{P}} \) becomes larger as the difference between \( g_c \) and \( g_x \) widens. This is because the Laspeyres index \( \mathcal{L} \), which is a weighted average of the growth rates, will tend to be pulled upward by the faster-growing component \( g_x \), while the Paasche index \( \mathcal{P} \), which inversely weights the components, is more affected by the slower-growing component \( g_c \). Therefore, the ratio \( \frac{\mathcal{L}}{\mathcal{P}} \) reflects an increasing divergence as \( g_c \) and \( g_x \) deviate further from each other.

Finally, for given growth rates, provided \( g_c \neq g_x \), the ratio \( \frac{\mathcal{L}}{\mathcal{P}} \) is hump-shaped with respect to \( s_c \), reaching its maximum when \( s_c = \frac{1}{2} \).
This can be explained as follows: when \( s_c = \frac{1}{2} \), the Laspeyres index \( \mathcal{L} \) and the Paasche index \( \mathcal{P} \) are balanced in terms of the weight given to \( g_c \) and \( g_x \). This equal weighting maximizes the effect of the divergence between \( g_c \) and \( g_x \) on \( \frac{\mathcal{L}}{\mathcal{P}} \), making the ratio the largest at \( s_c = \frac{1}{2} \). For values of \( s_c \) closer to 0 or 1, the ratio decreases as one of the growth rates dominates the calculation, reducing the impact of the difference between \( g_c \) and \( g_x \).

In this context, a larger decline rate of the relative investment price does not generate any substitution effect, since at equilibrium consumption and investment shares remain constant, but raises the distance between the growth rate of consumption and the growth rate of investment. As a consequence, it also increases the distance between the corresponding Paasche and Laspeyres indices. Figure~\ref{fig:GDP_LBD (b)} shows the 1947 and 2023 fixed-base indices for an economy with a 3.3\% annual decline of the relative investment price.  When compared to Figure~\ref{fig:GDP_LBD}, the distance in 2023 between the two indices jumped up to 106\%.
The fundamental reason is that the difference between the annual investment growth rate and the annual consumption growth rate raised from 2.59\% in the benchmark economy to 3.25\% in the economy with larger decline in the relative price of investment.

\begin{figure}[t!]
\begin{center}
\includegraphics[width=.6\textwidth]{Figures/GDP_LBD (b).pdf}
\end{center}
\captionsetup{justification=centerlast} % Centers the caption
 \caption{ Real GDP \\ \vspace{.3cm}
{\footnotesize BEA data. Real GDP is measured as a chained Fisher-ideal index (solid), a 1947 base-year index (dashed) and a 2023 base-year index (dotted)}}
\label{fig:GDP_LBD (b)}
\end{figure}


Notice that a Paasche index can be writen as
\[
{\cal P}_{t,s} = \widehat s_c g_c + (1-\widehat s_c)g_x = \left(s_c g_c^{-1} + (1-s_c) g_x^{-1} \right)^{-1}.
\]
It is easy to show that the weight of consumption in the Paashce index is
\[
\widehat s_c = \frac{s_c g_i}{(1-s_c) g_c + s_c g_i} > s_c .
\]
Since the Paasche index gives to consumption a weight larger than its income share, and consumption growth at a lower rate than investment, the Paasche index is smaller than the Laspeyres index, irespective of any substitution bias.
The larger $g_i$ is, the higher the consumption weight and the lower the Paasche index are.

More generally, let us assume there is a set of $n$ items with quantities and prices $\mathbf{x}_t = \{x_{1,t} x_{2,t},...,x_{n,t}\}$ and $\mathbf{p}_t = \{p_{1,t}, p_{2,t},...,p_{n,t}\}$, respectively. For this set of items, let us compare times $t$ and $s$, $t>s$, by the mean of the ratio of the corresponding Laspeyres and Paasche quantity indices, s.t.,
$$
\frac{{\cal L}_{t,s}}{{\cal P}_{t,s}} = \frac{\mathbf{p}_{s} \mathbf{x}_t}{\mathbf{p}_s \mathbf{x}_s} \times  \frac{\mathbf{p}_{t} \mathbf{x}_s}{\mathbf{p}_t \mathbf{x}_t}
$$
Let us assume that the shares $s_i = \frac{p_{i,t}x_{i,t}}{\mathbf{p}_t \mathbf{x}_t}$ are time invariant, then
$$
\frac{{\cal L}_{t,s}}{{\cal P}_{t,s}} =
\left(\sum s_{i} g_{i,t}\right) \left(\sum s_{i} g^{-1}_{i,t}\right) \geq 1
$$
where $g_{i,t} = \frac{x_{i,t}}{x_{i,s}}$ is the growth factor of item $i$ between $s$ and $t$.
The object at the right hand side is the product of the mean and the harmonic mean, represented by the Laspeyres and the inverse of the Paasche, respectively. The property that $\frac{{\cal L}_{t,s}}{{\cal P}_{t,s}}$ is larger than one is known as the {\it arithmetic-harmonic mean inequality}. The left-hand side of the inequality is increasing on the variance of vector $\mathbf{g}=\{g_1,g_2, ..., g_n\}$.

In an economy with log preferences defined on a vector $\mathbf{x}_t$, income shares are equal over time irrespective of prices, since income effect and substitution effect compensate each other. In this framework, the disperion of quantities will depend on relative price trends, making the Laspeyres index to overestimate growth relative to the Paasche index. It makes clear that this bad property of fixed-base indices may be unrelated to the substitution bias. 

Interestingly, the introduction of quality corrections in prices, by definition, changes the growth rate of quantities without affecting the income shares. They cannot then produce any substitution bias. However, by changing the growth rate of quantities differently, affect the variance of their growth rates. There is no economic reason, but it is a property of the indices themselves.


\section{Past- and Current-base BBEV indices}

Let us define the current-base BBEV index as
\[
{\cal P}^{\text{BB}}_{t,s} = \frac{y_t}{y_{t,s}} = y_t \left(\frac{p_t}{\nu_t}\left(\log p_s - \log p_t\right) + \frac{p_t y_s}{p_s} \right)^{-1}
\]
and the past-base BBEV index as
\[
{\cal L}^{\text{BB}}_{t,s} = \frac{y_{s,t}}{y_{s}} = \frac{1}{y_s} \left(\frac{p_s}{\nu_s}\left(\log p_t - \log p_s\right) + \frac{p_s y_t}{p_t} \right) ,
\]
where from equation (\ref{eq:HI})
\[
y_{t,s}
= \frac{p_t}{\nu_t} 
\left(\log p_s - \log p_t \right)
+  \frac{p_t y_s}{p_s} 
 .
\]
Consequently, at equilibrium of the LBD economy,
\begin{eqnarray*}
\frac{{\cal L}^{\text{BB}}_{t,s}}{{\cal P}^{\text{BB}}_{t,s}} &=&
\left(\frac{p_s}{\nu_s} \frac{1}{y_t}\left(\log p_t - \log p_s\right) + \frac{p_s}{p_t} \right)
\left(\frac{p_t}{\nu_t} \frac{1}{y_s}\left(\log p_s - \log p_t\right) + \frac{p_t}{p_s} \right) \nonumber \\
 &=&
\left(\frac{p_t}{\nu_s} \frac{1}{y_t}\left(\log p_t - \log p_s\right) + 1 \right)
\left(\frac{p_s}{\nu_t} \frac{1}{y_s}\left(\log p_s - \log p_t\right) +1 \right) \nonumber \\
 &=&
\Big( 1 - s_c \lambda g_k \,
 \text{e}^{- g_k(t-s)} (t-s)  \Big)
\Big(1 +  s_c \lambda g_k \, \text{e}^{ g_k(t-s)} (t-s)  \Big)
\end{eqnarray*}
The ratio of the past-base BBEV to the current-base BBEV behave like the ratio of the Laspeyres to the Paasche indices. It's striclty larger than one for $s<t$ and growing with the distance $t-s$. (Provide a proof)

Let us give a deeper look to ${\cal P}^{\text{BB}}_{t,s}$
\[
{\cal P}^{\text{BB}}_{t,s} =   \left(\frac{p_t}{\nu_t y_t}\left(\log p_s - \log p_t\right) + \frac{p_t y_s}{p_s y_t} \right)^{-1}
\]

%%%%%%%%%%%%%
\end{document}
%%%%%%%%%%%%%

%%%%%%%%%%%%%
\subsubsection{Three-Sector Learning-by-Doing Model}
%%%%%%%%%%%%%


Let us assume population is constant, defined on a continuum of measure one. 
A representative households has constant intertemporal elasticity of substitution preferences defined on a flow of consumption, with intertemporal elasticity of substitution $\sigma >0$.
Let us adopt the consumption good as numeraire.  

There are three sectors, one producing non-durable goods in quantity $y_t$, another producing structures in quantity $h_t$, and the third one producing investment goods in quantity $i_t$. 
A measure one of perfectly competitive firms operates in each sector.
In the non-durable sector, a representative firm produces $y_t$ by means of a Cobb-Douglas technology defined on capital and labour. 
The per capita technology is
 
\begin{equation}\label{eq:y}
y_t = z_t k_t^\alpha s_t^\beta
= c_t + m_{i,t} +  m_{h,t} ,
\end{equation}
 
where $z_t$ is the state of technology in this sector. There are decreasing returns, meaning that $\alpha >0 $, $\beta > 0$ and $\alpha + \beta < 1$.
Non-durable production is allocated to consumption, $c_t$, and as an input (materials) in the investment and construction sectors, $m_{i,t}$ and $m_{h,t}$, respectively. All variables are in per capita terms.
In the investment and construction sectors, materials are transformed into the investment good and structures at the constant rates $q_{i,t}$ and $q_{h,t}$, s.t.,
which at the time transform into the stock of capital and sturctures
\begin{equation}\label{eq:ks}
\dot k_t = q_{i,t} m_{i,t} - \delta_k k_t \ \ \ \ \text{and}\ \ \ \ \dot s_t =  q_{h,t} m_{h,t} - \delta_s s_t ,
\end{equation}
where $\delta_i > 0$ and $\delta_s>0$ represent the depreciation rates in the investment and structure technologies.

The state of technology in the investment and construction sectors is described by the vector $\{ q_{i,t}, q_{h,t} \}$.
In the following, we will refer to improvement in $\{ q_{i,t}, q_{h,t} \}$ as embodied technical progress and to improvements in $z_t$ as disembodied. (give here some references)

A planner solves the following problem
 
\begin{equation}\label{eq:household}
v(k_{s,t},k_{o,t},;\Theta_t) = \max \int_t^\infty u( c_s) \ \text{e}^{-\rho (s-t)} \text{d} s 
\end{equation}
s.t. equations (\ref{eq:y}) and (\ref{eq:ks}); $\rho > 0$ is subjective discount rate.
Vector $\Theta_t = \{z_t, q_{i,t}, q_{s,t}\}$ represents the state of technology at $t$.
Preferences are constant intertemporal elasticity of substitution (CIES), with $\sigma >0$ representing the IES.
The vector $\Theta_t = \{p_t,w_t,r_t\}$ is the set of equilibrium prices, where $p_t$ is the price of the investment good, $w_t$ the wage rate and $r_t$ the net return to capital assets.
The first order conditions of the household problem collapse to the standard Euler equation
 
\[
\frac{\dot c_t}{c_t} = \sigma\left(r_t - \rho \right) .
\]



For the following, it is important to notice that, in this economy, $y_t$ measures nominal gross income.
All the exercise that follows consists in transforming it in a measure of real income emerging from people's preferences, ideally measuring gains in welfare.

All sectors benefit from learning-by-doing (LBD) operating as knowledge spillovers from capital goods production.
Disembodied technical progress in the non-durable sector takes the form $z_t = z k_t^\gamma$.
In the investment and construction sectors embodied technical progress take the form $q_{i,t} = q_i k_t^{\lambda_i}$ and $q_{h,t} = q_h k_t^{\lambda_h}$.
Parameters verify the following constraints: $z>0$, $q_i>0$, $q_h>0$, $\gamma>0$, $\lambda_i>\lambda_h>0$ and $\gamma_i+\lambda = 1-\alpha$.

%Let us adopt the non-durable good as numeraire, and denote by $p_t$ to the relative price of the durable good. 
Because of linearity in the durable goods technology, at equilibrium,
 
\begin{equation}\label{eq:p}
p_t = q_t^{-1} =  q^{-1} k_t^{-\lambda}.
\end{equation}
 
At equilibrium, the relative price of investment goods permanently declines at the rate of embodied technical progress $\lambda g_k$, where $g_k$ is the growth rate of capital.
From the FOC for $k_t$ in the non-durable sector $r_t = \alpha z q$, which is constant thanks to the assumption that $\gamma+\lambda = 1-\alpha$.


In this framework, an equilibrium is a path $\{c_t,k_t\}$ verifying the Euler equation
 
\[
\frac{\dot c_t}{c_t} = \sigma \left( 
\alpha z q - \delta - \rho  - \lambda \frac{\dot k_t }{k_t}
\right)
,
\]
 
and the feasibility condition
% 
\[
\frac{\dot k_t}{k_t} = \big(zq  - \delta \big)  - q k_t^{\lambda-1} c_t .
\]
%The user cost of capital is $r +\delta - \frac{\dot p}{p}$ with, as shown above, $\frac{\dot p}{p} = -\lambda \frac{\dot k}{k}$.

It is easy to show that the economy is at its balanced growth path from the initial time with 
 
\begin{equation}\label{eq:BGP}
k_t = k_0 \,\text{e}^{g_k  t}
\ \ \ \ \ \text{and}\ \ \ \ 
c_t =   \widehat s_c\, \underbrace{z k_t^{1-\lambda}}_{y_t} = 
\underbrace{\widehat s_c\, z k_0^{1-\lambda}}_{c_0}  \,\text{e}^{g_c  t} ,
\end{equation}
 
where $\widehat s_c = \frac{zq-\delta - g_k}{zq}$ %\frac{(1-\alpha)zq  + \rho }{zq} $ 
is the share of consumption on total non-durable production $y_t$.
The equilibrium growth rates of capital and consumption, respectively, are
 
\begin{equation}\label{eq:growth}
g_k = \frac{\sigma\Big( \alpha z q - \delta - \rho \Big)}{1-(1-\sigma)\lambda} ,
\ \ \ \ \ \text{and}\ \ \ \ 
g_c = (1-\lambda) g_k .
\end{equation}
 
The shares of consumption and investment on  non-durable production are
 
\begin{equation}\label{eq:shares}
s_x =  \frac{\alpha z q  - \rho - \delta}{z q - \delta}
\ \ \ \ \text{and}\ \ \ \
s_c = 1 - s_x.
\end{equation}
 
In the following, let us assume the condition $\alpha z q  - \rho - \delta > 0$ holds, implying that $s_x\in(0,1)$.
In order for an economy to growth, technology must be productive enough to cover for depreciation and the value of time.

%%%%
\paragraph{Bellman representation of preferences.}
%%%%

Following Duran and Licandro (2024), from the Bellman equation associated to the representative household problem in (\ref{eq:household}), we derive the so-called Bellman representation of preferences
 
\[
w(c,x;\nu_t) = \log c + \nu_t  x ,
\ \ \ \ \ \text{where}\ \ \ \
\nu_t = v'(k_t;\Theta_t)  %= \frac{1-\lambda}{\rho k_0} \, \text{e}^{-g_k t} >0 
\]
 
is the marginal value of capital and $x = \dot k$ is net investment. 
In continuous time, the Hamilton-Jacobi-Bellman (HJB) equation transforms a dynamic problem into an infinitesimal sequence of static problems.
In this sense,  $w(c,x;\nu_t)$ represents at time $t$ the same preference order represented in (\ref{eq:household}).
It embodies the utility of current consumption and the utility of all future consumption generated by current net investment.
Even if the instantaneous preferences, $\log c_t$, are time invariant, since the marginal value of capital $\nu_t$ changes over time the Bellman representation is time dependent.
In the following, we will use $\nu_t$ as a shortcut to refer to the time-$t$ Bellman representation of preferences.

The primal problem faced by the representative household at time $t$ is
 
\[
\max_{c,x}\  \log c + \nu_t x ,
\]
s.t.
 
\begin{equation}\label{eq:d}
c + p_t x = y_t ,
\end{equation}
 
where  $y_t$ is time-$t$ net income of the representative household at  equilibrium.
The optimal solution is
 
\begin{equation}\label{eq:c}
c = \frac{p_t}{\nu_t} 
\ \ \ \ \text{and}\ \ \ \ 
x = \frac{y_t}{p_t} - \frac{1}{\nu_t}.
\end{equation}
 
Since the Bellman representation is quasilinear in net investment, optimal consumption depends not on current income but on the ratio of the relative price of investment to the marginal value of capital. Non-consumed income is residually allocated to investment.


\noindent REVISE FROM HERE

It is important to notice that at equilibrium, as shown above, the share of consumption in net income is $ s_c = \frac{(1-\alpha)zq  + \rho }{zq - \delta}$. From (\ref{eq:c}),
\[
 s_c = \frac{c_t}{y_t} =  \frac{p_t}{\nu_t y_t} = \frac{1}{\nu_t z q} \ \frac{1}{k_t}  ,
\]
which requires 
\[
\nu_t = \frac{1}{(1-\alpha)zq  + \rho}  \ \frac{1}{k_t}.
\]

 
\noindent In other words, the value function of the representative household at equilibrium reads
\begin{equation}\label{eq:v}
v(k_t;\Theta_t) = A_t +  \frac{\log(k_t)}{(1-\alpha)zq  + \rho}  ,
\end{equation}
 
where the effect of $\Theta_t$ is in $A_t$.

Substituting the optimal solution (\ref{eq:c}) in the objective, we obtain the indirect utility function
 
\begin{equation}\label{eq:indirect}
u(y_t,p_t;\nu_t) =
\log p_t - \log \nu_t  + \frac{\nu_t y_t  }{p_t} - 1 ,
\end{equation}
 
which depends on the utility index $\nu_t$.

What are the effects of an increase in the relative price of investment on the indirect utility function?
First, an increase in investment prices has a negative income effect on utility through the term $\frac{\nu_t y_t  }{p_t} -1$, which measures the utility of current income allocated to investment, valued at the marginal value of capital $\nu_t$.
Second, a higher investment price makes the representative household optimally substitute current consumption by current investment.
%This is the substitution effect of an investment price increase on utility.
The utility of current consumption is represented by the term $\log p_t- \log \nu_t$. 

In this framework, the income effect dominates the substitution effect. To show this, take the derivate of the indirect utility function w.r.t. $p_t$, i.e.,
 
\[
\frac{\partial u(y_t,p_t;\nu_t)}{\partial p_t} = \frac{1}{p_t} - \frac{\nu_t y_t  }{p_t^2} = \frac{1}{p_t} \left( 1-\frac{1}{\widehat s_c} \right) < 0 .
\]
 
The last equality results from using equation (\ref{eq:c}) and the equilibrium value of the consumption share in (\ref{eq:shares}).
An increase in the relative price of investment has the unequivocal effect of reducing welfare. 
Notice that in this model, a reduction in the price of investment results from an increase in the state of embodied technological knowledge, improving the production possibility frontier. An optimal allocation based on a more efficient technology will then bring more welfare.

Let us now solve the dual problem
 
\[
\min_{c,x} c + p_t x ,
\]
s.t.
\[
\log c + \nu_t x = u_t .
\]
 
The associated expenditure function is
 
\[
e(u_t,p_t;\nu_t) = \frac{p_t}{\nu_t} u_t + \frac{p_t}{\nu_t} \big( 1 - \log p_t + \log \nu_t\big).
\]
 
An increase in investment prices requires a larger income to provide the same utility $u_t$, as measured by the first term at the right-hand-side.
However, substituting out of investment may reduce this cost, as measured by the second term of the right-hand-side.
Let us compute the first derivate of the expenditure function w.r.t. $p_t$, i.e.,
 
\[
\frac{\partial e(u_t,p_t;\nu_t)}{\partial p_t} = \frac{u}{\nu_t} + \frac{1 - \log p_t + \log \nu_t}{\nu_t} - \frac{1}{\nu_t} 
 = \frac{1}{\nu_t} \left( \frac{1}{\widehat s_c} -1 \right) > 0 .
\]
 
It is easy to see that the income effect of an investment price increase on total expenditure, as expected from the primal problem, dominates. 

What is the effect of changes in the marginal value of capital $\nu_t$ on both the indirect utility function and the expenditure function? It is easy to see that both are equal to the effect of prices multiplied by minus one.

In summary, in this framework, an improvement in the environment that makes the investment sector productivity permanently increase has the direct effect of making the price of investment goods to decline; this permanent decline in the price of investment goods raises utility and reduces the income needed to generate it. Any correction of past income that takes into account the declining path of investment prices should be larger than observed past income $m_s$.

\noindent TO HERE
 


\paragraph{Base-year equivalent variation measures.}

Let us now use index number theory to compare income at the current time $t$, with income at any past time $s < t$, controlling for changes in prices.
Since the Bellman representation of preferences $w(c,x;\nu_t)$ is changing over time, as pointed out by Fisher and Shell (1968), a common preference set should be used to make such a comparison.
Following the application of the Fisher-Shell principle by Baqaee and Burstein (2023), we first adopt the current representation $w(c,x;\nu_t)$ as a benchmark for intertemporal comparisons of income. 
We will refer to it as current-base equivalent variation measure. 
We will then study the alternative of adopting past preferences $w(c,x;\nu_s)$ as the benchmark, which we will refer as past-base equivalent variation measure. We will finally, in line with the well-known Fisher-ideal index used to combine Laspeyres and Paasche indices, create a sort Fisher-ideal index by combining the current-base and past-base equivalent variation measures.

In line with Baqaee and Burstein (2023), let us try to answer the question ``how much better off is the representative household in $t$ compared to $s$?” In answering this question, for any $s < t$, we take the perspective of the current representative agent and define the hypothetical income
 
\begin{equation}\label{eq:HI}
y_{t,s} = e\Big(u(y_s,p_s;\nu_t) ,p_t;\nu_t\Big)
= \frac{p_t}{\nu_t} 
\left(\log p_s - \log p_t \right)
+  \frac{p_t y_s}{p_s} 
 ,
\end{equation}
 
where $y_{t,s} $ is the expenditure at time $s$ that the representative household would needed at current prices $p_t$ to support the utility attainable with income $y_s$ at prices $p_s$ when the indirect utility function and the expenditure function are both evaluated using current preferences $\nu_t$.
In other words, $y_{t,s}$ is the level of income required at time $s$ to, at current prices $p_t$ and using current preferences $\nu_t$, provide the utility that the representative household would have got with past income $y_s$ at past prices $p_s$. 
For the representative household at current time $t$, $y_{t,s}/y_t$ is a money metric measure of the welfare loses of moving back to the past, from current time $t$ to past time $s$.
Its inverse measures in this particular metrics how much better off is the representative household in $t$ compared to $s$.
Notice that if past prices were equal to current prices, the hypothetical income would be $y_s$, irrespective of the preference set adopted to evaluate past choices. 
In an economy with time invariant prices, the ratio of past income to current income, $y_s/y_t$, measure them these welfare loses. 

%Notice also that the logic behind the construction of the hypothetical income $y_{t,s}$ is similar to the logic behind a Paasche index. We build a hypothetical past income valuing past quantities at current prices. The fundamental difference emerging from the use of the Fisher-Shell principle is that we don't use past observed quantities but, given income $y_s$ and prices $p_s$, the optimal quantities emerging from current instead of past preferences.

%When evaluating the time $s$ problem using current prices $p_t$, consumers would like to consume less. The term $\log p_s -\log p_t$ measures then the difference between represents the change in consumption due to 

In a world of declining investment prices, at current prices, less income would be required at time $s$ to buy investment goods that the income required at past prices. It makes the second term at the right-hand-side of (\ref{eq:HI}) smaller than past income $y_s$. As we move back into the past, the relative price of investment goods is higher and higher, and the amount of investment goods that can be afforded with income $y_s$ is smaller and smaller. 

The first term at the right-hans-side, indeed, measures the substitution effect. As we move to the past, investment prices are larger and larger. Optimal consumption, as given by equation (\ref{eq:c}), is then larger and larger too, requiring more and more income to afford it. 
Valuing past investment at the current marginal value of capital $\nu_t$ makes past optimal consumption higher. Consequently, the current representative household has the perception that she was richer in the past that she previously though she was.

How do income and substitution effects interact in the two-sector LBD economy when comparing current income $d_t$ with the past hypothetical income evaluated at current prices $p_t$ and current preferences $\nu_t$? For that, let us substitute the equilibrium conditions into (\ref{eq:HI}) to get
\begin{equation}\label{eq:HI2}
\frac{y_{t,s}}{y_t} = 
  s_c 
\left(\log p_s - \log p_t \right)
+  \frac{p_t}{p_s} \frac{y_s}{y_t}=
\underbrace{  s_c \lambda g_k (t-s) }_{\text{substitution effect}}+ 
\underbrace{\ \text{e}^{g_k (s-t)} \ }_{\text{income effect}}.
\end{equation}
The permanent decline on investment prices, $g_p = - \lambda g_k$, plus the permanent increase in nominal income, $g_c = (1-\lambda) g_k$, make the income effect to reduce the hypothetical income relative to current income.
Since investment declines with the raise of investment prices, in the limit it goes to zero, making the income effect of a decline in the investment price negligible.
The substitution effect makes the hypothetical past income to raise, since the decline in prices makes consumption grow. The substitution effect will dominate since in computing our hypothetical income consumption can grow unboundedly. This is nothing else than the well known {\it substitution bias} problem of fixed based quantity indices.

Since past investment prices were higher than current ones, as the evidence clearly show, when making the intertemporal comparison, the representative household would like to substitute out of past investment by increasing past consumption. Consequently, the term $\log p_s - \log p_t$ measures the substitution bias. From the optimal condition for consumption (\ref{eq:c}), more and more income is needed to raise consumption.
The last term, $\frac{\nu_t d_s}{p_s}$ the max possible investment at $s$, resulting of allocating all income to buy investment goods, but valued at the current marginal value of capital $\nu_t$.

If alternatively, past preferences were used to evaluate the hypothetical past income, the past-base equivalent variation measure would be
 
\begin{equation}\label{eq:HIL}
\widetilde y_{t,s} = e\Big(u(d_s,p_s;\nu_s) ,p_t;\nu_s\Big)
= \frac{p_t}{\nu_s} 
\left(\log p_s - \log p_t \right)
+  \frac{p_t d_s}{p_s} 
 ,
\end{equation}
 
which, after substituting for the equilibrium solution, becomes
 
\begin{equation}\label{eq:HI2L}
\frac{\widetilde y_{t,s}}{d_t} = 
 s_c \frac{\nu_t}{\nu_s}
\left(\log p_s - \log p_t \right)
+  \frac{p_t}{p_s} \frac{d_s}{d_t}
=
\underbrace{s_c \lambda g_k (t-s)  \text{e}^{g_k (s-t)} }_{\text{substitution effect}} + 
\underbrace{\ \text{e}^{g_k (s-t)} \ }_{\text{income effect}}.
\end{equation}
 
The substitution effect operates now in the opposite direction than in the BBEV measure.

Since the BBEV measure, based on current preferences, tends to overestimate past income, and the alternative $\widetilde y_{t,s}$  measure, based on past preferences, tends to underestimate it, we will also create, in line with the Fisher-ideal index, a geometric mean of both indices for comparability. We expect that an appropriate weighting of both indices will eliminate the bias, as it does in standard static quantity indices.

\paragraph{Fisher-Shell index.}

Let us follow Duran and Licandro (2024) and first compute the derivative of $y_{t,s} $ in (\ref{eq:HI}) w.r.t. $s$, which after some simplifications becomes
\begin{equation}\label{eq:derivativeHI}
\dot{y}_{t,s} = 
\frac{p_t}{p_s} \dot y_s + 
c_t \left( 1  - \frac{p_t}{p_s}\frac{y_s}{c_t}
\right) 
\frac{\dot p_s}{p_s} .
\end{equation}
The instantaneous growth rate at $s$ of the hypothetical income $y_{t,s}$ is
\[
\frac{\dot{y}_{t,s}}{y_{t,s}} = 
\left(
\log p_s - \log p_t  + \frac{p_t}{p_s}  \frac{y_s}{c_t}
\right)^{-1}
\left(
\frac{p_t}{p_s} \frac{y_s}{c_t} g_{d,s} + 
\left( 1  - \frac{p_t}{p_s}\frac{y_s}{c_t}
\right) 
g_{p,s}
\right) ,
\]
where $\dot{y}_{t,s}$ is the derivative with respect to s, and the instantaneous growth rates of $y_s$ and $p_s$ are $g_{d,s} = \frac{\dot y_s}{y_s}$ and $g_{p,s} = \frac{\dot p_s}{p_s}$, respectively.
Moreover, from the definition of current income
\[
g_{d,s} - s_{x,s} g_{p,s} = s_{c,s} g_{c,s} + s_{x,s} g_{x,s}  ,
\]
where the income shares are $s_{c,s} = \frac{c_s}{y_s}$ and $s_{x,s} = \frac{p_s x_s}{y_s}$, respectively. Substituting it in the previous equation
\[
\frac{\dot{\widehat d}_{t,s}}{y_{t,s}} = 
\left( \frac{p_s}{p_t}  \frac{c_t}{y_s}
\big(\log p_s - \log p_t \big) + 1
\right)^{-1}
\left(
\frac{ g_{d,s}  -
\left( 1- \frac{p_s}{p_t}\frac{c_t}{y_s} \right) g_{p,s}} {g_{d,s} - s_{x,s} g_{p,s}}\right)
\Big(s_{c,s} g_{c,s} + s_{x,s} g_{x,s}\Big)
\]


When we evaluate it at $s=t$, s.t.,
\[
\dot{\widehat{d}}_{t,s}|_{s=t} =
\dot y_t - \underbrace{\left( y_t - \frac{p_t}{\nu_t}  \right)}_{p_t x_t} \frac{\dot p_t}{p_t} 
.
\]
Differentiating  the definition of income w.r.t. time, we get $\frac{\dot y_t}{y_t} - s_{it} \frac{\dot p_t}{p_t} = s_{ct} \frac{\dot c_t}{c_t}  + s_{it} \frac{\dot x_t}{x_t} $, with income shares $s_c + s_i =1$.
To finally define the Fisher-Shell index as
\[
g_t^{\text{FS}} \equiv
\frac{\dot{\widehat{d}}_{t,s}|_{s=t}}{y_t} =
s_{ct} \frac{\dot c_t}{c_t}  + s_{it} \frac{\dot x_t}{x_t} .
\]
As in Durand and Licandro (2024), the Fisher-Shell index is equal to the Divisia index. Morevoer, since the LBD economy is at its balanced growth path from the initial time, all shares and growth rates are time independent, determined in equations (\ref{eq:growth}) and  (\ref{eq:shares}).

It is important to notice that the base-preferences indices and the Fisher-Shell index are all welfare based. Even if they provide different quantitative measures of real income, they all emerge from a representation of the same preference map. However, they have different properties. In the following section, we study the behaviour of these different measures in the framework of the two-sector AK model under analysis.

\paragraph{Calibration.}

The calibration below uses the annual US GDP data published by John Fernald for the period 1947-2023.%
\footnote{See https://www.johnfernald.net/TFP.}
We set $q=1$, without any lose of generality, and $\rho = 0.05$.
From the Fernald data set, we set $\alpha = 0.3356$ to the average capital's share of income, the average decline rate of the relative price of investment is used to set $g_{p} =  -0.01985$, the average growth rate of gross GDP per capital is $\widehat g = 0.02267$, and the average gross investment share is set at $\widehat s_x = 0.223$.

From (\ref{eq:y}) and the LBD assumption, the growth rate of gross output is
$$
\widehat g = \big(1- \lambda (1-\widehat s_x)\big) g_k .
$$
From (\ref{eq:p}),  $g_p = - \lambda g_k$.
We can then use the observed growth rate of gross output  per capital $\widehat g$, the gross investment share $\widehat s_x $ and the decline rate of investment goods prices $g_{p}$ to obtain $g_k=0.0381$ and $\lambda = 0.5208$.
From the definition of the net income share of consumption in (\ref{eq:shares}), the productivity scale factor is calibrated at $z = 0.666$ and the depreciation rate emerging from the equilibrium growth rate of capital in (\ref{eq:growth}) is $\delta = 0.11$.


\paragraph{Fisher-Shell vs base-year equivalent variaton measures.}

For the above mentioned calibration, Figure \ref{fig:1} represents the evolution of real income in the calibrated LBD economy using three alternative base-year indices and the Fisher-Shell index. The x-axis measures time, going from 1950 to 2020, and the y-axis measures the logarithm of the different real income measures. 

The diagonal in the three figures represents the chained Divisia index, which as showed above is equal to the chained Fisher-Shell index suggested by Duran and Licandro (2024). It is normalised to one (zero in logs) at year 2020. In this representation per capita net income is growing at the constant yearly rate $g = 2.02\%$. It has the property of delivering a time invariant measure, which not depend on any particular base year, which is constant consistently with the economy being at its balanced growth path.

How do the base-year equivalent variation measures behave? In each of the three graphs in Figure 1, the four dashed lines represent the corresponding equivalent variation measures evaluated at  $t=\{1960, 1980, 2000, 2020\}$. At the evaluation time, all three measures are normalised to be in the diagonal. 
The top panel of Figure 1 represents the measure suggested by Baqaee and Burstein (2023). This equivalent variation measure, when evaluated at a particular time, does not report constant growth rates, but growth rates decline farther the economy is from the current time. Moreover, when time passes and the current time moves to the right, the evaluation of the past performances also change. In other words, a Statistical Office using this measure will be revising past growth continuously. When equivalent variation is measured at current-base preferences, since the relative price of investment permanent declines, the equivalent variation tends to overestimate past income, the overestimation growing with the distance to the current time.%
\footnote{It is important to notice that, when evaluating past income at past prices using current preferences, there is a past time before which optimal consumption is larger then income. In our calibrated economy, this arrives around forty years before the current time. For this reason, we don't report the measures more than 40 years before the evaluation time.}

Explain the substitution bias.

The middle graph in Figure 1 reports the past-based equivalent variation measure. 
 
\begin{figure}[t!]
\begin{center}
\includegraphics[width=.6\textwidth]{BBEV}
\includegraphics[width=.6\textwidth]{BBEVLaspeyres}
\includegraphics[width=.6\textwidth]{BBEVFisher}
\end{center}
\caption{Fixed-Base Preferences Equivalent Variation Measures}
\label{fig:1}
\end{figure}



%For any time $s<t$, the BBEV measure, when applied to the Bellman representation of preferences, is the $\phi^s_t$ that verifies
%\[
%u(d_t,p_t;\nu_t) =
%u(\text{e}^{\phi^s_t} \, d_s,p_s;\nu_t) .
%\]
%The BBEV condition reads
%\[
%\widetilde d_{t,s} = \text{e}^{\phi^s_t} \, d_s
%= \frac{p_s}{\nu_t} \left(
%\log p_t + \frac{\nu_t d_t}{p_t} - \log p_s
%\right)
% .
%\]
%Taking the first derivative wrt $s$ and evaluating it at $s=t$, the instantaneous growth rate at $t$ of the BBEV is
%$$
%\frac{\dot{\widetilde{d}}_{t,s}|_{s=t}}{d_t} = \left(1-\frac{p_t}{\nu_t d_t} \right) \frac{\dot p_t}{p_t}
%$$


%For any $s < t$ the right-hand-side is negative, requiring that the growth rate implicit in the BBEV measure is smaller than $g_k$. It converges to $g_k$ when $s$ converges to $t$, but farther $s$ is from $t$ larger the distance. This property reminds the famous substitution bias effect on fixed-based quantity indices when, as it is the case in the example, the relative price of investment is permanently declining. When the economy moves ahead over time and the BBEV is computed again and again all growth rates will be revised up, concluding that we were underestimating the growth rates.


%ALTERNATIVE. For any time $s<t$, the BBEV measure, when applied to the Bellman representation of preferences, is the $\phi^s_t$ that verifies
%\[
%u(\text{e}^{\phi^s_t} \,d_t,p_t;\nu_t) =
%u( d_s,p_s;\nu_t) .
%\]
%The BBEV condition reads
%\[
%\widetilde d_{t,s} = \text{e}^{\phi^s_t} \, d_t
%= \frac{p_t}{\nu_t} \left(
%\log p_s + \frac{\nu_t d_s}{p_s} - \log p_t
%\right)
% .
%\]
%Taking the first derivative wrt $s$ and evaluating it at $s=t$, the instantaneous growth rate at $t$ of the BBEV is
%$$
%\frac{\dot{\widetilde{d}}_{t,s}|_{s=t}}{d_t} = \left(1-\frac{p_t}{\nu_t d_t} \right) \frac{\dot p_t}{p_t}
%$$


\noindent IT WILL BE INTERESTING HERE TO RUN THE GREENWOOD ET AL (1997) ECONOMY AND COMPARE THE INDICES.


%%%%%%%%%%%%%%%%%%%%%%%%%%%%%%%%%%%%%%%
\end{document}
\newpage

\noindent {\bf LOOK FOR THE SUBSTITUTION BIAS}

\noindent {\bf (HERE give an intuitive explanation on how the substitution bias operates in the BBEV measure.)}

\paragraph{On the substitution bias.}
Let's divide the derivative of the hypothetical income in equation (\ref{eq:derivativeHI}) by $s$, and make some substitutions, to get
\begin{equation}\label{eq:gds}
\widehat g_{t,s} =
\frac{\dot{\widehat d}_{t,s}}{d_s} = 
\frac{p_t}{p_s} \frac{\dot d_s}{d_s} + 
\left( \frac{d_t}{d_s} s_c   - \frac{p_t}{p_s}
\right) 
g_p ,
\end{equation}
where, as defined above, $s_c = c/d$ and $g_p = \frac{\dot p_s}{p_s}$, which are both time-invariant at equilibrium.
Differentiate net income $d_s = c_s + p_s x_s$ w.r.t. $s$
\begin{equation}\label{eq:subst}
\frac{\dot{d}_{s}}{d_s} = s_c g_c + s_x (g_x + g_p) .
\end{equation}
To simplify notation, we profit from the property that the income shares and the growth rates are time independent at equilibrium.
Most of the derivations that follow don't depend on this.
Substitute (\ref{eq:subst}) into (\ref{eq:gds}) to get 
\[
\widehat g_{t,s} =
\frac{p_t}{p_s}\big(s_c g_c + s_x g_x\big) + 
\left( \frac{d_t}{d_s} s_c   - \frac{p_t}{p_s} +  \frac{p_t}{p_s} s_x 
\right) 
g_p ,
\]

From the equilibrium condition for consumption (\ref{eq:c}),
\[
g_p = g_c + g_v .
\]
reflecting changes in $c$ due to changes in $p$ and preferences, as represented by the change in the marginal value of capital $g_v = \frac{\dot {\nu}_s}{{\nu}_s}$, 
which is also constant at equilibrium. Then,
\[
\widehat g_{t,s} =
\frac{d_t}{d_s} s_c g_c +  \frac{p_t}{p_s}  s_x g_x + 
\left( \frac{d_t}{d_s} s_c   - \frac{p_t}{p_s} +  \frac{p_t}{p_s} s_x 
\right)  g_v .
\]
Notice that if preferences were not changing over time, the second term at the right-hand-side would be zero and the change on the BBEV index relative to current income at $s$ will only depend on the growth rates of consumption and investment. Notice that
\[
\frac{d_t}{d_s} s_c g_c +  \frac{p_t}{p_s}  s_x g_x = \left(\frac{d_t}{d_s} s_c  +  \frac{p_t}{p_s}  s_x \right) \Big(\widehat s_c g_c + \widehat s_x g_x\Big) ,
\]
where the perceived shares are
\[
\widehat s_c = \frac{\frac{d_t}{d_s} s_c}{\frac{d_t}{d_s} s_c  +  \frac{p_t}{p_s}  s_x}
\ \ \ \ \text{and}\ \ \ \ \
\widehat s_x = \frac{\frac{p_t}{p_s}  s_x}{\frac{d_t}{d_s} s_c  +  \frac{p_t}{p_s}  s_x}
\]
It clearly shows that when using current prices to evaluate the allocation of income in the past, ....

%%%%%%%%%%%%%
\subsubsection{Greenwood, Hercowitz and Krusell (1997)}
%%%%%%%%%%%%%

This section studies a non-stochastic continuous time version of Greenwood, Hercowitz and Krusell (1997). Preferences are represented by the utility function
\[
\int_{t}^\infty U(c_s,\ell_s) \text{e}^{-\rho(s-t)} \text{d}s
\ \ \ \ \ \text{with}\ \ \ \ \
U(c,\ell) = \theta \ln c + (1-\theta)  \ln (1-\ell) ,
\]
where $c$ is consumption per capita and $\ell$ is the fraction of time allocated to production activities. Parameters $\rho>0$ and $\theta\in(0,1)$.

A final non-durable good is produced by means of technology
\[
y_t = z_t k_t^\alpha \ell_t^{1-\alpha} ,
\]
where $k$ is the stock of capital per capita and $z$ is total factor productivity in the non-durable sector. Parameter $\alpha\in(0,1)$.
The production of the non-durable good is allocated to consumption $c_t$ and as an input  in the production of investment goods, such that
\[
y_t = c_t + \frac{i_t}{q_t} ,
\]
where the accumulation law of capital is 
\[
\dot k_t =  i_t - \delta k_t.
\]
Net investment $\dot k_t$ is gross investments $i_t$ minus depreciation, where $\delta>0$ is the depreciation rate. The investment technology transforming the non-durable good into capital benefits from the investment specific total factor productivity $q_t$. %, which is assumed to raise exogenously at the rate of embodied technical progress $\gamma$.

Let us adopt the final non-durable good as numeraire. A competitive equilibrium is a path for the exogenous states $\{z_t,q_t\}$, the endogenous state $k_t$, the aggregates quantities $\{y_t,c_t,x_t\}$, where $x_t = \dot k_t$, the relative price of investment goods $p_t=1/q_t$, the wage rate $w_t$ and the interest rate $r_t$, s.t.,
\begin{itemize}
\item The representative household solves
\[
 \rho V(k_t,z_t,q_t) = \max_{\{c_t,x_t,\ell_t\}}  U(c_t,\ell_t) + V_1(k_t,z_t,q_t) x_t + V_2(k_t,z_t,q_t) \dot z_t + V_3(k_t,z_t,q_t) \dot q_t
\]
s.t.
\[
c_t + p_t x_t = r_t p_t k_t + w_t \ell_t -\delta p_t k_t .
\]
\item The representative non-durable good firm solves
\[
\max_{\{k_t,\ell_t\}} z_t k_t^\alpha \ell_t^{1-\alpha} - r_t p_t k_t - w_t \ell_t .
\]

\item The aggregate resource constraints hold
\[
c_t +\frac{i_t}{q_t} = z_t k_t^\alpha \ell_t^{1-\alpha} 
\ \ \ \ \text{and}\ \ \ \ 
\dot k_t = i_t - \delta k_t . 
\]
\end{itemize}

\paragraph{Balanced Growth Path.}

Let us assume that $z_t = z_0\, \text{e}^{(1-\alpha)\gamma_z t}$ and $q_t = q_0\, \text{e}^{(1-\alpha)\gamma_q t}$, with $\gamma_z >0$ and $\gamma_q >0$. It is easy to see that at the balanced growth path the growth rates of consumption and investment are, respectively,
\[
g_c = \gamma_z + \alpha \gamma_q 
\ \ \ \ \text{and}\ \ \ \ 
g_k = \gamma_z + \gamma_q .
\]
Since $p_t = \frac{1}{q_t}$, the relative price of investment goods permanently decline at the rate $\frac{\dot p_t}{p_t} = (\alpha-1) \gamma_q < 0$.

The equilibrium interest rate at the balanced growth path solves the Euler equation, s.t.,
\[
r^* = \rho + \delta +\gamma_z + \gamma_q.
\]
From the non-durable firm's FOC for capital, the stationary value of capital is
\[
k_t = k^* \text{e}^{g_k t}
\ \ \ \ \text{with}\ \ \ \
k^* = \left(\frac{\alpha z_0 q_0}{r^*}  \right)^{\frac{1}{1-\alpha}} ,
\]
and the equilibrium wage rate
\[
w_t = \underbrace{(1-\alpha) z_0 k^{*\alpha}}_{w^*} \text{e}^{g_c t} .
\]
From the resource constraints
\[
i_t = \underbrace{(g_k +\delta) k^*}_{i^*} \text{e}^{g_k t} 
\ \ \ \ \text{and}\ \ \ \
c_t = \underbrace{\left( z_0 k^{*\alpha} -\frac{i^*}{q_0} \right)}_{c^*} \text{e}^{g_c t} .
\]
Finally, from the household's FOC for labor
\[
\ell^* = 1-\frac{1-\theta}{\theta} \frac{c^*}{w^*} .
\]
Let us define gross income $m_t \equiv  r_t p_t k_t + w_t  -\delta p_t k_t $ as the equilibrium return to physical capital and the value of the labour endowment owned by the individual, which is given to the individual at time $t$. Notice that at the BGP, $m_t$ is growing at the constant rate $g_c$.

When we apply the Duran and Licandro Fisher-Shell index, the growth rate of GDP is 
\[
g = s_c g_c + (1-s_c) g_k  
\ \ \ \ \text{where}\ \ \ \
s_c = \frac{z_0 k^{*\alpha} - \frac{i^*}{q_0}}{z_0 k^{*\alpha}} .
\]

Following Baqaee and Burstein (2023), let us first solve the primal problem at time $t$
\[
\max_{c,\ell,x}  \theta \ln c + (1-\theta)  \ln (1-\ell)+ %\nu\, \text{e}^{(\rho - \alpha A)t} 
v_t x ,
\]
s.t.
\[
c + p_t x + w_t(1-\ell)= m_t .
\]
The FOC for $c$ and $\ell$ are
\[
c= \frac{\theta p_t}{v_t} 
\ \ \ \ \
1-\ell =  \frac{(1-\theta)p_t}{v_t w_t} .
\ \ \ \ \text{and}\ \ \ \ 
x= \frac{m_t}{p_t} - \frac{p_t}{v_t}  .  
\]
Substituting the optimal solution in the objective, we obtain the indirect utility function
\[
u_t(m_t,p_t,w_t) = \ln p_t - (1-\theta) \ln w_t + v_t \frac{m_t}{p_t} -p_t + {\cal C}_t ,
\]
where {\small${\cal C}_t = \theta\ln\theta + (1-\theta)\ln(1-\theta)- \ln v_t$}.
Since preferences are quasilinear in $x$, the utility of income is given by the value of allocating all income $m$ to investment. 
Since the opportunity cost of increasing consumption or leisure is the value $v_t/p_t$ of reducing investment, optimal consumption and leisure depend negatively on it.

For any time $s<t$, the BBEV applied to the Bellman representation of preferences is the $\phi^s_t$ that verifies --see Baqaee and Burstein (2023) Definition 3--
\[
\ln p_t - (1-\theta) \ln w_t + v_t \frac{m_t}{p_t}-p_t  =
\ln p_s - (1-\theta) \ln w_s + v_t \frac{\text{e}^{\phi^s_t} \,m_s}{p_s} - p_s.
\]
Since at equilibrium $m_t = m^*\, \text{e}^{g_c t}$, $w_t = w^*\, \text{e}^{g_c t}$ and $p_t = p^*  \text{e}^{(\alpha-1)\gamma_q t}$, the BBEV measure becomes
\[
\text{e}^{\phi^s_t} = \text{e}^{g_k(t-s)} - \frac{\left((1-\alpha)\gamma_q +(1-\theta)g_c\right) + p^* \left(\text{e}^{(\alpha-1)\gamma_q t} - \text{e}^{(\alpha-1)\gamma_q s}\right)}{v_t \frac{m^*}{p^*} \text{e}^{g_k s}}
\]
For any $s < t$ the second term in the right-hand-side is negative, requiring that the growth rate implicit in the BBEV measure is smaller than $g_k$. It converges to $g_k$ when $s$ converges to $t$, but farther $s$ is from $t$ larger the distance. This property reminds the famous substitution bias effect on fixed-based quantity indices when, as it is the case in the example, the relative price of investment is permanently declining. When the economy moves ahead over time and the BBEV is computed again and again all growth rates will be revised up, concluding that we were underestimating the growth rates.


%%%%%%%%%%%%%
\subsection{Structural Transformation}
%%%%%%%%%%%%%

Let us interpret the structural transformation faced by the US economy  from the perspective of Comin et al. (2021). In this example, the problem is static and preferences are time invariant, then there is no need of applying a Fisher-Shell index.

Following Comin et al. (2021), let us assume there are three sectors in the economy that we denote by $j$, with $j\in\{a,m,s\}$ corresponding to agriculture, manufacturing and services, respectively. They produce the consumption goods $\mathbf{c} = \{c_a, c_m,c_s\}$ by means of linear technologies using homogeneous labor $\mathbf{l} = \{\ell_a, \ell_m,\ell_s\}$ as the sole production factor. Labor productivities are $A_j$, for $j\in\{a,m,s\}$. For simplicity, let us normalise the labor endowment to one and adopt labor as the numeraire, implying that the nominal wage is one. The problem of the representative firms in all three sectors is trivial, requiring that prices $p_j = A_j^{-1}$. 
It is easy to see that, under these assumptions, total expenditure is equal to total income, equal to one.

Household utility from consuming $\mathbf{c}$ is a function $U(\mathbf{c})$ implicitly defined by 
\begin{equation}\label{eq:NHutility}
1 = \sum_{j} \eta_j^{\frac{1}{\sigma}} \left( \frac{c_j} {U^{\epsilon_j}}\right)^{\frac{\sigma-1}{\sigma}} ,
\end{equation}
where the weights $\eta_j>0$, the elasticity of substitution between goods $\sigma >0$, and parameters $\epsilon_j > 0$ control the income elasticities.
Let us denote by $\mathbf{p} = \{p_a, p_m,p_s\}$ to the equilibrium price vector. From the expenditure minimisation problem of the household, the Hicksian demand functions, for $j=\{a,m,s\}$, are
\begin{equation}\label{eq:NHdemand}
c_j = \eta_j \left( \frac{p_j}{E}\right)^{-\sigma} U^{(1-\sigma)\epsilon_j} ,
\end{equation}
where expenditure $E = \sum_j p_jc_j$. The elasticity of the relative demand $c_j/c_i$ with respect to the utility level $U$ is $(1-\sigma)(\epsilon_j-\epsilon_i)$, implying that  non-homotheticity does not vanish in the long term. Substituting the Hicksian demands (\ref{eq:NHdemand}) into (\ref{eq:NHutility}) defines the indirect utility function $u(E,\mathbf{p}) = U$ as the implicit solution for 
\begin{equation}\label{eq:NHindutility}
1 = \sum_{j} \eta_j \left( \frac{p_j} {E}\right)^{1-\sigma} u(E,\mathbf{p})^{(1-\sigma)\epsilon_j} .
\end{equation}
The indirect utility function is homogeneous of degree zero in prices and expenditure.

After substituting optimal consumption into the definition of total expenditure, the expenditure function, representing the cost of achieving utility $U$ at prices $\mathbf{p}$, reads
\begin{equation}\label{eq:NHexpfunction}
e(U,\mathbf{p}) = \left(\sum_j \eta_j \, U^{(1-\sigma)\epsilon_j} p_j^{1-\sigma} \right)^{\frac{1}{1-\sigma}} .
\end{equation}
It is increasing in both $U$ and $\mathbf{p}$, and homogenous of first degree in $\mathbf{p}$.

At equilibrium, the level of utility is implicitly given by 
\begin{equation}\label{eq:NHU}
1 = \sum_{j} \eta_j A_j^{\sigma-1} U^{(1-\sigma)\epsilon_j} ,
\end{equation}
and the allocation of labor across sectors by
\begin{equation}\label{eq:NHl}
\ell_j = \eta_j A_j^{\sigma-1}  U^{(1-\sigma)\epsilon_j}.
\end{equation}

Let us assume that all $A_j$'s, for $j=\{a,m,s\}$, grow at the positive rates $\gamma_j$.
From (\ref{eq:NHU}), the progress in technology makes utility $U$ to increase at the rate
\[
\frac{\dot U}{U} = \sum_j \omega_j \gamma_j ,
\ \ \ \text{where}\ \ \ 
\omega_j = \frac{\eta_j A_j^{\sigma-1} U^{(1-\sigma)\epsilon_j}}{\sum_j \epsilon_j\eta_j A_j^{\sigma-1} U^{(1-\sigma)\epsilon_j}} ,
\]
and from (\ref{eq:NHl})
\[
\frac{\dot \ell_j}{\ell_j} = (\sigma-1) \left(\gamma_j-\epsilon_j \frac{\dot U}{U} \right) .
\]

In the following, we use the parameter estimations in Comin et al (2021) to study the differential properties of the BBEV and DLFS. The benchmark analysis is performed using their estimation in the first column of Table 1, implying $\sigma = 0.26$, $\epsilon_a = 0.2$ and $\epsilon_s = 1.65$, with $\epsilon_m$ normalised to one. 

\noindent{\bf TO BE COMPLETED}

%%%%%%%%%%%%%
\paragraph{References}
%%%%%%%%%%%%%

\begin{itemize}

\leftskip -20pt
\rightskip 0pt

\item[] Baqaee, David, and Ariel Burstein (2023) ``Welfare and output with income effects and taste shocks."  Quarterly Journal of Economics 138(2), 769-834.

\item[] Boucekkine, Raouf, Fernando Del Rio, and Omar Licandro (2003) ``Embodied technological change, learning‐by‐doing and the productivity slowdown." Scandinavian Journal of Economics 105(1), 87-98.

\item[] Dur\'an, Jorge and Omar Licandro (2024) ``Is the output growth rate in NIPA a welfare measure?" Economic Journal, forthcoming.

\item[] Felbermayr, Gabriel, and Omar Licandro (2005) ``The underestimated virtues of the two-sector AK model."  B.E. Journal in Macroeconomics, Topics in Macroeconomics 5(1).

\item[] Fisher, Franklin, and Karl Shell (1968) ``Taste and quality change in the pure theory of the true-cost-of-living index," in Wolfe, J. (ed.) Value, Capital, and Growth. Edinburgh University Press.

\item[] Greenwood, Jeremy, Zvi Hercowitz and Per Krusell (1997) ``Long-run implications of investment-specific technological change."  American Economic Review 879(3), 342-362.

\item[] Kongsamut, Piyabha, Sergio Rebelo, and Danyang Xie (2001) ``Beyond balanced growth."  Review of Economic Studies 68(4), 869-882.

\end{itemize}

\appendix

\section{Appendix: Primal and dual problems.}
The primal problem faced by the individual is
$$
 \max U(c;x)
\ \ \ \text{s.t.}\ \ \ 
\sum_{i=1}^N p_{i} c_{i} \leq m.
$$
The FOCs read
$$
U'_i(c;x) = \mu p_{i} .
$$
The dual problem faced by the individual is
$$
\min \sum_{i=1}^N p_{i} c_{i}
\ \ \ \text{s.t.}\ \ \ 
U(c;x) \geq  u.
$$
The FOCs read
$$
p_i = \lambda U'_i(c;x)  .
$$


\paragraph{Embodied technical progress with non-homothetic preferences.}
The planner's problem is to maximize the utility function:

\[
\max_{c(t), k(t)} \quad \int_0^\infty e^{-\rho t} \left( \left( c(t)^\beta \cdot k(t)^{1-\beta} \right)^{1-\frac{1}{\sigma}} \right) \, dt
\]

subject to the feasibility condition:

\[
\dot{k}(t) = q(0) A(0) e^{(\eta_q + \eta_A) t} \cdot k(t)^\alpha - q(0) e^{\eta_q t} \cdot c(t) - \delta \cdot k(t)
\]

and the non-negativity constraints:

\[
c(t) \geq 0, \quad k(t) \geq 0
\]

with initial capital:

\[
k(0) = k_0
\]

Preferences are non-homothetic in the sense that instantaneous utility of consumption $c_t$ depends on the level of wealth as measure by the stock of capital $k_t$. The intertemporal elasticity of substitution $\sigma >0$ is constant. 

The Bellman equation associated with this problem is:

\[
\rho V(k, t) = \max_{c} \left\{ \left( c^\beta \cdot k^{1 - \beta} \right)^{1 - \frac{1}{\sigma}} + \frac{\partial V(k, t)}{\partial t} + \frac{\partial V(k, t)}{\partial k} \left[ q(0) A(0) e^{(\eta_q + \eta_A) t} \cdot k^\alpha - q(0) e^{\eta_q t} \cdot c - \delta \cdot k \right] \right\}
\]

\section{Calibration LBD Model}


From (\ref{eq:growth}), at equilibrium, the growth rates of capital and consumption, respectively, are
\[
g_k = \alpha z q - \delta - \rho  ,
\ \ \ \ \ \text{and}\ \ \ \ 
g_c = (1-\lambda) g_k .
\]
From (\ref{eq:p}), the decline rate of the relative price of investment goods is
\[
g_p =  -\lambda g_k .
\]


The share of consumption on gross income is $\widehat s_c = \frac{(1-\alpha)zq  + \rho }{zq} $ .
From (\ref{eq:shares}), shares of consumption and investment on net income are
\[
s_x =  \frac{\alpha z q  - \rho - \delta}{z q - \delta}
\ \ \ \ \text{and}\ \ \ \
s_c = 1 - s_x.
\]

The growth rate of gross and net output, respectively, are
$$
\widehat g = (1-\widehat s_x) g_c + \widehat s_x g_k 
\ \ \ \ \text{and} \ \ \ 
g = (1-s_x) g_c +  s_x g_k 
.
$$


The calibration  uses the annual US GDP measures published by Fernald for the period 1947-2023.
We set $q=1$, without any lose of generality, and $\rho = 0.075$.
From the Fernald data set, $\alpha = 0.3356$ is the average capital's share of income, $g_{p} = 0.01985$ is the average decline rate of the relative price of investment, the growth rate of gross GDP per capital is $\widehat g = 0.02267$ and the gross investment share at $\widehat s_x = 0.223$.

Consequently, the growth rate of gross output is
$$
\widehat g = (1-\widehat s_x) g_c + \widehat s_x g_k .
$$
From (\ref{eq:y}) and the LBD assumption, $g_c = (1-\lambda) g_k$.
From (\ref{eq:p}),  $g_p = \lambda g_k$.
We can then use the observed growth rate of gross output  per capital $g = 0.0227$, the gross investment share $\widehat s_x = 0.223\%$ and $g_{p} = 1.985 \%$ to obtain $g_k=0.0381$ and $\lambda = 0.5208$.

With a discount rate $\rho = 0.075$, from the definition of the share of consumption growth income, the productivity scale factor $z = 0.666$ and the depreciation rate emerging from the equilibrium growth rate of capital in (\ref{eq:growth}) is $\delta = 0.11$.

\end{document}