\documentclass{beamer}% [handout] for  a printable copy


\usepackage{amssymb,amsfonts,amsmath,epsf,lscape}

\usepackage{mathpazo}
\usepackage{eurosym}
\usepackage{hyperref}
\usepackage{multimedia}
\usepackage{multicol}
\usepackage{color}
\usepackage{colortbl}
\usepackage{accents}
\definecolor{Red}{rgb}{1,.2,0.1}
\setcounter{MaxMatrixCols}{10}

\definecolor{vdarkblue}{rgb}{ 0.00,0.20,0.70} % This generates a text blue similar to colors in subsection
\newcommand{\blue}{\textcolor{vdarkblue}}

\usetheme{Madrid}

\setbeamertemplate{footline} {\hfill \insertframenumber{} / \inserttotalframenumber\mbox{ }\vspace{1mm}}

\setbeamertemplate{navigation symbols}{}

\newenvironment{stepenumerate}{\begin{enumerate}[<+->]}{\end{enumerate}}
\newenvironment{stepitemize}{\begin{itemize}[<+->]}{\end{itemize} }
\newenvironment{stepenumeratewithalert}{\begin{enumerate}[<+-| alert@+>]}{\end{enumerate}}
\newenvironment{stepitemizewithalert}{\begin{itemize}[<+-| alert@+>]}{\end{itemize} }

\newtheorem{assumption}{Assumption}
\newtheorem{proposition}{Proposition}

\parskip5mm
\mode<presentation>

\title{Is the output growth rate in NIPA a welfare measure?}

%\author{Jorge Dur\'an {\small (European Commission)}\\ \bigskip  Omar Licandro {\small (University of Nottingham)}}

\author{Jorge Dur\'an \hspace{1.5cm} Omar Licandro\\ \smallskip
{\small European Commission \hspace{.5cm} University of Leicester and BSE}
%\\ \hspace{4cm} {\footnotesize IEA-B}{\small arcelona} {\footnotesize{GSE}}
}

\date{Universidad de Montevideo\\ {\small May 2024}}

\begin{document}

\frame{\titlepage}



%%%%%%%%%%%%%%%%%%%%%%%%%%%%%%%%%%%%%%%%%%%%%%%%%%%%%%%%%%
\begin{frame}[label=thispaper]
\frametitle{This Paper} \pause

\begin{itemize}
	%\item Starting from the general principle that there is no measurement without theory \pause\medskip
	\item We provide a rational for output growth measurement in NIPA by \smallskip
	
	applying index number theory to dynamic general equilibrium theory \pause\bigskip
	 \begin{itemize}
	 \item {\bf The economic theory of index numbers} \smallskip
	 \item[] is at the core of GDP and CPI measurement\smallskip
	 \item[] but traditional index number theory is static \pause\bigskip
	 
	 \item Modern macroeconomics is founded on  \smallskip
	 
	 \item[] {\bf dynamic general equilibrium theory}
	 \end{itemize}
		
\end{itemize}
\end{frame}

%%%%%%%%%%%%%%%%%%%%%%%%%%%%%%%%%%%%%%%%%%%%%%%%%%%%%%%%%%
\begin{frame}
\frametitle{Main Result} \pause

	\begin{itemize}
	\item In {dynamic general equilibrium} models\smallskip
	
	 intertemporal preferences depend on the flow of consumption\pause\medskip
	 
	 \item But, NIPA aggregate final demand (consumption and investment)\pause\medskip
	 
	\item By the mean of the {\bf Bellman equation}, preferences can be represented as a function of current consumption and investment
	
	\pause\medskip
	\item {Index number theory} is applied to this representation of preferences
	
	{\small In particular, we use a \blue{\bf Fisher-Shell (FS)} index}\pause\medskip
	
	\item We show that \blue{\bf\small the FS index is equal to the Divisia index}\pause\bigskip
	
	$\Rightarrow$ \blue{\bf\small Output growth in NIPA is welfare based}\medskip
	
	\item[] {\small Under some conditions, output growth measures growth in welfare}
	\end{itemize}
\end{frame}

%%%%%%%%%%%%%%%%%%%%%%%%%%%%%%%%%%%%%%%%%%%%%%%%%%%%%%%
\part{Introduction} 
\frame{\partpage} 
\section{Introduction} 
%%%%%%%%%%%%%%%%%%%%%%%%%%%%%%%%%%%%%%%%%%%%%%%%%%%%%%%



%%%%%%%%%%%%%%%%%%%%%%%%%%%%%%%%%%%%%%%%%%%%%%%%%%%%%%%%%%
\begin{frame}[label=substitution]
\frametitle{Introduction}
\framesubtitle{National Income and Product Accounts (NIPA)} 

\begin{itemize}
\item Before the 90's, NIPA used a fixed-base Laspeyres quantity index to measure output growth\pause\medskip

\item The permanent decline in the relative price of investment goods induced the so-called \blue{\bf substitution bias}
\item[]{\small The Bureau of Economic Analysis (BEA) was underestimating growth rates}

%\hyperlink{substitutionbias}{\beamergotobutton{example}}
\pause\medskip
	
\item BEA then moved to a \blue{\bf %Fisher-ideal 
chained quantity index} to measure  growth\smallskip
\begin{itemize}
\item Compute a Fisher-ideal index for contiguous periods\smallskip
\item Chain them to compute a real GDP series
\end{itemize}
\pause\medskip

\item \blue{\bf A Fisher-ideal index is  $\simeq$ to a Divisia index}
\item[] {\small In practice, NIPA chain a Divisia index}\pause\medskip

\item \blue{\bf This paper suggests a rational for this change}

\end{itemize}
\end{frame}


%%%%%%%%%%%%%%%%%%%%%%%%%%%%%%%%%%%%%%%%%%%%%%%%%%%%%%%%%%
\begin{frame}[label=thispaper]
\frametitle{Introduction} 
\framesubtitle{Dynamic General Equilibrium (DGE) and Welfare} 

\begin{itemize}
	
	%\item We provide a rational for output growth measurement in NIPA \smallskip
	
	\item Macroeconomists study the behaviour of aggregates by means of DGE\medskip
	
	\begin{itemize}
	\item Discipline is achieved by asking models to replicate NIPA figures\medskip
	\item Models are then used as artificial labs where policies are quantitatively evaluated by their effects on economic growth and welfare
	\end{itemize}\medskip\pause
	
	\item But, \blue{\bf\small{is the output growth rate in NIPA a welfare measure?}}\medskip\pause
	
	\item The answer is YES {\footnotesize (a qualified YES)}:\smallskip
	\blue{\bf\small
	\item[] When NIPA's methodology is applied to DGE economies,	
	\item[] the growth rate of real output measures welfare gains
	}\medskip
	\item[] {\small The argument is true even when  households  are heterogeneous}
	
	%\item[] {\footnotesize See Licandro, Ruiz-Castillo and Duran (2002) and Duran and Licandro (2022)}
		
\end{itemize}
\end{frame}


%%%%%%%%%%%%%%%%%%%%%%%%%%%%%%%%%%%%%%%%%%%%%%%%%%%%%%%%%%
\begin{frame}
\frametitle{Introduction} 
\framesubtitle{Strategy} 

	\begin{itemize}
	\item In DGE models intertemporal preferences depend on a consumption flow, present and future\pause\medskip
	 
	 \item But, NIPA aggregate final demand {\footnotesize(current consumption and investment)}
	\item[]{\footnotesize (Preferences and future consumption are not observed)}\pause\medskip
	 
	\item By means of the \blue{\bf\small Bellman equation}, preferences can be represented as a function of current consumption and investment
	
	\pause\medskip
	\item {Index number theory} is applied to this representation of preferences
	
	{\small In particular, a \blue{\bf Fisher-Shell (FS)} index}\pause\medskip
	
	\item In this framework, \blue{\bf\small the FS index is equal to a Divisia index}\pause\medskip
	
	\item Since NIPA approximate well a Divisia index\pause\smallskip
	
	$\Rightarrow$ \blue{\bf\small Output growth in NIPA is welfare based}
	\end{itemize}
\end{frame}

%%%%%%%%%%%%%%%%%%%%%%%%%%%%%%%%%%%%%%%%%%%%%%%%%%%%%%%
\part{DGE, Bellman Representation and  Fisher-Shell } 
\frame{\partpage} 
\section{DGE, Bellman Representation and  Fisher-Shell }
%%%%%%%%%%%%%%%%%%%%%%%%%%%%%%%%%%%%%%%%%%%%%%%%%%%%%%%

%%%%%%%%%%%%%%%%%%%%%%%%%%%%%%%%%%%%%%%%%%%%%%%%%%%%%%%%%%
\begin{frame}
\frametitle{Dynamic General Equilibrium}
\framesubtitle{Social Planner}

\begin{itemize}
\item[]
A DGE model usually takes the form of a social planner problem% welfare at equilibrium is
\begin{equation*}
v(k_t)=\max\int_{t}^{\infty } U(c_{s})\,\,\text{e}^{-\rho (s-t)}\,\,\text{d}s
\end{equation*}
{\small s.t. the Production Possibility Frontier (PPF)}
$$
(c_t,x_t)\in\Gamma(k_t)
$$
{\small and the capital accumulation law}
$$
\dot k_t = x_t
$$
{\footnotesize $U(.)$ and $\Gamma(.)$ s.t. a locally stable equilibrium path exists and is unique}\medskip
\pause

{\small 
\begin{itemize}
\item \blue{\bf Preferences depend on the flow of consumption}\pause\medskip

\item \blue{\bf $v(k_t)$ measures the value of capital }
\end{itemize}
}

\end{itemize}

\end{frame}

%%%%%%%%%%%%%%%%%%%%%%%%%%%%%%%%%%%%%%%%%%%%%%%%%%%%%%%%%%
\begin{frame}
\frametitle{Dynamic General Equilibrium}
\framesubtitle{Office of National Statistics  (ONS)}

\begin{itemize}

\item In this DGE there is an Office for National Statistics  (ONS)\pause\medskip

\item \blue{\bf\small Aim}: Build an index of output growth that reflects changes in welfare% using observables 

\pause\medskip
\item \blue{\bf\small Problems}:\pause\medskip
\begin{itemize}
\item Preferences and future consumption are not observable\pause\medskip
\item Preferences are not univocally represented% by a welfare function $v(k)$
\pause\medskip
\end{itemize}
\item Like in NIPA, the ONS only observes price and quantities of current consumption and current investment


\end{itemize}
\end{frame}

%%%%%%%%%%%%%%%%%%%%%%%%%%%%%%%%%%%%%%%%%%%%%%%%%%%%%%%%%%
\begin{frame}
\frametitle{Bellman Representation}
\framesubtitle{Bellman Equation}

\begin{itemize}

\item The Bellman equation is
\begin{equation*}
\rho v(k_{t})= \max_{(c,x)\in\Gamma(k_t)}\
 \underbrace{ U(c) + v'(k_t) \,\,x}_{w_{t}(c,x)}
\end{equation*}
 {\small $x=\dot {k}$ is net investment}\medskip

\item \blue{\bf\small The return to capital = }%\vspace{-.3cm}

\hspace{2cm}\blue{\bf\small consumption utility + value of net investment }\pause\medskip

\item  \blue{\bf\small Bellman representation of preferences}
\item[]\blue{\small $w_{t}(c,x)$  represents the same preferences but in the domain $(c,x)$ }\medskip\pause

\item {\small Instead of maximising the value of capital, we maximise the return to capital}

\end{itemize}

\end{frame}

%%%%%%%%%%%%%%%%%%%%%%%%%%%%%%%%%%%%%%%%%%%%%%%%%%%%%%%%%%
\begin{frame}
\frametitle{Bellman Representation}
\framesubtitle{Household Problem}

\begin{itemize}

\item Household welfare results from solving at any $t$
\begin{equation*}
\max_{c,x}\,\,\, \underbrace{U(c) +v'(k_t) \,\,x}_{w_{t}(c,x)}
\end{equation*}
subject to
$$c+p_tx=m_{t}$$

\item[] Households take income $m_t$ and prices $p_t$ as given

\item[]{\small (the consumption good is the numeraire)}\medskip\pause

\item Notice that the Bellman representation \blue{\bf $w_t(c,x)$ \small is time dependent}

\item[] {\small due to changes in the shadow value of capital}\smallskip

\item[] {\footnotesize (Over time the same household changes her valuation of $c$ vs $x$ trade-off)} \medskip\pause

\item \blue{\bf\small Fisher-Shell indices} make welfare comparisons under changing preferences

\end{itemize}

\end{frame}

%%%%%%%%%%%%%%%%%%%%%%%%%%%%%%%%%%%%%%%%%%%%%%%%%%%%%%%%%%
\begin{frame}
\frametitle{Bellman Representation}
\framesubtitle{Household and Social Planner Problems}

\vspace{-4.5cm}
\begin{figure}
\begin{center}
\includegraphics[width=.95\textwidth]{recursive-graph1.pdf}
\end{center}
\vspace{-5.5cm}
{\scriptsize Bellman representation of references $w(x,c)$, PPF and budget constraint}
%\caption{Evaluation using time $t$ preferences}
\label{fig1}
\end{figure}
\end{frame}

%%%%%%%%%%%%%%%%%%%%%%%%%%%%%%%%%%%%%%%%%%%%%%%%%%%%%%%%%%
\begin{frame}
\frametitle{Fisher-Shell Quantity Index}
\framesubtitle{Money Metric}


\begin{itemize}
\item Associated to the Bellman representation of preferences, there are\medskip

\begin{itemize}
\item \blue{\bf\small Indirect utility function}
\begin{equation*}
u_{t}(m_{t},p_t)=\max_{c+p_tx\leq m_{t}}w_{t}(c,x)
\end{equation*}\pause%\bigskip

\item \blue{\bf \small Expenditure function}
\begin{equation*}
e_{t}(u_{t},p_t)=\min_{w_{t}(c,x)\geq u_{t}}c+p_tx
\end{equation*}\pause
\end{itemize}

\item \blue{\bf \small Money metric utility}\medskip

\begin{itemize}
\item For any $(c,x)$, $e_t$ measures the income required at prices $p_t$ to attaint utility $w_t(c,x)$\smallskip\pause

\item Since $e_t$ is increasing in $w_t$, for a given $p_t$,  $e_t(w_t(c,x),p_t)$ represents the same preferences as $w_t(c,x)$\pause\medskip

\item$e_t(w_t(c,x),p_t)$  is a  \blue{\bf money metric representation }of the underlying preferences $w_t(c,x)$ at given prices $p_t$

\end{itemize}

\end{itemize}
\end{frame}

%%%%%%%%%%%%%%%%%%%%%%%%%%%%%%%%%%%%%%%%%%%%%%%%%%%%%%%%%%
\begin{frame}
\frametitle{Fisher-Shell Quantity Index}
\framesubtitle{Money Metric}

\begin{itemize}

\item In order to compare $u_{t}(m_{t},p_t)$ to $u_{t+h}(m_{t+h},p_{t+h})$

\item[] let us define a \blue{hypothetical tomorrow's income}
\blue{\begin{equation*}
\boxed{
\hat{m}_{t+h}=e_{t}\Big(u_{t}(m_{t+h},p_{t+h}),p_t\Big)
}
\end{equation*}}\pause

	\item It uses today's representation of preferences $u_t(.)$ to evaluate \smallskip
	
	the utility of tomorrow's income $m_{t+h}$ at tomorrow's prices $p_{t+h}$\pause\medskip
	
	\item Using today's representation of preferences $e_t(.)$\smallskip
	
	$\hat m_{t+h}$ measures the level of income needed tomorrow\smallskip
	
	to attain this utility level at today's prices $p_t$\pause\medskip
	
	\item \blue{\bf\small $\hat m_{t+h}$ is a money metric measure}\smallskip
	\item[]{\footnotesize The level of income today that makes households indifferent btw the current and the future situations}

\end{itemize}
\end{frame}

%%%%%%%%%%%%%%%%%%%%%%%%%%%%%%%%%%%%%%%%%%%%%%%%%%%%%%%%%%
\begin{frame}
\frametitle{Fisher-Shell Quantity Index}

\begin{figure}
\begin{center}
\includegraphics[width=.6\textwidth]{omar-graph1.pdf}
\end{center}
\caption{Evaluation using time $t$ preferences}
\label{fig1}
\end{figure}
\end{frame}

%%%%%%%%%%%%%%%%%%%%%%%%%%%%%%%%%%%%%%%%%%%%%%%%%%%%%%%%%%
\begin{frame}
\frametitle{Fisher-Shell Quantity Index}

\begin{itemize}
\item The \blue{\bf Fisher-Shell quantity index} is defined as
\begin{equation*}
\boxed{
\text{FS}_{t}=\frac{1}{m_{t}}\left. \frac{d\hat{m}_{t+h}}{dh}\right\vert _{h=0}
}
\end{equation*}\pause

\item \blue{\bf\small FS$_t$ is an equivalent variation measure}
\item[]{\footnotesize The change in income today that makes the  household indifferent between its current and its future situation}
\pause\medskip

\item The FS quantity index is the growth rate of the factor $\hat m_{t+h}$ 

\end{itemize}

\end{frame}

%%%%%%%%%%%%%%%%%%%%%%%%%%%%%%%%%%%%%%%%%%%%%%%%%%%%%%%%%%
\begin{frame}
\frametitle{Fisher-Shell Quantity Index}
\framesubtitle{Divisia Index and NIPA Growth Rate}

\blue{\sc Main Proposition}\vspace{-.2cm}

\begin{itemize}
\item \blue{\bf The Fisher-Shell quantity index is equal to a Divisia index}\hspace{.2cm}
\begin{equation*}
\boxed{\text{FS}_{t}=\underbrace{\big(1-s_t\big)\,\,\frac{\dot{c}%
_{t}}{c_{t}}+s_t\,\,\frac{\dot{x}_{t}}{x_{t}}}_{{\text {Divisia index}}}}
\hspace{1cm}s_t=\frac{p_t x_t}{m_t}
\end{equation*}\pause%\vspace{-.3cm}

\item[] {\small To compute the FS index, the ONS only needs to know quantities and prices of current and past consumption and investment}\medskip

\item Since the growth rate in NIPA approximates well a Divisia index\medskip

\item[]$\Rightarrow$ \blue{\bf NIPA delivers a welfare based measure of output growth}

\end{itemize}
%{\footnotesize
%At the BGP of the two-sector AK economy 
%$$
%\text{FS}_t = 
%\boxed{
%g = (1-s) \alpha\gamma + s \gamma 
%}
%\in (\alpha\gamma,\gamma)
%$$
%where $\gamma$ and $s$ depend on preferences and technology parameters}

\end{frame}

%%%%%%%%%%%%%%%%%%%%%%%%%%%%%%%%%%%%%%%%%%%%%%%%%%%%%%%%%%
\begin{frame}
\frametitle{Fisher-Shell Quantity Index}
\framesubtitle{Chained Indices and Welfare}

\begin{itemize}
\item We have shown that the FS index measures gains in welfare around $t$ from the perspective of time $t$ preferences\medskip\pause
\item \blue{\bf\small Chained index}:\\
When comparing two distant times $t$ and $s$, $s > t$, we chain the instantaneous FS indices
$$
\Gamma_{t,s} = \text{e}^{\, \int_t^s \text{FS}_v \text{d} v}\, .
$$
\item[] \blue{\bf\small {A chained index between two distant times $t$ and $s$ is welfare based}}
%\medskip\pause

%\item But, it is different from a Fisher-Shell index directly comparing $s$ to $t$\\ \blue{\small (Baqaee and Burstein, QJE, 2023)}
%$$
%\frac{\hat{m}_{t,s}}{m_t} 
%\ \ \ \ \text{where}\ \ \ \
%\hat{m}_{t,s}=e_{s}\Big(u_{s}(m_{s},p_{s}),p_t\Big)
%$$
%{\footnotesize comparing past income in $t$ tothe hypothetical income in $s$ evaluated using current preferences but past prices}
\end{itemize}

\end{frame}

%%%%%%%%%%%%%%%%%%%%%%%%%%%%%%%%%%%%%%%%%%%%%%%%%%%%%%%
\part{General Framework} 
\frame{\partpage} 
\section{General Framework} 
%%%%%%%%%%%%%%%%%%%%%%%%%%%%%%%%%%%%%%%%%%%%%%%%%%%%%%%


%%%%%%%%%%%%%%%%%%%%%%%%%%%%%%%%%%%%%%%%%%%%%%%%%%%%%
\begin{frame}
\frametitle{General Framework}
\framesubtitle{Recursive Preferences}

{\small
\begin{itemize}
\item\blue{\bf\small The equality between the FS and Divisia index holds for a DGE model with general recursive preferences}\pause\medskip
\item { Preferences} are represented by the \blue{\bf\small recursive welfare function} $U$%
\begin{equation*}
\frac{d}{dt}U(_{t}C)=-f(c_{t},U(_{t}C))
\end{equation*}

\begin{itemize}

\item[] $_{t}C=(c_{t+s})_{s=0}^\infty$ is a consumption path, 
$f_{1}>0$ and $f_{2}<0$ 
\end{itemize}\pause\medskip
\item \blue{\bf\small Quasi-concave technology}: 
$\{c_{t},x_{t}\}\in \Gamma (\Theta_t,k_{t})$,\ \ \ \ {\footnotesize $\Theta_t$ vector of parameters}\pause\medskip
\item $\{c_{s},x_{s}\}_{s\geq t}$  maximizes $U(_{t}C)$ subject to the feasibility
constraint\medskip\pause
\item Bellman equation representation
\begin{equation*}
0=\max_{(c,x)\in \Gamma (\Theta_t,k_{t})} \underbrace{f\big(c,v(k_{t},\Theta_t)\big) + v_1(k_{t},\Theta_t)x}_{w_t(c,x)} + v_2(k_{t},\Theta_t)\dot\Theta_t
\end{equation*}

\end{itemize}
}

\end{frame}

%%%%%%%%%%%%%%%%%%%%%%%%%%%%%%%%%%%%%%%%%%%%%%%%%%%%%
\begin{frame}
\frametitle{General Framework}
\framesubtitle{Household heterogeneity}

\begin{itemize}
\item The result also holds for heterogeneous households\\
{\footnotesize (in preferences, assets, income)} \pause\medskip
\item[] \blue{\bf\small The Fisher-Shell index is, indeed, equal to the Divisia index}\pause
\item[]{\footnotesize (even if equilibrium may be different from the representative household one)}\pause\medskip

\item \blue{\bf\small Key assumptions}\smallskip
\begin{itemize}
\item Money metric utility: income is the metrics\smallskip
\pause\smallskip
\item The Bellman representation of preferences is quasi-linear on investment
\item[] (Gorman aggregation conditions hold)\pause\medskip
\end{itemize}
\item The Fisher-Shell index implicitly assumes that \pause\smallskip

\begin{itemize}

\item The \blue{\bf\small social welfare function is utilitarian}
\item[] with households weighted \blue{\bf\small proportionally to their income}
\end{itemize}

\end{itemize}

\end{frame}

%%%%%%%%%%%%%%%%%%%%%%%%%%%%%%%%%%%%%%%%%%%%%%%%%%%%%%%%%%
\begin{frame}
\frametitle{General Framework}
\framesubtitle{Extensions} 


	\begin{itemize}
	\item There is only one nondurable consumption good\smallskip\pause
	
	{\small Easy to extend to a basket of nondurable consumption goods and services}\medskip\pause
	
	\item There is only one investment good\smallskip\pause
	
	{\small Easy to extend to many investment and durable consumption goods}
	
	{\small Less easy to different forms of human and intangible capital investment}\medskip\pause
	
	\item Prices reflect marginal productivities\smallskip\pause
	
	{\small Easy to extend to frameworks where prices are distorted}
	\medskip\pause
	
	\item DSGE models
	
	{\small Adding uncertainty to the model shouldn't be an issue}
	\medskip\pause

	\item Dealing with externalities is still an open issue
	
	
	
	%\item Dealing with the non-market economy is also an open issue
	
	\end{itemize}
\end{frame}



%%%%%%%%%%%%%%%%%%%%%%%%%%%%%%%%%%%%%%%%%%%%%%%%%%%%%%%
\part{Two-Sector AK Model} 
\frame{\partpage} 
\section{Two-Sector AK Model} 
%%%%%%%%%%%%%%%%%%%%%%%%%%%%%%%%%%%%%%%%%%%%%%%%%%%%%%%

%%%%%%%%%%%%%%%%%%%%%%%%%%%%%%%%%%%%%%%%%%%%%%%%%%%%%%%%%%
\begin{frame}[label=embodied]
\frametitle{Two-Sector AK Model} 
\framesubtitle{Investment Specific Technical Progress}


\begin{itemize}
\item Following the observations by Gordon (1990), statistical offices introduced quality corrections in price indices\medskip
\item[] $\Rightarrow$ New facts emerged\smallskip

\begin{itemize}
\item {\bf Fact 1}: The price of investment declines relative to the price of non-durable consumption
\hyperlink{relativeprice}{\beamergotobutton{US prices}}\pause\medskip

\item {\bf Fact 2}: The investment share is roughly constant, but \\
 the investment ratio is permanently increasing
	\hyperlink{shares}{\beamergotobutton{US investment ratio}}
\pause\medskip
	
\item {\bf Fact 3}: $\Rightarrow$ Investment grows faster than consumption\pause\medskip

\end{itemize}\medskip

%\item Why? Technical progress is primarly embodied in capital goods
%\item[] {\footnotesize (I will refer to this issue later)}\pause\medskip

\item \blue{\bf \small The aggregation of consumption and investment is an issue}\smallskip
\item[] {\small BUT, traditional index number theory, behind national accounts, is static}

\end{itemize}

\end{frame}

%%%%%%%%%%%%%%%%%%%%%%%%%%%%%%%%%%%%%%%%%%%%%%%%%%%%%%%%%%
\begin{frame}
\frametitle{Two-Sector AK Model}
\framesubtitle{Rebelo (1991), Felbermayr-Licandro (2005)} \pause

Let us use a particular example to better understand the welfare properties of the GDP growth rate\medskip

%It replicates the observed decline of investment goods prices and the fact that investment grows faster than consumption\pause
The social planner problem is
\begin{equation*}
v(k_0)=\max\int_{0}^{\infty }\frac{c_{t}^{1-\sigma }}{1-\sigma }\,\,\text{e}^{-\rho t}\,\,\text{d}t
\end{equation*}
s.t.	the PPF
\begin{eqnarray*}
\text{\footnotesize{consumption  technology:}}  & \ \ \ \ 
c_{t}  \ \  = & b_{t}^{\alpha } \ \ \ \ \ \ \ \ \ \ \ \ \ \ \ \ \ \  \ \ \ \ \ \ \ \ \ \ \ \ \ \ \ \ \ \  \ \ \ \ \ \ \ \ \ \ \ \ \ \ \ \ \ \ \\
\text{\footnotesize{investment  technology:}} \ \ & \ \ \ \ 
 \dot k_{t} \ \   = & A(k_{t}-b_{t})-\delta k_{t}  \ \ \ \ \ \ \ \ \ \ \ \ \ \ \ \ \ \  \ \ \ \ \ \ \ \ \ \ \ \ \ \ \ \ \ \
\end{eqnarray*}

{\footnotesize $k_0 >0$ and $\rho >0$, $\sigma>0$, $\alpha \in (0,1)$,  $\delta >0$,  $A>\rho+\delta$}
\end{frame}

%%%%%%%%%%%%%%%%%%%%%%%%%%%%%%%%%%%%%%%%%%%%%%%%%%%%%%%%%%
\begin{frame}
\frametitle{Two-Sector AK Model}
\framesubtitle{Equilibrium} 

At equilibrium of the two-sector AK model, from the initial time {\footnotesize $t=0$}
%It replicates the observed decline of investment goods prices and the fact that investment grows faster than consumption\pause
\begin{itemize}
\item Capital $k_t$ grows at the constant rate
$$
k_t = k_0\ \text{e}^{\,\gamma t}
\ \ \ \ \text{\small where}\ \ \ \ 
\text{\footnotesize$\gamma = \frac{A-\rho}{1-\alpha(1-\sigma)}$}
\ \  \text{\small and}\ \ 
\text{\footnotesize  $k_0>0$ given}
$$
\pause
\item Consumption $c_t$ grows at the constant rate {\footnotesize $\alpha\gamma < \gamma$}
$$
c_t = \text{e}^{\,\alpha\gamma t}
$$\pause\vspace{-.5cm}

\item GDP growth, as given by the Divisia index, measures welfare gains
$$
g = (1-s) \alpha\gamma + s \gamma \in(\alpha\gamma,\gamma)
$$

\end{itemize}
\end{frame}

%%%%%%%%%%%%%%%%%%%%%%%%%%%%%%%%%%%%%%%%%%%%%%%%%%%%%%%%%%
\begin{frame}[label=embodied]
\frametitle{Two-Sector AK Model}
\framesubtitle{Evidence on Investment Specific Technical Progress} 

The two-sector AK model replicates the evidence on investment specific technical progress:
\begin{itemize}

\item {\bf Fact 1}: The price of investment declines relative to the price of non-durable consumption\medskip

\item {\bf Fact 2}: The investment share is roughly constant, but \\
 the investment ratio is permanently increasing
\medskip
	
\item {\bf Fact 3}: $\Rightarrow$ Investment grows faster than consumption


\end{itemize}


\end{frame}


%%%%%%%%%%%%%%%%%%%%%%%%%%%%%%%%%%%%%%%%%%%%%%%%%%%%%%%%%%
\begin{frame}
\frametitle{Two-Sector AK Model}
\framesubtitle{On Money Metric Utility}

\begin{itemize}
\item At equilibrium of the two-sector AK model 
$$
v(k_t) = % \int_t^\infty \frac{c_s^{1-\sigma}}{1-\sigma} \text{ e}^{-\rho (s-t)} \text{ d} t =
B\,  k_t^{\alpha (1-\sigma)}
$$
{\footnotesize  grows at the rate $ \gamma\, \alpha  (1-\sigma)$ different from the Divisia index $g$}\pause\medskip

\item There exists a money metric representation of preferences
{\small
$$
\hat v(k_t)  = C\ v(k_t)^{\frac{g}{\gamma\alpha(1-\sigma)}} 
$$
}
{\footnotesize which grows at rate $g$}\pause\medskip


\item{ The return to assets at the reference time is equal to nominal income}\smallskip%\pause\medskip
\item[] {\small Choose $C$  such that }
{\small
$$
\rho \hat v(k_0) = c_0 + p_0 x_0
$$
}
\end{itemize}
\end{frame}

%%%%%%%%%%%%%%%%%%%%%%%%%%%%%%%%%%%%%%%%%%%%%%%%%%%%%%%%%%
\begin{frame}
\frametitle{Two-Sector AK Model}
\framesubtitle{Investment also Matters}

\begin{itemize}

\item If current and future consumption is all that matters for welfare\smallskip

Should the consumption growth rate 
summarize all what is relevant?\pause\smallskip

The answer is \blue{\bf NO}\pause\medskip

\item  \blue{\bf Net investment also matters} {\small (Weitzman, 1976)}\smallskip

since it reflects welfare gains from postponed consumption\pause
%\item[] (notice that $x$ is net investment) \pause
\medskip

%\item But net and gross output grow at the same rate\medskip

\item  \blue{\bf It is not GDP that matters but Net National Product (NNP)}
\item[] {\footnotesize(in a closed economy national and domestic product are equal)}
\end{itemize}
\end{frame}

%%%%%%%%%%%%%%%%%%%%%%%%%%%%%%%%%%%%%%%%%%%%%%%%%%%%%%%%%%
\begin{frame}
\frametitle{Two-Sector AK Model}
\framesubtitle{Consumption Equivalence}

{\small
\begin{itemize}

%\item \blue{\bf\small Consumption equivalence} %is a compensating variation measure
%\item[] {\small (in terms of the entire consumption path instead of current income)}\pause\medskip

\item $\lambda_h$ is a hypothetical increase in the consumption path that makes  a household indifferent between staying at $t$ or jumping to $t+h$ 
{\footnotesize
$$
\lambda_h^{1-\sigma} v(k_t) = v(k_{t+h})
\ \ \ \ \ \text{or}\ \ \ \ \ 
\boxed{\lambda_h= \left(\frac{v(k_{t+h})}{ v(k_t) }\right)^{\frac{1}{1-\sigma}}
=
\left(\frac{ k_{t+h}}{k_t}\right)^{\alpha}
}
$$
}\pause
\item The growth rate associated to the compensating variation measure is
$$
g^{ce} =\left. \frac{\text{d}\lambda_h}{\text{d} h} \right|_{h=0}= \alpha\gamma
$$\pause
\item\blue{\bf\small The growth rate of consumption is a (consumption equivalent) welfare based measure too}
\end{itemize}
}
\end{frame}

%%%%%%%%%%%%%%%%%%%%%%%%%%%%%%%%%%%%%%%%%%%%%%%%%%%%%%%%%%
\begin{frame}
\frametitle{Two-Sector AK Model}
\framesubtitle{Paradox: Endowment Economy }

A true quantity index measures output
growth \blue{\bf conditional on both preferences and technology}\vspace{-.3cm}\pause
\begin{itemize}
\item Assume a \blue{\bf two-sector AK economy} such that at equilibrium\smallskip
\begin{itemize}
\item The growth rate of investment is 6\%; the investment share is 20\%\smallskip
\item The growth rate of consumption is 2\%\smallskip
\item The Divisia index delivers a 2.8\% growth rate
\end{itemize}\pause\medskip
\item Consider an \blue{\bf endowment economy} with exactly the same preferences \smallskip
\begin{itemize}
\item The same equilibrium consumption path (mana from haven)\smallskip
\item The Divisia index measures output growing at 2\%
\end{itemize}\pause\medskip
\item Why two economies with identical preferences and the same consumption path do not grow at the same rate?\pause\smallskip
\begin{itemize}
\item Current income (the norm) is defined differently\pause\smallskip
\begin{itemize}
\item In the AK model, investments are needed for permanently grow\pause\smallskip
\item In the endowment economy, God provides
\end{itemize}
\end{itemize}
\end{itemize}

\end{frame}

%%%%%%%%%%%%%%%%%%%%%%%%%%%%%%%%%%%%%%%%%%%%%%%%%%%%%%%%%%
\begin{frame}
\frametitle{Two-Sector AK Model}
\framesubtitle{A Word of Caution}


\begin{itemize}
\item It is well-known in endogenous growth theory that 
\item[] \blue{\bf\small there is an optimal growth rate}\pause\medskip
\item Suppose the two-sector AK economy is at (the optimal) equilibrium
\item[] but, a government introduces an incentive to increase growth\pause\medskip
\item At time zero, the economy will suffer a reduction in welfare\pause\medskip
\item From them, welfare will be growing at a larger rate (Divisia index) 
\item[] {\small (at some point in time consumption will be larger in the distorted economy)}\pause\medskip
\item Unfortunately, \blue{\bf\small NIPA does not measure capital loses}
\end{itemize}

\end{frame}

%%%%%%%%%%%%%%%%%%%%%%%%%%%%%%%%%%%%%%%%%%%%%%%%%%%%%%%
\part{Conclusion} 
\frame{\partpage} 
\section{Conclusion} 
%%%%%%%%%%%%%%%%%%%%%%%%%%%%%%%%%%%%%%%%%%%%%%%%%%%%%%%

%%%%%%%%%%%%%%%%%%%%%%%%%%%%%%%%%%%%%%%%%%%%%%%%%%%%%%%%%%
\begin{frame}
\frametitle{Conclusion}

\begin{itemize}

\item The growth rate in NIPA is a welfare measure\medskip
\begin{itemize}
\item For a large family of preference orders and technological environments\smallskip
\item For heterogeneous household economies\bigskip
\end{itemize}
\item It's not only consumption that matters for output growth:\medskip

Investment does also matter

\end{itemize}
\end{frame}




%%%%%%%%%%%%%%%%%%%%%%%%%%%%%%%%%%%%%%%%%%%%%%%%%%%%%%%
\part{Open Economy} 
\frame{\partpage} 
\section{Open Economy} 
%%%%%%%%%%%%%%%%%%%%%%%%%%%%%%%%%%%%%%%%%%%%%%%%%%%%%%%


%
%
\begin{frame}
\frametitle{Open Economy}
\framesubtitle{Bellman representation of preferences}
%
%
{\small 
\begin{itemize}
\item In a small open economy, the PPF is expanded thanks to trade
\item Goods can be traded at  international prices  $(1,p^*_{t})$\\
\blue{\footnotesize decoupling production from final demand}

\begin{itemize}
\item  $y_{c,t}$ and $y_{x,t}$ are produced quantities\smallskip
\item    $c_{t}$ and $x_{t}$ are  consumption and net investment
\end{itemize}\smallskip

\item \blue{The representative household holds $a_{t}$ units of capital abroad}

\item The production possibility frontier is give by 
\[
(y_{c,t},y_{x,t})  \in  \Gamma(k_{t})%,  A_{t}), 
\quad\quad
\dot k_{t} =  x_{d,t} \quad \text{and} \quad  \blue{\dot {a}_{t} = x_{f,t}}
\]
\[
c_{t} +  \underbrace{\blue{\ \ \eta_t p^*_{t} x_{dt} }}_{p_t x_{dt}}
 + \underbrace{\blue{\ \ p^*_{t}  x_{ft} \ \ }}_{\text{\scriptsize net exports}} =  
y_{ct}  + p^*_{xt} y_{xt} + \underbrace{\blue{ r_t  p^*_{t}a_{t} }}_{=\ 0}
\nonumber
\]
\item[]{\footnotesize $\eta_{t}$ represents domestic price distortions}
\item {\footnotesize When $r_t=0$, $x_f$ is the trade balance}

\end{itemize}
}
\end{frame}


%
%
\begin{frame}
\frametitle{Open Economy}
\framesubtitle{Bellman representation of preferences}
%
%
{\small 
\begin{itemize}
\item The Bellman equation representation of the problem is
$$
\rho v(k_t,a_{t}) %,X_{t})   
=   \max_{c, x_d,x_f}\  \underbrace{u(c) + v_{1} (.) x_d +  \blue{v_{2} (.) x_f }}_{w_{t}(c,x_d,\blue{x_f})}
%+ v_3(.)\dot X_{t}
$$
$$
\text{s.t.} \quad  c + p_{t}\, x_d + \blue{p^{*}_{t}\, x_f} = m_{t} ,
$$

\item $w_{t}(c,x_d,x_f) $ is the \blue{Bellman representation of preferences} over current consumption and current net investment (domestic + foreign)

\item Associated to the indirect utility and expenditure functions
\[
u_{t}(m_{t}, \underbrace{p_{x,t},\blue{p^*_{x,t}}}_{\text{\bf p}_{t}})  \equiv \max_{c + p_{t}\, x_d + \blue{p^{*}_{t}\, x_f}} = m_{t} w_{t}(c,x_d,\blue{x_f})
\]\vspace{-.3cm}
\[\ \ \ 
e_{t}(u_{t},\blue{\text{\bf p}_{t}}) 
\equiv \min_{w_{t}(c,x_d,\blue{x_f})= u_{t}}  c + p_{x,t}\, x_d + \blue{p^{*}_{x,t}\, x_f}
\]

\item In an open economy, \blue{the additional term is the trade balance $x_f$ priced at $p^{*}_x$}

\end{itemize}
}
\end{frame}

%
%
\begin{frame}
\frametitle{Open Economy}
\framesubtitle{Economic Growth and Welfare Gains}
%
%
{\small 
\begin{itemize}

\item Let us compare income at $t$ to income at $t+h$

\item To compare income $m_t$ at prices {\bf p}$_t$ to income $m_{t+h}$ at prices {\bf p}$'_{t+h}$, we use time $t$ preferences to create an artificial income
$$
\widehat m_{t+h} = e_t\Big(u_t(m_{t+h},\text{\bf p}_{t+h}),\text{\bf p}_t\Big)
$$


\item[]  {\bf Definition}: $\widehat m_{t+h}$ is the level of income tomorrow 
necessary to attain utility $u_t(m_{t+h}, \text{\bf p}_{t+h})$ {\footnotesize (future income at  future prices)} at country current prices $\text{\bf p}_t$ 

\item[] {\footnotesize (we are correcting by changing in prices: purchasing power equivalence)} \medskip

\item The Fisher-Shell index measuring the welfare gains btw {\footnotesize $t$ and $t+h$} is
\[
\gamma^{\text{\text{FS}}}_{t} \equiv 
%=
\frac{1}{m_t} \left. \frac{\text{d} \widehat m_{t+h}}{\text{d} h} \right|_{h= 0}
\]
\item[] {\footnotesize As if  a household evaluates the welfare gains of moving from $t$ to {\scriptsize $t+h$}}


\end{itemize}
}
\end{frame}

%
%
\begin{frame}
\frametitle{Open Economy}
\framesubtitle{Economic Growth and Welfare Gains}
%
%
{\small 
\begin{itemize}

\item {\bf Proposition:} The Fisher-Shell index is equal to a Divisia index
\[ \hspace{-1.5cm}
\boxed{
\gamma^{\text{\text{FS}}}_{t} = 
s_{ct} \  \frac{\dot c_t}{c_t } + s_{x_d t} \ \frac{\dot x_{dt}}{x_{dt}} + \blue{s_{x_f t} \ \frac{\dot x_{ft}}{x_{ft}}}
}
\ \ \ \  \text{\scriptsize s.t.\ \ $s_c + s_{x_d} + \blue{s_{x_f} }=1$}
\]

\item Important 
\begin{itemize}

\item \blue{The trade balance $x_f$ is an investment in foreign capital}
\item[] {\footnotesize that must be deflated using an investment price index}\smallskip

\end{itemize}\medskip

\item However, NA deflate exports and imports by their own deflators
\item[] \blue{NA measure growth in domestic production, not in welfare}\medskip

\item \blue{\footnotesize\bf Conveniently adjusted, growth rates in NA are welfare based}\medskip

\end{itemize}
}
\end{frame}

%%%%%%%%%%%%%%%%%%%%%%%%%%%%%%%%%%%%%%%%%%%%%%%%%%%%%%%
\part{Appendix} 
\frame{\partpage} 
\section{Appendix} 
%%%%%%%%%%%%%%%%%%%%%%%%%%%%%%%%%%%%%%%%%%%%%%%%%%%%%%%

%%%%%%%%%%%%%%%%%%%%%%%%%%%%%%%%%%%%%%%%%%%%%%%%%%%%%%%%%%
\begin{frame}[label=relativeprice]
\frametitle{Relative Price of Equipment}
\vspace{-.4cm}
\begin{figure}
\begin{center}
\includegraphics[width=.9\textwidth]{prices.pdf}
\end{center}
\vspace{-1.5cm}
\label{fig1}
\end{figure}
{\small Fernald (US-NIPA):  Equipment plus consumer durables price, relative to price of other business output}
\hyperlink{embodied}{\beamergotobutton{back}}

\end{frame}

%%%%%%%%%%%%%%%%%%%%%%%%%%%%%%%%%%%%%%%%%%%%%%%%%%%%%%%%%%
\begin{frame}[label=shares]
\frametitle{Investment to GDP Share and Ratio}
\vspace{-.4cm}
\begin{figure}
\begin{center}
\includegraphics[width=.9\textwidth]{shares}
\end{center}
\vspace{-1.5cm}
\label{fig1}
\end{figure}
{\small Fernald (US-NIPA):  Equipment and consumer durables as a share of business output}
\hyperlink{embodied}{\beamergotobutton{back}}

\end{frame}


\end{document}

%%%%%%%%%%%%%%%%%%%%%%%%%%%%%%%%%%%%%%%%%%%%%%%%%%%%%%%%%%
\section{Introduction}
\subsection{ }

%%%%%%%%%%%%%%%%%%%%%%%%%%%%%%%%%%%%%%%%%%%%%%%%%%%%%%%%%
%\begin{frame}
%\frametitle{NIPA} \pause

%Growth measurement in National Income and Product Accounts (NIPA)
%\begin{itemize}

%\item {\bf GDP} measures  
%\blue{\bf\small final demand} of goods \& services in current \$ or £
%\pause\medskip

%\item {\bf Real GDP growth} is measured by the mean of \blue{\bf\small chained indices}
%\pause\medskip

%\item In practice, different items are aggregated by the mean of an Ideal Fisher index

%\item In facts, GDP growth rate is roughly equal to a {\bf Divisia Index}%\medskip

%$$
%	{\underbrace{\,\,\,g^{F}\,\,\,}_{\text{Fisher-chain}} \approx 
%	\underbrace{\,\,\,g^{D}\,\,\,}_{\text{Divisia}} =
%	\underbrace{\,\,\,s_c \,\,g_c\,\,\,}_{\text{consumption}}+ \underbrace{\,\,\,s_i\,\, g_i\,\,\,}_{\text{investment}}} \hspace{.2cm}{\small{\text{}}}\hspace{.2cm}{\small \underbrace{\ \ s_c+s_i\ \ }_{\text{GDP shares}}=1}
%$$
%\item[] {\small (a weighted average of the growth rate of the final demand components)}\pause\medskip

%\item This paper is about the theoretical foundations of this methodology% are still missing

%\end{itemize}
%\end{frame}

%%%%%%%%%%%%%%%%%%%%%%%%%%%%%%%%%%%%%%%%%%%%%%%%%%%%%%%%%%
\begin{frame}
\frametitle{Motivating Example: Two-Sector AK Model}
\framesubtitle{Rebelo (1991), Felbermayr-Licandro (2005)} \pause

%It replicates the observed decline of investment goods prices and the fact that investment grows faster than consumption\pause
The social planner problem is
\begin{equation*}
v(k_0)=\max\int_{0}^{\infty }\frac{c_{t}^{1-\sigma }}{1-\sigma }\,\,\text{e}^{-\rho t}\,\,\text{d}t
\end{equation*}
s.t.	
\begin{eqnarray*}
\text{\footnotesize{consumption  technology:}}  & \ \ \ \ 
c_{t}  \ \  = & b_{t}^{\alpha } \ \ \ \ \ \ \ \ \ \ \ \ \ \ \ \ \ \  \ \ \ \ \ \ \ \ \ \ \ \ \ \ \ \ \ \  \ \ \ \ \ \ \ \ \ \ \ \ \ \ \ \ \ \ \\
\text{\footnotesize{investment  technology:}} \ \ & \ \ \ \ 
 \dot k_{t} \ \   = & A(k_{t}-b_{t})-\delta k_{t}  \ \ \ \ \ \ \ \ \ \ \ \ \ \ \ \ \ \  \ \ \ \ \ \ \ \ \ \ \ \ \ \ \ \ \ \
\end{eqnarray*}

{\footnotesize $k_0 >0$ and $\rho >0$, $\sigma>0$, $\alpha \in (0,1)$,  $\delta >0$,  $A>\rho+\delta$}
\end{frame}

%%%%%%%%%%%%%%%%%%%%%%%%%%%%%%%%%%%%%%%%%%%%%%%%%%%%%%%%%%
\begin{frame}
\frametitle{Motivating Example: Two-Sector AK Model}
\framesubtitle{Equilibrium} \pause

At equilibrium of the two-sector AK model
%It replicates the observed decline of investment goods prices and the fact that investment grows faster than consumption\pause
\begin{itemize}
\item From the initial time the economy is on its BGP
$$
k_t = k_{0}\, \text{e}^{\gamma t}
\ \ \ \ \text{\small with}\ \ \ \ 
\text{\footnotesize $\gamma = \frac{A-\rho}{1-\alpha(1-\sigma)}$}
$$

\item Consumption grows at the rate 
$$
\alpha\gamma < \gamma
$$\pause\vspace{-.4cm}

\item The value function takes the form
$$
v(k_t) =
B\,  k_t^{\alpha (1-\sigma)}
$$

\end{itemize}
\end{frame}

%%%%%%%%%%%%%%%%%%%%%%%%%%%%%%%%%%%%%%%%%%%%%%%%%%%%%%%%%%
\begin{frame}[label=embodied]
\frametitle{Is it an Interesting Example?} 

It replicates the following facts:
\begin{itemize}
\item {\bf Fact 1}: The price of investment declines relative to the price of consumption
% relative to the price of non durable consumption
\hyperlink{relativeprice}{\beamergotobutton{US prices}}\pause\medskip

%\item {\bf Fact 2}: The investment share is roughly constant
% The investment ratio increases but the investment share is stationary
%	\hyperlink{shares}{\beamergotobutton{US investment ratio}}
%\pause\medskip
	
\item {\bf Fact 2}: Investment grows faster than consumption
\hyperlink{shares}{\beamergotobutton{US shares}}
 \pause\vspace{1cm}

\blue{\bf Aggregation is then an issue}

\end{itemize}
\end{frame}

%%%%%%%%%%%%%%%%%%%%%%%%%%%%%%%%%%%%%%%%%%%%%%%%%%%%%%%%%%
\begin{frame}[label=embodied2]
\frametitle{Is it an Interesting Example?} 

It replicates the following facts:
\begin{itemize}
\item {\bf Fact 1}: The price of investment declines relative to the price of consumption
% relative to the price of non durable consumption
\hyperlink{relativeprice}{\beamergotobutton{US prices}}\pause\medskip

%\item {\bf Fact 2}: The investment share is roughly constant
% The investment ratio increases but the investment share is stationary
%	\hyperlink{shares}{\beamergotobutton{US investment ratio}}
%\pause\medskip
	
\item {\bf Fact 2}: $\Rightarrow$ Investment grows faster than consumption
\hyperlink{shares}{\beamergotobutton{US shares}}
 \vspace{1cm}

\blue{\bf Aggregation is then an issue}

\end{itemize}
\end{frame}

%%%%%%%%%%%%%%%%%%%%%%%%%%%%%%%%%%%%%%%%%%%%%%%%%%%%%%%%%%
\begin{frame}[label=substitution]
\frametitle{Related Changes in NIPA Methodology} \pause

\begin{itemize}
\item Before the 90's, NIPA used a fixed-base Laspeyres quantity index to measure output growth\pause\medskip

\item The permanent decline in the relative price of investment goods induced the so-called \blue{\bf substitution bias}
\item[]{\small The Bureau of Economic Analysis (BEA) was underestimating growth rates}

%\hyperlink{substitutionbias}{\beamergotobutton{example}}
\pause\medskip
	
\item NIPA then moved to a \blue{\bf %Fisher-ideal 
chained quantity indices} to measure  growth
\begin{itemize}
\item Compute a Fisher-ideal index for contiguous periods\smallskip
\item Chain them to compute a real GDP series
\end{itemize}
\pause\medskip


\item \blue{\bf A Fisher-ideal index is  $\simeq$ to a Divisia index}
\item[] {\small In practice, NIPA chain a Divisia index}\pause\medskip


\item \blue{\bf This paper suggests a rational for this change}

\end{itemize}
\end{frame}


%%%%%%%%%%%%%%%%%%%%%%%%%%%%%%%%%%%%%%%%%%%%%%%%%%%%%%%%%%
%\begin{frame}
%\frametitle{Fact 1: Equilibrium Prices} \pause

%The two-sector AK model predicts Fact 1: 
%\begin{itemize}
%\item The  price of investment relative to the price of consumption 
%
%\begin{equation*}
%p_t=%
%\underbrace{\hspace{.9cm}\frac{\alpha b_{t}^{\alpha -1}}{A}\hspace{.9cm}}_{{\text{ratio of marginal productivities}}}
%\end{equation*}\pause

%	\item $p_t$ declines at $(1-\alpha) g$\bigskip\pause

%	{\small  Marginal product of capital grows at different rates in both sectors}
%\end{itemize}
%\end{frame}

%%%%%%%%%%%%%%%%%%%%%%%%%%%%%%%%%%%%%%%%%%%%%%%%%%%%%%%%%%
%\begin{frame}
%\frametitle{Facts 2 and 3: Equilibrium Allocations}

%At equilibrium\pause
%\begin{itemize}
%\item The economy is at its balanced growth path from $t=0$\pause\medskip
%\item The investment share is constant 
%\pause\smallskip

%but, the investment ratio increases (because of prices) 
%\pause\medskip
%\item Investment grows faster than consumption
%$$ {g_c=\alpha g_i}>0$$\pause%\vspace{-.5cm}


%\item[] We are back to the question\medskip

%\item[] \blue{\bf How should we measure output growth?}

%\end{itemize}
%\end{frame}


%%%%%%%%%%%%%%%%%%%%%%%%%%%%%%%%%%%%%%%%%%%%%%%%%%%%%%%%%%
\begin{frame}
\frametitle{Measuring Welfare Changes in Practice}

\begin{itemize}
\item \blue{\bf\small Aim}: Build an index of output growth that reflects changes in welfare% using observables 

\pause\medskip
\item \blue{\bf\small Problems}:\pause\medskip
\begin{itemize}
\item Preferences and foreseen consumption are not observable\pause\medskip
\item Preferences are not univocally represented% by a welfare function $v(k)$
\pause\medskip
\end{itemize}
\item  Like in NIPA, we would like to measure output growth using current consumption, current investment and current prices

\end{itemize}
\end{frame}

%%%%%%%%%%%%%%%%%%%%%%%%%%%%%%%%%%%%%%%%%%%%%%%%%%%%%%%%%%
\begin{frame}
\frametitle{Social Planner}

\begin{itemize}
\item[]
Welfare at equilibrium is
\begin{equation*}
v(k_t)=\max\int_{t}^{\infty }\frac{c_{s}^{1-\sigma }}{1-\sigma }\,\,\text{e}^{-\rho (s-t)}\,\,\text{d}s
\end{equation*}
{\small s.t.	the technological constraints (PPF)}
\begin{eqnarray*}
\text{\footnotesize{consumption  technology:}}  & \ \ \ \ 
c_{t}  \ \  = & b_{t}^{\alpha } \ \ \ \ \ \ \ \ \ \ \ \ \ \ \ \ \ \  \ \ \ \ \ \ \ \ \ \ \ \ \ \ \ \ \ \  \ \ \ \ \ \ \ \ \ \ \ \ \ \ \ \ \ \ \\
\text{\footnotesize{investment  technology:}} \ \ & \ \ \ \ 
 \dot k_{t} \ \   = & A(k_{t}-b_{t})-\delta k_{t}  \ \ \ \ \ \ \ \ \ \ \ \ \ \ \ \ \ \  \ \ \ \ \ \ \ \ \ \ \ \ \ \ \ \ \ \
\end{eqnarray*}
\vspace{-.2cm}

\begin{itemize}
\item \blue{\bf Preferences depend on the flow of consumption}\pause\bigskip

\item \blue{\bf Welfare at equilibrium  is $v(k_t)$, i.e. the value of capital }
\end{itemize}

\end{itemize}

\end{frame}

%%%%%%%%%%%%%%%%%%%%%%%%%%%%%%%%%%%%%%%%%%%%%%%%%%%%%%%%%%
\begin{frame}
\frametitle{Bellman Representation}

\begin{itemize}

\item The Bellman equation is
\begin{equation*}
\rho v(k_{t})= \max_{c,x}\
 \underbrace{\frac{c^{1-\sigma }}{1-\sigma }+v'(k_t) \,\,x}_{w_{t}(c,x)}
\end{equation*}
{\small (maximised under the technological constraints)}

 \blue{\small ${x=\dot {k^{\ }}}$ is net investment}\pause\medskip%\vspace{-.2cm}

\item \blue{\bf\small The return to capital = }%\vspace{-.3cm}

\hspace{2cm}\blue{\bf\small consumption utility + value of net investment }\pause\bigskip

\item \blue{\bf\small The Bellman representation of preferences}\smallskip
\item[] \blue{ $w_{t}(c,x)$  represents preferences in consumption and investment }\medskip\pause

\item {\small From maximising the value of capital to maximise the return to capital}

\end{itemize}

\end{frame}

%%%%%%%%%%%%%%%%%%%%%%%%%%%%%%%%%%%%%%%%%%%%%%%%%%%%%%%%%%
\begin{frame}
\frametitle{Household Problem}

\begin{itemize}

\item Let's apply index number theory to the household problem\medskip

\item At any time $t$, the representative household solves
\begin{equation*}
\max_{c,x}\,\,\, \underbrace{\frac{c^{1-\sigma }}{1-\sigma }+v'(k_t) \,\,x}_{w_{t}(c,x)}
\end{equation*}
subject to
$$c+p_tx=m_{t}$$

\item[] {\small Households take (equilibrium) income $m_t$ and prices $p_t$ as given}

\item[]{\small (the consumption good is the numeraire)}\medskip\pause

\item Notice that \blue{\bf $w_t(c,x)$ \small is time dependent}

\item[] {\small due to changes in the marginal value of capital}

\end{itemize}

\end{frame}

%%%%%%%%%%%%%%%%%%%%%%%%%%%%%%%%%%%%%%%%%%%%%%%%%%%%%%%%%%
\begin{frame}
\frametitle{Household and Social Problems}

\vspace{-4.5cm}
\begin{figure}
\begin{center}
\includegraphics[width=1\textwidth]{recursive-graph1.pdf}
\end{center}
\caption{Evaluation using time $t$ preferences}
\label{fig1}
\end{figure}
\end{frame}


%%%%%%%%%%%%%%%%%%%%%%%%%%%%%%%%%%%%%%%%%%%%%%%%%%%%%%%%%%
\begin{frame}
\frametitle{Indirect Utility and Expenditure Functions}
\begin{itemize}
\item \blue{\bf\small Indirect utility function}
\begin{equation*}
u_{t}(m_{t},p_t)=\max_{c+p_tx\leq m_{t}}w_{t}(c,x)
\end{equation*}\pause%\bigskip

\item \blue{\bf \small Expenditure function}
\begin{equation*}
e_{t}(u_{t},p_t)=\min_{w_{t}(c,x)\geq u_{t}}c+p_tx
\end{equation*}\pause

\item \blue{\bf \small Money metric utility}\medskip

\begin{itemize}
\item% For any $(c,x)$, 
$e_t$ measures the income required at prices $p_t$ to attaint $w_t(c,x)$\smallskip\pause

\item Since $e_t$ is increasing in $w_t$, for a given $p_t$,  $e_t(w_t(c,x),p_t)$ represents the same preferences as $w_t(c,x)$\pause\smallskip

\item$e_t(w_t(c,x),p_t)$  is a  \blue{\bf money metric representation }of the underlying preferences at constant prices

\end{itemize}

\end{itemize}
\end{frame}

%%%%%%%%%%%%%%%%%%%%%%%%%%%%%%%%%%%%%%%%%%%%%%%%%%%%%%%%%%
\begin{frame}
\frametitle{Fisher-Shell Quantity Index}

In order to compare $u_{t}(m_{t},p_t)$ with $u_{t+h}(m_{t+h},p_{t+h})$

let us define a hypothetical tomorrow's income
\begin{equation*}
\hat{m}_{t+h}=e_{t}[u_{t}(m_{t+h},p_{t+h}),p_t]
\end{equation*}\vspace{-.6cm}
\begin{itemize}\pause
	\item It uses \blue{today's preferences} $u_t$ to evaluate \smallskip
	
	the utility of tomorrow's income $m_{t+h}$ at tomorrow's prices $p_{t+h}$\pause\medskip
	\item Using today's preferences $e_t$\smallskip
	
	$\hat m_{t+h}$ measures the level of income needed tomorrow\smallskip
	
	to attain this utility level at today's prices $p_t$\pause\medskip
	
	\item $\hat m_{t+h}$ is a money metric utility representation

\end{itemize}
\end{frame}

%%%%%%%%%%%%%%%%%%%%%%%%%%%%%%%%%%%%%%%%%%%%%%%%%%%%%%%%%%
\begin{frame}
\frametitle{Fisher-Shell Quantity Index}

\begin{figure}
\begin{center}
\includegraphics[width=.7\textwidth]{omar-graph1.pdf}
\end{center}
\caption{Evaluation using time $t$ preferences}
\label{fig1}
\end{figure}
\end{frame}

%%%%%%%%%%%%%%%%%%%%%%%%%%%%%%%%%%%%%%%%%%%%%%%%%%%%%%%%%%
\begin{frame}
\frametitle{Fisher-Shell index}

\begin{itemize}
\item The \blue{\bf Fisher-Shell quantity index} is defined as
\begin{equation*}
\text{FS}_{t}=\frac{1}{m_{t}}\left. \frac{d\hat{m}_{t+h}}{dh}\right\vert _{h=0}
\end{equation*}\pause

\item \blue{\bf\small FS$_t$ is an equivalent variation measure}\pause\medskip

\item The FS quantity index is the growth rate of the factor $\hat m_{t+h}$ 

\end{itemize}

\end{frame}

%%%%%%%%%%%%%%%%%%%%%%%%%%%%%%%%%%%%%%%%%%%%%%%%%%%%%%%%%%
\begin{frame}
\frametitle{NIPA Growth Rate is a Welfare Measure}
\blue{\sc Main Proposition}\vspace{-.2cm}

\blue{\bf The Fisher-Shell quantity index is equal to a Divisia index}\hspace{.2cm}
\begin{equation*}
\boxed{\text{FS}_{t}=\underbrace{\big(1-s_t\big)\,\,\frac{\dot{c}%
_{t}}{c_{t}}+s_t\,\,\frac{\dot{x}_{t}}{x_{t}}}_{{\text {Divisia index}}}}
\hspace{1cm}s_t=\frac{p_t x_t}{m_t}
\end{equation*}\pause\vspace{-.3cm}

$\Rightarrow$ \blue{\bf NIPA delivers a welfare based measure of output growth}\pause

{\footnotesize
At the equilibrium of the two-sector AK economy 
$$
\text{FS}_t = 
\boxed{
g = (1-s) \alpha\gamma + s \gamma 
}
\in (\alpha\gamma,\gamma)
$$
where $\gamma$ and $s$ depend on preferences and technology parameters}

\end{frame}

%%%%%%%%%%%%%%%%%%%%%%%%%%%%%%%%%%%%%%%%%%%%%%%%%%%%%%%%%%
\begin{frame}
\frametitle{Interpretation}

\begin{itemize}
\item $v(k)$ measures the value of the representative agent's assets: \smallskip

{\small The discounted flow of consumption utility}\pause\medskip
\item $v(k)$ emerges from one among many possible utility representations  \smallskip

{\small The growth rate of $v(k)$ is meaningless}\pause\medskip

\item \blue{\bf Money metric utility}\smallskip

\begin{itemize}
\item {\small A FS-index is an equivalent variation measure adopting income as norm }
\pause\smallskip
\end{itemize}

\item The Bellman representation 
\item[] {\small The FS-index measures gains in the return to assets (output)}\pause\medskip

\item  Under additively separable preferences, $\rho$ is given 
\item[] {\small$\Rightarrow$\ the FS index measures welfare gains}

\end{itemize}
\end{frame}

%%%%%%%%%%%%%%%%%%%%%%%%%%%%%%%%%%%%%%%%%%%%%%%%%%%%%%%%%%
\begin{frame}
\frametitle{On Money Metric Utility}

\blue{\bf\small There exists a representation of preferences growing at rate $g$}\pause
\begin{itemize}
\item At equilibrium of the two-sector AK model 
$$
v(k_t) =
B\,  k_t^{\alpha (1-\sigma)}
$$
{\small which grows at the rate $ \gamma\, \alpha  (1-\sigma)$ different from the Divisia index $g$}\pause\bigskip

\item There exists an alternative representation
$$
\hat v(k_t) =
\max C \left(
 \int_t^\infty \frac{c_s^{1-\sigma}}{1-\sigma} \text{ e}^{-\rho (s-t)} \text{ d} t
\right)^{\frac{g}{\gamma\alpha(1-\sigma)}} \pause = C\ v(k_t)^{\frac{g}{\gamma\alpha(1-\sigma)}} 
$$
{\small that grows at the rate $g$, where $g$ is the Divisia index\pause\bigskip

\item $C$ is such that 
$$
\rho \hat v(k_0) = c_0 + p_0 x_0
$$}
{\small The return to assets at the reference time is equal to nominal income}%\pause\medskip

%\item \blue{\bf\small Output growth measures then welfare gains using an equivalent variation measure}

\end{itemize}
\end{frame}

%%%%%%%%%%%%%%%%%%%%%%%%%%%%%%%%%%%%%%%%%%%%%%%%%%%%%%%%%%
\begin{frame}
\frametitle{Investment also Matters}

\begin{itemize}

\item If current and future consumption is all that matters for welfare\smallskip

Should the consumption growth rate 
summarize all what is relevant?\pause\smallskip

The answer is \blue{\bf NO}\pause\medskip

\item  \blue{\bf Net investment also matters} {\small (Weitzman, 1976)}\smallskip

since it reflects welfare gains from postponed consumption\pause
\item[] (notice that $x$ is net investment) \pause\medskip

\item \blue{\bf NIPA provides a welfare-based measure of output growth}
\end{itemize}
\end{frame}

%%%%%%%%%%%%%%%%%%%%%%%%%%%%%%%%%%%%%%%%%%%%%%%%%%%%%
\begin{frame}
\frametitle{General Framework}
\framesubtitle{Recursive Preferences/General PPF}

\begin{itemize}
\item\blue{\bf\small The equality between the FS and Divisia index holds for a DGE model with general recursive preferences and technology}\pause\medskip
\item { Preferences} are represented by the \blue{\bf\small recursive welfare function} $U$%
\begin{equation*}
\frac{d}{dt}U(_{t}C)=-f(c_{t},U(_{t}C))
\end{equation*}

\begin{itemize}
\item $_{t}C=(c_{t+s})_{s=0}^\infty$ is a consumption path, 
%\item 
$f_{1}>0$ and %is marginal utility from current consumption
%\item  
$f_{2}<0$ %relates to the discount of future returns
\end{itemize}\pause\medskip
\item \blue{\bf\small Quasi-concave technology}: %$c_{t}$ and $i_t$, $i_t=\dot{k}_{t}$, such that 
$\{c_{t},x_{t}\}\in \Gamma (\Theta_t,k_{t})$\pause\medskip
\item $\big(\{c_{s},x_{s}\}\big)_{s\geq t}$  maximizes $U(_{t}C)$ subject to the feasibility
constraint\medskip\pause
\item Bellman equation representation
\begin{equation*}
0=\max_{(c,x)\in \Gamma (\Theta_t,k_{t})} \underbrace{f\big(c,v(k_{t},\Theta_t)\big) + v_1(k_{t},\Theta_t)x}_{w_t(c,x)} + v_2(k_{t},\Theta_t)\dot\Theta_t
\end{equation*}\pause
\item The two-sector AK model is a particular case% of recursive preferences
\end{itemize}
\end{frame}

%%%%%%%%%%%%%%%%%%%%%%%%%%%%%%%%%%%%%%%%%%%%%%%%%%%%%
\begin{frame}
\frametitle{General Framework}
\framesubtitle{On household heterogeneity}

\begin{itemize}
\item The result also holds for heterogeneous households (in preferences, assets, income) \pause\medskip
\item[] \blue{\bf\small The Fisher-Shell index is, indeed, equal to the Divisia index}\pause
\item[]{\small (even if  different from the representative household growth rate)}\pause\medskip

\item \blue{\bf\small Key assumptions}
\begin{itemize}
\item Money metric utility
%: income is the metric used to measure households' utility
\pause\smallskip
\item The Bellman representation of preferences is quasi-linear on investment
\item[] (Gorman aggregation conditions hold)\pause\medskip
\end{itemize}
\item The Fisher-Shell index implicitly assumes that \pause

\begin{itemize}

\item The \blue{\bf\small social welfare function is utilitarian}
\item[] with households weighted \blue{\bf\small proportionally to their income}
\end{itemize}

\end{itemize}

\end{frame}


%%%%%%%%%%%%%%%%%%%%%%%%%%%%%%%%%%%%%%%%%%%%%%%%%%%%%%%%%%
\begin{frame}
\frametitle{Paradox: Endowment Economy }

A true quantity index measures output
growth \blue{\bf conditional on both preferences and technology}\vspace{-.3cm}\pause
\begin{itemize}
\item Assume a \blue{\bf two-sector AK economy} such that at equilibrium
\begin{itemize}
\item The growth rate of investment is 6\%; the
investment share is 20\%
\item The growth rate of consumption is 2\%
\item The Divisia index delivers a 2.8\% growth rate
\end{itemize}\pause\medskip
\item Consider an \blue{\bf endowment economy} with exactly the same
preferences 
\begin{itemize}
\item The same equilibrium consumption path (mana from haven)
\item The Divisia index measures output growing at 2\%
\end{itemize}\pause\medskip
\item Why two economies with identical preferences and the
same consumption path do not grow at the same rate?\pause
\begin{itemize}
\item Current income (the norm) is defined differently\pause
\begin{itemize}
\item In the AK model, investments are needed for permanently grow\pause
\item In the endowment economy, God provides
\end{itemize}
\end{itemize}
\end{itemize}

\end{frame}

%%%%%%%%%%%%%%%%%%%%%%%%%%%%%%%%%%%%%%%%%%%%%%%%%%%%%%%%%%
\begin{frame}
\frametitle{A Word of Caution}


\begin{itemize}
\item It is well-known in endogenous growth theory that 
\item[] \blue{\small\bf there is an optimal growth rate}\pause\smallskip
\item Suppose the two-sector AK economy is at (the optimal) equilibrium
\item[] but, a government introduces an incentive to increase growth\pause\smallskip
\item At time zero, the economy will suffer a reduction in welfare\pause\smallskip
\item From them, welfare will be growing at a larger rate (Divisia index) 
\item[] {\small (at some point in time consumption will be larger in the distorted economy)}\pause\smallskip
\item Unfortunately, NIPA does not measure capital loses
\end{itemize}

\end{frame}

%%%%%%%%%%%%%%%%%%%%%%%%%%%%%%%%%%%%%%%%%%%%%%%%%%%%%%%%%%
\begin{frame}
\frametitle{Simplifying Assumptions and Extensions} \pause


	\begin{itemize}
	\item There is only one nondurable consumption good\smallskip\pause
	
	{\small Easy to extend to a basket of nondurable consumption goods and services}\medskip\pause
	
	\item There is only one investment good\smallskip\pause
	
	{\small Easy to extend to many investment and durable consumption goods}
	
	{\small Less easy to different forms of human and intangible capital investment}\medskip\pause
	
	\item Prices reflect marginal productivities\smallskip\pause
	
	{\small Easy to extend to frameworks where prices are distorted}%\smallskip
	
	%{\small In practice, prices need to be measured or imputed}
	\medskip\pause
	
	
	\item Dealing with externalities is still an open issue%\medskip\pause
	
	
	
	%\item Dealing with the non-market economy is also an open issue
	
	\end{itemize}
\end{frame}



%%%%%%%%%%%%%%%%%%%%%%%%%%%%%%%%%%%%%%%%%%%%%%%%%%%%%%%
\part{Open Economy} 
\frame{\partpage} 
\section{Open Economy} 
%%%%%%%%%%%%%%%%%%%%%%%%%%%%%%%%%%%%%%%%%%%%%%%%%%%%%%%


%
%
\begin{frame}
\frametitle{Open Economy}
\framesubtitle{Bellman representation of preferences}
%
%
{\small 
\begin{itemize}
\item In an open economy, the PPF is expanded thanks to trade
\item Goods can be traded at  international prices  $(1,p^*_{t})$\\
\blue{\footnotesize decoupling production from final demand}

\begin{itemize}
\item  $y_{c,t}$ and $y_{x,t}$ are produced quantities\smallskip
\item    $c_{t}$ and $x_{t}$ are  consumption and net investment
\end{itemize}\smallskip

\item \blue{The representative household holds $a_{t}$ units of capital abroad}

\item The production possibility frontier is give by 
\[
(y_{c,t},y_{x,t})  \in  \Gamma(k_{t})%,  A_{t}), 
\quad\quad
\dot k_{t} =  x_{d,t} \quad \text{and} \quad  \blue{\dot {a}_{t} = x_{f,t}}
\]
\[
c_{t} +  \underbrace{\blue{\ \ \eta_t p^*_{t} x_{dt} }}_{p_t}
 + \underbrace{\blue{\ \ p^*_{t}  x_{ft} \ \ }}_{\text{\scriptsize net exports}} =  
y_{ct}  + p^*_{xt} y_{xt} + \underbrace{\blue{ r_t  p^*_{t}a_{t} }}_{=\ 0}
\nonumber
\]
\item[]{\footnotesize $\eta_{t}$ represents domestic price distortions}
\item {\footnotesize When $r_t=0$, $x_f$ is the trade balance}

\end{itemize}
}
\end{frame}


%
%
\begin{frame}
\frametitle{Open Economy}
\framesubtitle{Bellman representation of preferences}
%
%
{\small 
\begin{itemize}
\item The Bellman equation representation of the problem is
$$
\rho v(k_t,a_{t}) %,X_{t})   
=   \max_{c, x_d,x_f}\  \underbrace{u(c) + \blue{v_{1} (.) x_d +  v_{2} (.) x_f }}_{w_{t}(c,x_d,x_f)}
%+ v_3(.)\dot X_{t}
$$
$$
\text{s.t.} \quad  c + \blue{p_{t}\, x_d + p^{*}_{t}\, x_f} = m_{t} ,
$$

\item $w_{t}(c,x_d,x_f) $ is the \blue{Bellman representation of preferences} over current consumption and current net investment (domestic + foreign)

\item Associated to the indirect utility and expenditure functions
\[
u_{t}(m_{t}, \underbrace{p_{x,t},p^*_{x,t}}_{\text{\bf p}_{t}})  \equiv \max_{p_{c,t} c + p_{x,t} x = m_{t}} w_{t}(c,x_d,x_f)
\]\vspace{-.3cm}
\[\ \ \ 
e_{t}(u_{t},\text{\bf p}_{t}) 
\equiv \min_{w_{t}(c,x_d,x_f)= u_{t}}  c + p_{x,t}\, x_d + \blue{p^{*}_{x,t}\, x_f}
\]

\item In an open economy, \blue{the additional term is the trade balance $x_f$ priced at $p^{*}_x$}

\end{itemize}
}
\end{frame}

%
%
\begin{frame}
\frametitle{Open Economy}
\framesubtitle{Economic Growth and Welfare Gains}
%
%
{\small 
\begin{itemize}

\item Let us compare income at $t$ to income at $t+h$

\item To compare income $m_t$ at prices {\bf p}$_t$ to income $m_{t+h}$ at prices {\bf p}$'_{t+h}$, we use time $t$ preferences to create an artificial income
$$
\widehat m_{t+h} = e_t\Big(u_t(m_{t+h},\text{\bf p}_{t+h}),\text{\bf p}_t\Big)
$$


\item[]  {\bf Definition}: $\widehat m_{t+h}$ is the level of income tomorrow 
necessary to attain utility $u_t(m_{t+h}, \text{\bf p}_{t+h})$ {\footnotesize (future income at  future prices)} at country current prices $\text{\bf p}_t$ 

\item[] {\footnotesize (we are correcting by changing in prices: purchasing power equivalence)} \medskip

\item The Fisher-Shell index measuring the welfare gains btw {\footnotesize $t$ and $t+h$} is
\[
\gamma^{\text{\text{FS}}}_{t} \equiv 
%=
\frac{1}{m_t} \left. \frac{\text{d} \widehat m_{t+h}}{\text{d} h} \right|_{h= 0}
\]
\item[] {\footnotesize As if  a household evaluates the welfare gains of moving from $t$ to {\scriptsize $t+h$}}


\end{itemize}
}
\end{frame}

%
%
\begin{frame}
\frametitle{Open Economy}
\framesubtitle{Economic Growth and Welfare Gains}
%
%
{\small 
\begin{itemize}

\item {\bf Proposition:} The Fisher-Shell index is equal to a Divisia index
\[ \hspace{-1.5cm}
\boxed{
\gamma^{\text{\text{FS}}}_{t} = 
s_{ct} \  \frac{\dot c_t}{c_t } + s_{x_d t} \ \frac{\dot x_{dt}}{x_{dt}} + s_{x_f t} \ \frac{\dot x_{ft}}{x_{ft}}
}
\ \ \ \  \text{\scriptsize s.t.\ \ $s_c + s_{x_d} + s_{x_f} =1$}
\]

\item Important 
\begin{itemize}

\item \blue{The trade balance $x_f$ is an investment in foreign capital}
\item[] {\footnotesize that must be deflated using an investment price index}\smallskip

\end{itemize}\medskip

\item However, NA deflate exports and imports by their own deflators
\item[] \blue{NA measure growth in domestic production, not in welfare}\medskip

\item \blue{\footnotesize\bf Conveniently adjusted, growth rates in NA are welfare based}\medskip

\end{itemize}
}
\end{frame}


%%%%%%%%%%%%%%%%%%%%%%%%%%%%%%%%%%%%%%%%%%%%%%%%%%%%%%%%%%
\begin{frame}[label=relativeprice]
\frametitle{Relative Price of Equipment}
\vspace{-.4cm}
\begin{figure}
\begin{center}
\includegraphics[width=.9\textwidth]{prices.pdf}
\end{center}
\vspace{-1.5cm}
\label{fig1}
\end{figure}
{\small Fernald (US-NIPA):  Equipment plus consumer durables price, relative to price of other business output}
\hyperlink{embodied}{\beamergotobutton{back}}

\end{frame}

%%%%%%%%%%%%%%%%%%%%%%%%%%%%%%%%%%%%%%%%%%%%%%%%%%%%%%%%%%
\begin{frame}[label=shares]
\frametitle{Investment to GDP Share and Ratio}
\vspace{-.4cm}
\begin{figure}
\begin{center}
\includegraphics[width=.9\textwidth]{shares}
\end{center}
\vspace{-1.5cm}
\label{fig1}
\end{figure}
{\small Fernald (US-NIPA):  Equipment and consumer durables as a share of business output}
\hyperlink{embodied2}{\beamergotobutton{back}}

\end{frame}

%%%%%%%%%%%%%%%%%%%%%%%%%%%%%%%%%%%%%%%%%%%%%%%%%%%%%%%%%%
\end{document}

%%%%%%%%%%%%%%%%%%%%%%%%%%%%%%%%%%%%%%%%%%%%%%%%%%%%%%%%%%
\begin{frame}
\frametitle{Bellman Representation}
The planner's problem becomes 
\begin{equation*}
\boxed{\max_{c,x} \underbrace{\,\,\,\frac{c^{1-\sigma }}{1-\sigma }+v'(k_t) \,\,x\,\,\,}_{w_{t}(c,x)} }
\end{equation*}

subject to the production possibility frontier (PPF)
$$
x=% \underbrace{
(A-\delta) k_{t}%}_{\text{net income }m_t}
- A {c}^{\frac{1}{\alpha} }
$$

\end{frame}

%%%%%%%%%%%%%%%%%%%%%%%%%%%%%%%%%%%%%%%%%%%%%%%%%%%%%%%%%%
\begin{frame}
\frametitle{Competitive Equilibrium}

\begin{figure}
\begin{center}
\includegraphics[width=.7\textwidth]{omar-graph2.pdf}
\end{center}
\caption{Preferences, production possibility frontier and budget constraint}
\label{fig1}
\end{figure}
\end{frame}


%%%%%%%%%%%%%%%%%%%%%%%%%%%%%%%%%%%%%%%%%%%%%%%%%%%%%%%%%%
\end{document}

%%%%%%%%%%%%%%%%%%%%%%%%%%%%%%%%%%%%%%%%%%%%%%%%%%%%%%%%%%
%%%%%%%%%%%%%%%%%%%%%%%%%%%%%%%%%%%%%%%%%%%%%%%%%%%%%%%%%%
\begin{frame}[label=substitutionbias]
\frametitle{Mayas and Aztecs}

Assumptions:
\begin{itemize}\vspace{-.4cm}
	\item The population size of Mayas and Aztecs is identical \pause
	\item Aztecs produce $2 C$ units of cacao (the numeraire)\pause
	\item Mayas produce $2 M_t$ units of maize, $M_t=C\, (1+\gamma)^{t}$, $\gamma>0$\pause
	\item Both have the same log preferences (weights 1/2)\pause
	\item Mayas and aztecs trade with each other and trade is costless\pause
\end{itemize}\vspace{-.2cm}
Equilibrium:
\begin{itemize}\vspace{-.4cm}
	\item At equilibrium both consume $C$ and $M_t$ every period (symmetry) \pause
	\item The equilibrium price of corn, $P_t = (1+\gamma)^{-t}$, decreasing at rate $\gamma$\pause
	\item GDP (in current cacao units) is equal to $C +P_tM_t$ for both\pause
	\item The share of cacao and corn in total income (for Aztecs and Mayas)
	{\small $$
	\frac{C}{C + P_{t}M_{t}}=
	\underbrace{\frac{P_{t}M_{t}}{C + P_{t}M_{t}}}_{\omega}=1/2
	$$}
	%{\small  The raise in $M_t$ is perfectly compensate for the decline in $P_t$}
\end{itemize}
\end{frame}

%%%%%%%%%%%%%%%%%%%%%%%%%%%%%%%%%%%%%%%%%%%%%%%%%%%%%%%%%%
\begin{frame}[label=substitutionbias]
\frametitle{NIPA: Mayas and Aztecs}

Assumptions:
\begin{itemize}\vspace{-.4cm}

	\item GDP for both will be $C+P_t M_t$\pause\medskip
	\item \blue{\bf Divisia index}: Output growth is $\gamma/2$ for both Mayas and Aztecs 
	\pause\medskip
	\item It does not measure growth in physical units:\smallskip
	\begin{itemize}
	\item Production in physical units grew at $\gamma$ for the Mayas (maize)\smallskip
	\item[] and zero for the Aztecs (cacao) \pause\medskip
	\end{itemize}
	\item Neither TFP growth is measured in physical units (maize or cacao) \pause\medskip
	\item We argue that GDP growth measures gains in consumption utility 
	\item[] {\small (the same for both Mayas and Aztecs)}
\end{itemize}
\vspace{-.5cm}

\end{frame}

%%%%%%%%%%%%%%%%%%%%%%%%%%%%%%%%%%%%%%%%%%%%%%%%%%%%%%%%%%
\begin{frame}
\frametitle{Substitution Bias: Aztecs NIPA} 

\begin{itemize}
\item Let Aztecs use a Laspeyres index to measure output growth\medskip\pause

\item The growth rate from $t-1$ to $t$, with base year $t_0$, is\pause
\begin{eqnarray*}
{\cal L}_{t_0,t} &=& \frac{C + P_{t_0}M_{t}}{C + P_{t_0}M_{t-1}} -1 \\ \pause
&=&%\frac{C}{C + P_{t_0}M_{t-1}} \underbrace{\frac{\Delta C}{C} }_{zero}+
\underbrace{\frac{P_{t_0}M_{t-1}}{C + P_{t_0}M_{t-1}}}_{\omega_{t_0,t-1}}\underbrace{\frac{\Delta M_t}{M_{t-1}}}_{\gamma}
\end{eqnarray*} %\vspace{-.6cm}
{\small
$\omega_{t_0,t-1}$ is the weight of maize in the Laspayeres index
}\pause\medskip
\item \blue{\bf Substitution bias}: \smallskip\pause
	\begin{itemize}
	\item The Laspeyres index coincides with the Divisia index at $t_0+1$ only \smallskip\pause
	\item The weight $\omega_{t_0,t-1}$ moves from 1/2 to one as $t$ goes to infinity	
	\end{itemize}
\end{itemize}
\end{frame}

%%%%%%%%%%%%%%%%%%%%%%%%%%%%%%%%%%%%%%%%%%%%%%%%%%%%%%%%%%
\begin{frame}
\frametitle{Substitution Bias: Implications} 

\begin{itemize}

\item \blue{\bf Substitution bias}: \smallskip\pause
	\begin{itemize}
	\item The Laspeyres index overstates the growth of income by giving a larger and larger weight to  maize  
{\small (benefiting from technical progress)}\smallskip\pause
	\item Since the price of maize is declining, aztecs substitute it for cacao\smallskip\pause
	\item Substitution supports the difference in technical progress between maize and cacao\smallskip\pause
	\item Using past prices to measure current income, overweights the item growing faster (maize)\smallskip\pause
	%\item  The substitution bias is increasing over $t$ \smallskip
	
	%{\small $\omega_{t_0,t-1}\rightarrow 1 $  as $t\rightarrow\infty$}   \smallskip\pause
	\item Of course, updating the base year moves the weights back to 1/2
	
	But, the revision changes past growth performance\smallskip\pause
	\end{itemize}	
	%Moving the base year to $t$ implies $\omega_{t,t+1}=1/2$ 
	
\item The solution are the \blue{\bf chained indexes }\pause

\item[$\Rightarrow$] The base year changes every period to keep the weights equal to 1/2
\hyperlink{substitution}{\beamergotobutton{back}}
\end{itemize}
\end{frame}

%%%%%%%%%%%%%%%%%%%%%%%%%%%%%%%%%%%%%%%%%%%%%%%%%%%%%%%%%%
\begin{frame}[label=mayasshare]
\frametitle{Aztecs `corn to income ratio'} 

\begin{itemize}

\item Aztecs `corn to income share' is 
$$
\frac{P_{t}M_{t}}{C + P_{t}M_{t}}=1/2
$$\medskip

\item Real income grows at ${\small{1}/{2}} \,\gamma$, but real corn at $\gamma$\medskip

The ratio of real corn consumption to real income grows at ${\small{1}/{2}} \,\gamma$
\hyperlink{thispaper}{\beamergotobutton{back}}
\end{itemize}
\end{frame}

%%%%%%%%%%%%%%%%%%%%%%%%%%%%%%%%%%%%%%%%%

\begin{frame}
\frametitle{Fundamental Question} \pause

\begin{itemize}
\item Is the output growth rate in NIPA a welfare measure? \pause

Our answer is YES\bigskip\pause
\item What does it mean that output growth is a welfare measure?\pause\medskip
	\begin{itemize}
	\item Human societies are endowed with assets: human, physical, intangible\pause\smallskip
	\item Assets are used to produce goods and services\pause\smallskip
	\item NIPA measures the welfare gains emerging from this production\pause\bigskip
	\item Welfare is the discounted flow of consumption utility produced by these \smallskip
	
	assets, \pause which value is equal to the associated human welfare\pause\smallskip
	 \item Output, indeed, measures the returns to assets\pause\smallskip
	 \item The growth rate of output measures changes in these returns\pause\smallskip
	 \item If the discount rate is constant, it also measures changes in welfare
\end{itemize}
\end{itemize}
\end{frame}

%%%%%%%%%%%%%%%%%%%%%%%%%%%%%%%%%%%%%%%%%%%%%%%%%%%%%%%%%%
\begin{frame}
\frametitle{Does Investment also Matter?} \pause

\begin{itemize}
\item Let resume production in consumption and investment\pause\medskip
\item Consumption produces welfare directly\smallskip

Investment is just a mean to cumulate assets \pause\medskip
\item Should we conclude that consumption growth is a sufficient statistic of welfare gains?\pause\smallskip

or does investment also matter?\pause\medskip

\item (Net) investment will produce future consumption and welfare\pause\smallskip

Then, the growth rate of investment also matters for welfare\pause\medskip
\item By the way, it is Net National Product (NNP) not GDP that matters for welfare:
Weitzman (QJE, 1976)
\end{itemize}
\end{frame}


%%%%%%%%%%%%%%%%%%%%%%%%%%%%%%%%%%%%%%%%%%%%%%%%%%%%%%%%%%
\begin{frame}
\frametitle{Recursive Preferences}
\framesubtitle{Additive separable utility function}

\begin{itemize}
\item Additive separable utility is a particular case
\begin{equation*}
U(_{t}C)=\int_{t}^{\infty }e^{-\rho (s-t)}u(c_{s})ds
\end{equation*}%
with $u^{\prime }(c)>0$ and $\rho >0$\pause\medskip
\item Differentiating with respect to time $t$
\begin{equation*}
\frac{d}{dt}U(_{t}C)=-u(c_{t})+\rho U(_{t}C).
\end{equation*}\pause\medskip
\item Hence, 
{\small \begin{eqnarray*}
f(c,U)&=&u(c)-\rho U\\
f_{1}(c,U)&=&u^{\prime}(c)>0\\
f_{2}(c,U)&=&-\rho <0
\end{eqnarray*}}

\end{itemize}
\end{frame}

%%%%%%%%%%%%%%%%%%%%%%%%%%%%%%%%%%%%%%%%%%%%%%%%%%%%%%%%%%
\begin{frame}
\frametitle{Recursive Preferences}
\framesubtitle{Additive separable utility function}

\begin{itemize}
\item Bellman equation representation
\begin{equation*}
0=\max_{(c,i)\in \Gamma (t,k_{t})} \underbrace{-f(c,v(k_{t})) + v'(k_{t})i}_{w_t(c,i)}
\end{equation*}\pause\medskip

\item $w_t(c,i)$ represents preference over consumption and investment\pause\medskip

\item Past actions summarize in $k_{t}$, making $w(c,i)$ time dependent
\end{itemize}
\end{frame}


%%%%%%%%%%%%%%%%%%%%%%%%%%%%%%%%%%%%%%%%%%%%%%%%%%%%%%%%%%
\begin{frame}
\frametitle{Investment does matter: An example}

\begin{itemize}

\item Consider two identical economies apart from $\rho >\tilde{\rho}$ and $\delta <\tilde{\delta}$, such that
$\delta +\rho =\tilde{\delta}+\tilde{\rho}$\pause
\item Consumption grows at the same rate in both economies since $g=\tilde{g}$\pause
\item However, the patient economy weights more future consumption

 so it has to value more the same growth rate of consumption\pause

\item The saving rate is larger in the more patient economy $\tilde s>s$\pause

\item Then, the Divisia index even if the growth rates of consumption and investment are the same in both economies\pause
\item The Divisia index is a better representation of preference than the consumption
growth rate

\end{itemize}
\end{frame}


%%%%%%%%%%%%%%%%%%%%%%%%%%%%%%%%%%%%%%%%%%%%%%%%%%%%%%%%%%
\begin{frame}
\frametitle{Bellman representation}


Let us assume the feasibility condition takes the following form
\begin{equation*}
\dot k_t\equiv i_t = f(k_t) - h_t(c_t),
\end{equation*}%

\begin{itemize}
\item Technology $f(.)$ is Neoclassical and net of depreciation
\item The transformation locus $h(c_t)$ is $C^{2}$ and concave
\end{itemize}


The Bellman representation of 
the planner's problem in the space $(c,i)$ is%
\begin{equation}
\rho v_t(k_{t})=\max_{i=f(k_t) - h_t(c)}{u(c_t)}+v_t^{\prime }(k_{t})i. \label{bellman}
\end{equation}%
\end{frame}

%%%%%%%%%%%%%%%%%%%%%%%%%%%%%%%%%%%%%%%%%%%%%%%%%%%%%%%%%%
\begin{frame}
\frametitle{Chain indexes} \pause

\begin{itemize}

\item NIPA traditionally featured a Laspeyres fixed-base quantity index to measure real output growth\pause\smallskip
\item Fixed-base quantity index yields a good measurement provided that relative prices remain stable\pause\smallskip
\item The observed fast decline of equipment prices made the weight of investment in the Laspeyres index became obsolete quickly
enough to have a relevant impact on growth measurement\hspace{1cm}\hyperlink{substitutionbias}{\beamergotobutton{substitution bias}}
\pause\smallskip
\item As a reaction, since the
early 1990's, NIPA moved to a chained-type index built on the Fisher Ideal index\pause\smallskip
\end{itemize}
\end{frame}

%%%%%%%%%%%%%%%%%%%%%%%%%%%%%%%%%%%%%%%%%%%%%%%%%%%%%%%%%%
\begin{frame}
\frametitle{Consumption Equivalence}

\begin{itemize}

\item \blue{\bf\small Consumption equivalence} is a compensating variation measure
\item[] {\small (in terms of the entire consumption path instead of current income)}\pause\medskip

\item $\lambda_h$ is a hypothetical increase in the consumption path that makes a household indifferent between staying at $t$ or jumping to $t+h$ 
$$
\lambda_h^{1-\sigma} v(k_t) = v(k_{t+h})
\ \ \ \ \ \text{or}\ \ \ \ \ 
\boxed{\lambda_h= \left(\frac{v(k_{t+h})}{ v(k_t) }\right)^{\frac{1}{1-\sigma}}
=
\left(\frac{ k_{t+h}^{\alpha (1-\sigma)}}{k_t^{\alpha (1-\sigma)} }\right)^{\frac{1}{1-\sigma}}
}
$$\pause
\item The growth rate associated to the compensating variation measure is
$$
g^{ce} =\left. \frac{\text{d}\lambda_h}{\text{d} h} \right|_{h=0}= \alpha\gamma. 
$$\pause
\item\blue{\bf\small The growth rate of consumption is a (consumption equivalent) welfare based measure too}
\end{itemize}
\end{frame}



