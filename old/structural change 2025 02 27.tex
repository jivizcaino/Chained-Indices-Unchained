\documentclass[12pt,a4paper]{article}

\usepackage{amssymb}
\usepackage{amsthm}
\usepackage{amsfonts}
\usepackage{graphicx}
\usepackage{amsmath}
\usepackage{caption}
\usepackage{xcolor}
\usepackage{booktabs} % For a better table layout
\usepackage{rotating} % For rotating the table
\usepackage{comment}
\usepackage{subcaption} % For subfigures#

\usepackage{float}
\usepackage[toc,page]{appendix}
\usepackage{natbib} 
\bibliographystyle{apalike}
\setcitestyle{authoryear,open={(},close={)}}
%\addbibresource{bibliography.bib} %Import the bibliography file


\setlength{\parskip}{10pt}

\textheight 23 true cm
\textwidth 15.8 true cm
\oddsidemargin 0 true cm
\evensidemargin 0 true cm
\topmargin -0.8 true cm

\thispagestyle{empty}

\renewcommand{\arraystretch}{1.2}

\renewcommand\baselinestretch{1.35}
\baselineskip=1.35\normalbaselineskip
\footnotesep=1.10\normalbaselineskip

\newtheorem{theorem}{Theorem}
\newtheorem{acknowledgement}{Acknowledgement}
\newtheorem{algorithm}{Algorithm}
\newtheorem{assumption}{Assumption}
\newtheorem{case}{Case}
\newtheorem{claim}{Claim}
\newtheorem{conclusion}{Conclusion}
\newtheorem{condition}{Condition}
\newtheorem{conjecture}{Conjecture}
\newtheorem{corollary}{Corollary}
\newtheorem{criterion}{Criterion}
\newtheorem{definition}{Definition}
\newtheorem{example}{Example}
\newtheorem{exercise}{Exercise}
\newtheorem{lemma}{Lemma}
\newtheorem{notation}{Notation}
\newtheorem{problem}{Problem}
\newtheorem{proposition}{Proposition}
\newtheorem{remark}{Remark}
\newtheorem{solution}{Solution}
\newtheorem{summary}{Summary}

\def\x{x}  % Change here the symbol for investment if you must

\def\equiv{\doteq}  % Change here the symbol for 'equal by definition'


\begin{document}

\title{Chained Indices Unchained: \\ {\Large On the Welfare Foundations of Income Growth Measurement II}}

\author{
Omar Licandro%
\footnote{The author expresses gratitude to Ariel Burstein for a highly beneficial discussion held during a visit to UCLA in April 2024.} \\ {\small U. of Leicester and BSE} 
\and
Juan Ignacio Vizcaino \\ {\small University of Nottingham}}

\date{February 2025}



\maketitle
{\small
\begin{abstract}
\vspace*{-.5em}
{\footnotesize
\noindent 


\medskip\noindent {\sc Keywords}: Chained quantity indexes, GDP, equivalent variation, Divisia index, Fisher-ideal index and Fisher-Shell index.

\medskip\noindent {\sc JEL classification numbers}: E01, O47, C43, O41.
}
\end{abstract}
}
\thispagestyle{empty}

\newpage

%%%%%%%%%%%%%%%%%%%%%%%%%%%%%%%%%%%%%%%%%


%%%%%%%%%%%%%
\section{Introduction}
%%%%%%%%%%%%%



%%%%%%%%%%%%%
\section{Structural Change Model} \label{sec:LBD}
%%%%%%%%%%%%%

%%%%%%%%%%%%%
\paragraph{Description of technology.}
%%%%%%%%%%%%%

Let's follow Herrendorf et al.~(2021) by assuming that there are two sectors, goods and services.
Goods and services value added, denoted by \( Y_{gt} \) and \( Y_{st} \), respectively, are produced according to the Cobb–Douglas production functions
\[
Y_{jt} = A_{jt} K_{jt}^{\theta} L_{jt}^{1-\theta}, \quad j \in \{g,s\} .
\]
These functions share the same capital intensity, represented by \( \theta\in(0,1) \), but may have different total factor productivities, \( A_{jt} \). Production in each sector relies on capital \( K_{jt} \), and labor \( L_{jt} \), where \( j \in \{g,s\} \) indexes the goods and services sectors, respectively.

Investment is produced using the CES technology
\[
I_t = A_{xt} \left( \omega^{\frac{1}{\varepsilon}} X_{gt}^{\frac{\varepsilon - 1}{\varepsilon}} + (1 - \omega)^{\frac{1}{\varepsilon}} X_{st}^{\frac{\varepsilon - 1}{\varepsilon}} \right)^{\frac{\varepsilon}{\varepsilon - 1}} ,
\]
where $X_{gt}$ and $X_{st}$ represent goods and services inputs, respectively, \( \varepsilon \) is the elasticity of substitution between them, and \( \omega \in(0,1) \) determines the weight of sectorial inputs. 
The state of investment-specific technology is neutral with respect to inputs and represented by the investment specific productivity, \( A_{xt} \).

A representative household has an initial capital stock \( K_0 > 0 \) and is endowed with one unit of time per period, which is supplied inelastically. Capital depreciates at a rate \( \delta >0 \), following the law of motion:
\begin{equation}\label{eq:K}
\dot{K}_t = X_t - \delta K_t.
\end{equation}
Both capital and labor can move freely between sectors. At equilibrium, the feasibility conditions impose the following constraints:
\[
K_{gt} + K_{st} = K_t, \quad L_{gt} + L_{st} = L_t,
\]
\[
C_{gt} + X_{gt} = Y_{gt}, \quad C_{st} + X_{st} = Y_{st},
\]
where \( C_{gt} \) and \( C_{st} \) are the consumption of goods and services, respectively. The labor force $L_t$ grows at the exogenous rate $n>0$.


%%%%%%%%%%%%%
\paragraph{Equilibrium prices.}
%%%%%%%%%%%%%

Let us adopt the investment good as numeraire. Following Herrendorf et al (2021), it is easy to show that the price of goods, $P_{gt}$, relative to the price of services, $P_{st}$, 
\[
\frac{P_{gt}}{P_{st}} = \frac{A_{st}}{A_{gt}}  ,
\]
is equal to the inverse of the relative sectorial TFPs. 
Notice that a dupla of production factors $(K,L)$ that produces $K^\theta L^{1-\theta} = 1$ has the same value in the goods and service sectors, since $P_{gt} A_{gt}= P_{st} A_{st}$.

Moreover, since we have adopted the investment good as numeraire, the prices of goods and services, relative to the investment good, read
\begin{equation}\label{eq:Pj}
\boxed{
P_{jt} = \frac{{\cal A}_{t}}{ A_{jt}}, \quad j \in \{g,s\}.
}
\end{equation}
where 
\begin{equation}\label{eq:calA}
\boxed{
{\cal A}_{t} = A_{xt} \left( \omega A_{gt}^{\varepsilon - 1} + (1 - \omega) A_{st}^{\varepsilon - 1} \right)^{\frac{1}{\varepsilon - 1}} .
}
\end{equation}
As shown below, ${\cal A}_{t} $ is the productivity of the investment sector.

In the investment sector, at equilibrium, the ratio of expenditure shares and input quantities on goods and services are, respectively, given by
\[
\frac{P_{gt} X_{gt}}{P_{st} X_{st}} = \frac{\omega}{1 - \omega} \left( \frac{A_{st}}{A_{gt}} \right)^{1 - \varepsilon} 
\ \ \ \ \text{and}\ \ \ \ 
\frac{X_{gt}}{X_{st}} =  \frac{\omega}{1 - \omega} \left( \frac{A_{gt}}{A_{st}} \right)^{ \varepsilon} 
.
\]
Notice that if goods and services are complementary in the investment technology, i.e. $\varepsilon\in(0,1)$, and technical progress is faster in the production of goods, relative to services, then the value added of the goods sector shrinks while real production increases, relative to the service sector.

%%%%%%%%%%%%%
\paragraph{Aggregate investment technology. [We can drop this paragarph.]}
%%%%%%%%%%%%%

At equilibrium, from Lemma 1 in Herrendorf et al. (2021),  the aggregate investment technology is
\begin{equation}\label{eq:investment}
I_t =  {\cal A}_{t} \left( \frac{X_{gt}}{A_{gt}} + \frac{X_{st}}{A_{st}} \right)
= {\cal A}_{t} K_{xt}^{\theta} L_{xt}^{1-\theta} %= {\cal A}_{xt} L_{xt} K_{t}^{\theta}
\end{equation}
where $L_{xt}$ and $K_{xt}$ are defined as
\[
L_{xt} = \frac{X_{gt}}{A_{gt} K_t^{\theta}} + \frac{X_{st}}{A_{st} K_t^{\theta}},
\ \ \ \ \text{and}\ \ \ \ 
K_{xt} = L_{xt} K_t .
\]
Let's think on $L_x\in(0,1)$ as the fraction of employment and capital allocated to produce inputs for the investment sector.

%%%%%%%%%%%%%
\paragraph{Aggregate production technology.}
%%%%%%%%%%%%%

Let us define aggregate final output, measured in units of final investment, as:  
\begin{equation}\label{eq:identity}
\boxed{
Y_t = P_{gt} C_{gt} + P_{st} C_{st} + I_t  .
}
\end{equation}
It can be easily shown that the aggregate production technology is
\begin{equation}\label{eq:output}
\boxed{
Y_t = {\cal A}_{t} K_t^{\theta} L_t^{1-\theta}.
}
\end{equation}
%The fraction $L_{xt}$ is allocated to investment and the complement to consumption.
%Equilibrium factor prices are given by:  
%\[
%R_t = \theta {\cal A}_{t} K_t^{\theta - 1},
%\]
%\[
%W_t = (1 - \theta) {\cal A}_{t} K_t^{\theta}.
%\]

%%%%%%%%%%%%%
\paragraph{Aggregate dynamics.}
%%%%%%%%%%%%%

From the equations (\ref{eq:identity}) and (\ref{eq:output}), for $t\geq 0$,  given the exogenous path of ${\cal A}_{xt}$ and an initial stock of capital $K_0 > 0$, the  $\{K_t\}$ law of motion reads
\[
\dot{K}_t = \underbrace{{\cal A}_{t} K_t^{\theta} - E_t}_{I_t} - \delta K_t.
\]
where 
\begin{equation}\label{eq:E}
\boxed{
E_t = P_{gt} C_{gt} + P_{st} C_{st} 
}
\end{equation}
is consumption expenditure in units of the investment good. 
%Notice also that, from the dethe investment rate $L_{xt}$ is determined by the saving behaviour of the representative household. HOW?


%%%%%%%%%%%%%
\paragraph{Non-homothetic preferences.}
%%%%%%%%%%%%%

The economy features an infinitely lived representative household which preferences are represented by the following utility function 
\[
\int_{0}^{\infty} U(C_{gt}, C_{st}) \, \text{e}^{-\rho t}\, dt .
\]
\noindent The discount rate is given by \( \rho > 0 \). 
The instantaneous utility function \( U(\cdot, \cdot) \) is assumed to be a price-independent-generalized-linear indirect
utility (PIGL henceforth).
Since the PIGL class generally lacks a known direct utility representation, we work with its indirect utility formulation $v(E_t, P_{gt}, P_{st}) $. Following Boppart (2014), let us assume
\begin{assumption}\label{ass:IUF}
The instantaneous utility function $U(C_{gt}, C_{st})$ is PIGL with indirect utility representation
\begin{equation}\label{eq:IUF}
V(E_t, P_{gt}, P_{st}) = \frac{1}{\chi}  \left( \frac{E_t}{P_{st}} \right)^\chi - \frac{\eta}{\gamma} \left( \frac{P_{gt}}{P_{st}} \right)^\gamma  - \frac{1}{\chi} + \frac{\eta}{\gamma},
\end{equation}
where \(E_t\) is consumption expenditure, \( \eta > 0 \) and \( 1 > \gamma > \chi > 0 \). 
\end{assumption}

%%%%%%%%%%%%%
\paragraph{Intertemporal problem and intratemporal allocation.}
%%%%%%%%%%%%%

Under Assumption~\ref{ass:IUF}, the representative household choses a path $\{E_t,K_t\}$, for consumption expenditure  and capital, that solves the following dynamic problem
\[
v(K_t) = \max \int_{0}^{\infty}  \frac{E_t^\chi}{\chi} \, \Gamma_t \, dt 
\]
subject to
\begin{equation}\label{eq:K2}
\boxed{
\dot K_t = {\cal A}_{t} K_t^\theta - E_t - \delta K_t ,
}
\end{equation}
where the discount factor, $\Gamma_t = P_{st}^{-\chi} \text{e}^{-\rho t}$, is smaller than one and declining over time since $P_{st}$, as in the data, is assumed to be increasing over time.
$E_t$ is nothing else that our measure, in units of the investment good, of total consumption. Preferences are then CIES, with elasticity of substitution larger than one.
All other terms in (\ref{eq:IUF}) are excluded as they are additive; their discounted integral remains a constant, having no impact on the determination of the optimal path.

The Euler equation associated to the household problem above is
\begin{equation}\label{eq:Euler}
\boxed{
\frac{\dot E_t}{E_t} = \frac{1}{1-\chi} \left(\theta {\cal A}_t K_t^{\theta - 1} - \rho - \delta - \chi \frac{\dot P_{st}}{P_{st}}\right) .
}
\end{equation}
The equilibrium path solves then (\ref{eq:K2}) and (\ref{eq:Euler}), given $K_0$.

The intratemporal allocation of $E_t$ to $C_{gt}$ and $C_{st}$ results from the use of  Roy’s Identity to derive the expenditure share of goods:
\begin{equation}\label{eq:Roy}
\boxed{
\frac{P_{gt} C_{gt}}{E_t} = \eta \left( \frac{E_t}{P_{st}} \right)^{-\chi} \left( \frac{P_{gt}}{P_{st}} \right)^\gamma.
}
\end{equation}
The last equation solves for $C_{gt}$ and then $C_{st}$ can be obtained inverting the definition of consumption expenditure $E_t$.

%%%%%%%%%%%%%
\paragraph{How compute the solution.}
%%%%%%%%%%%%%

In the following, to measure quantity indices, information on prices and quantities for consumption, investment and income will be required. 
The use of the investment good as numeraire, in this framework, is inconsequential. These are the steps to follow in order to measure the needed variables.

\begin{enumerate}

\item Assume that $A_{gt}$, $A_{st}$ and ${\cal A}_t$ all three grow at different constant rates, with $\widehat A_{g} > \widehat A_{s}$. In the data, $\widehat A_{g}$ is very close to $\widehat {\cal A}$.
\item Use equations (\ref{eq:Pj}) and (\ref{eq:calA}) to solve for  $P_{gt}$ and $P_{st}$.
\item Solve the dynamic system (\ref{eq:K2}) and (\ref{eq:Euler}) to compute $E_t$ and $K_t$, and the value function $v(K_t)$.
\item Use (\ref{eq:E}) and (\ref{eq:Roy}) to solve for $C_{gt}$ and $C_{st}$
\item Use (\ref{eq:identity}) and (\ref{eq:output}) to solve for $I_t$ and $Y_t$.

\end{enumerate}

%%%%%%%%%%%%%
\paragraph{Bellman representation.}
%%%%%%%%%%%%%

Following Duran and Licandro (2025), the Bellman representation of the representative household preferences is
\begin{equation}\label{eq:BR}
W(C_{gt}, C_{st}, X_t;  \nu_t) = U(C_{gt}, C_{st}) + \nu_t X_t ,
\end{equation}
where $X_t = \dot K_t$ is net investment and $\nu_t$ is the marginal value of capital at time $t$.
In the Bellman representation, preferences at $t$ are indexed by the marginal value of capital $\nu_t$.
However, it's very important to stress that $Y_t$ is not real expenditure but nominal expenditure, and as such it will be our measure of nominal GDP.
The representative household maximises (\ref{eq:BR})  with respect to $\{C_g,C_s,X\}$, subject to the budget constraint
\[
P_{gt} C_{gt} + P_{st} C_{st} + X_t = M_t ,
\]
where $M_t$ is current net income. Notice that consumption expenditure $E_t = M_t - X_t$. 

\begin{proposition}
The indirect utility and expenditure functions associated to the Bellman representation of preferences in (\ref{eq:BR}) are
\begin{equation}\label{eq:indirect}
u(M_t, P_{gt}, P_{st};\nu_t) = V\left(\left(\nu_t P_{st}^\chi\right)^{\frac{1}{\chi-1}},P_{gt},P_{st}\right) 
+ \nu_t \left( M_t - \left(\nu_t P_{st}^\chi\right)^{\frac{1}{\chi-1}} \right)
\end{equation}
\begin{equation}\label{eq:expenditure}
e(W_t, P_{gt}, P_{st};\nu_t) = \left(\nu_t P_{st}^\chi\right)^{\frac{1}{\chi-1}} + \frac{W_t}{\nu_t} - \frac{ V\left(\left(\nu_t P_{st}^\chi\right)^{\frac{1}{\chi-1}},P_{gt},P_{st}\right) }{\nu_t} .
\end{equation}
\end{proposition}

\noindent{\sc Proof:}
Since by definition of an indirect utility function
\[
\max_{\{C_{gt},C_{st}\}:P_{gt} C_{gt} + P_{st} C_{st} = E_t} U(C_{gt}, C_{st}) = V(E_t, P_{gt}, P_{st})  ,
\]
optimal net investment is
\[
X_t = \arg\max_{X}\ V(M_t - X, P_{gt}, P_{st}) +\nu_t X = M_t - \left(\nu_t P_{st}^\chi \right)^{\frac{1}{\chi-1}}.
\]
The indirect utility function associated to the Bellman representation of preferences is
\[
u(M_t,P_{gt},P_{st};\nu_t) = V\big(\left(\nu_t P_{st}^\chi \right)^{\frac{1}{\chi-1}}, P_{gt}, P_{st} \big)  +\nu_t \Big(M_t - \left(\nu_t P_{st}^\chi \right)^{\frac{1}{\chi-1}}\Big)
\]
and the expenditure function is
\[
e(W_t,P_{gt},P_{st};\nu_t) = \frac{W_t - V\big(\left(\nu_t P_{st}^\chi \right)^{\frac{1}{\chi-1}}, P_{gt}, P_{st} \big)}{\nu_t}  + \left(\nu_t P_{st}^\chi \right)^{\frac{1}{\chi-1}} . \qed
\]



%%%%%%%%%%%%%
\paragraph{Equivalent variation measures.}
%%%%%%%%%%%%%

Based on the Fisher and Shell principle that welfare comparisons must be done using the same preference set, let us define the hypothetical income at $z$, for \( z < t \), 
\begin{equation}\label{eq:Mhat}
\widehat M_{tz} = e \Big( u\big(M_z, P_{gz}, P_{sz};\nu_t\big),P_{gt}, P_{st};\nu_t\Big) .
\end{equation}
$\widehat M_{tz}$ is the level of income at current prices that the representative household would have needed at time \( z \) to attain the utility achievable with past income and prices but evaluated using the current Bellman representation of preferences.

%%%%%%%%%%%%%
\paragraph{Fixed-base indices.}
%%%%%%%%%%%%%

Let us adopt the following convention for an economy where an Office for National Statistics have recorded National Accounts data from some initial time $t_0$ to the current time $t$. 
In this economy, a current-base equivalent variation index, for $s\in\{t_0,t\}$, is
\begin{equation}\label{eq:CBEV}
{\cal P}_{t,s} =  \widehat m_{t,s} -  \widehat m_{t,t_0} ,
\end{equation}
where $\widehat m_{t,z} =\log \widehat M_{tz}$.
Notice that the index is normalized to ${\cal P}_{t,t_0} = 0$, such that ${\cal P}_{t,t} =  m_t -  \widehat m_{t,t_0}$ measures welfare gains from the initial time $t_0$ to the current time $t$, and ${\cal P}_{t,t} -{\cal P}_{t,s} =  m_t -  \widehat m_{t,s}$ welfare gains from any time $s\in(t_0,t)$ to $t$.
Implicit on this index, the equilibrium instantaneous growth rate of the economy at $s$ is measured by 
\begin{equation}\label{eq:BBEVgrowth}
\frac{\partial{\cal P}_{t,s}}{\partial s} = \frac{\partial \widehat m_{t,s}} {\partial s}  .%= 
%\frac{ \left( \frac{m_s}{p_s} - \frac{\lambda}{\nu_t} \right) p_t g_k}{\widehat m_{t,s}} .
\end{equation}
As we show below, the instantaneous growth rate at $t$ of the current-base equivalent variation index, $\frac{\partial{\cal P}_{t,s}}{\partial s}|_{s=t} $, %= (1-\lambda s_c) g_k$ 
 is equal to the Divisia index at $t$. %, defined as $g^{\text{D}}=s_c g_c + (1-s_c) g_x$. 
For any $s<t$, the instantaneous growth rate is lower than the Divisia index and declines as the welfare evaluation refers to a more distant point in the past. (prove it)
%In fact, the growth rate as measured by the current-base BBEV index in (\ref{eq:BBEVgrowth}) may become negative, since it evaluates the past using current preferences, $\nu_t$ is fixed, but income measured in units of the investment good, $\frac{m_s}{p_s}$, declines at the rate $g_k$ when moving back to the past.

%%%%%%%%%%%%%
\paragraph{Chained Divisia.}
%%%%%%%%%%%%%

Following Duran and Licandro (2025), 

%%%%%%%%%%%%%
\paragraph{Parameter values.}
%%%%%%%%%%%%%
Table~\ref{tab:parameters} shows the parameters values used by Herrendorf et al. (2021) in their quantitative exercise.
In the measurement of sectoral total factor productivity (TFP) they have followed a structured approach based on observable economic data. 
First, sector-specific TFP (\( A_{jt} \)), $j=\{g,s\}$,  was estimated using data from WORLD KLEMS on real value added, capital and labor inputs. Given the aggregate capital share \( \theta \), they computed the sectoral TFP growth rates using 
\[
\widehat{A}_{jt} = \widehat{Y}_{jt} - \theta \widehat{K}_{jt} - (1 - \theta) \widehat{L}_{jt}, \quad j \in \{g,s\},
\]
where $\widehat x$ measures the growth rate of variable $x$.
All variables except \( \widehat{A}_{jt} \) are directly observable. Normalizing initial TFP levels to \( A_{j0} = 1 \), they use the estimated growth rates to construct the time series for \( A_{jt} \).

Next, they estimated aggregate investment-specific TFP (\( A_{xt} \)). Again, setting the initial condition \( A_{x0} = 1 \), they computed its growth rates using 
\[
\widehat{A}_{xt} = \widehat{X}_t - \frac{P_{gt} X_{gt}}{X_t} \widehat{X}_{gt} - \frac{P_{st} X_{st}}{X_t} \widehat{X}_{st},
\]
where all components at the right-hand-side of this equation are observable. 
%Finally, we assess the theoretical conditions required to ensure the existence of an aggregate balanced growth path (ABGP).

\begin{table}[h]
    \centering
    \begin{tabular}{lc}
        \hline
        \textbf{Parameter} & \textbf{Value} \\
        \hline
         \hline
        \textbf{Preferences} & \\
        \hline
        $\rho$ (discount rate) & 0.04  \\
        $\chi$ & 0.55  \\
        $\eta$ & 0.44 \\
        $\gamma$ & 0.69 \\
         \hline
        \textbf{Technology} & \\\hline
        $\theta$ (capital share) & 1/3  \\
        $\delta$ (depreciation rate) & 0.08  \\
         $\omega$ & 0.65 \\
        $\varepsilon$ & 0.00 \\
       % $A_{st}$ growth rate (pre-1965) & 0.8\% \\
       % $A_{st}$ growth rate (post-1965) & 0.2\% \\
       % $A_{xt}$  constant \\
        %Decrease in $A_{st}$ growth rate & 0.006 \\
        %Increase in rental rate of capital & 0.0034 \\        
        %Implied level effect on output & Approximately 1\% \\
       % $A_{gt} > A_{st}$ & Holds for average growth rates over the period \\
        \hline
    \end{tabular}
    \caption{Herrendorf et al. (2021) calibration}
    \label{tab:parameters}
\end{table}

\appendix

\section{ABGP}

Let us assume that at the ABGP, $A_{gt} = A_{g0}\, \text{e}^{\gamma_g t}$, $A_{st} = A_{s0}\, \text{e}^{\gamma_s t}$, and ${\cal A}_{t} = {\cal A}_{0}\, \text{e}^{\gamma_{\cal A} t}$, $\gamma_x> \gamma_g > \gamma_s$. Consequently, the growth rate of $P_{gt}$ and $P_{st}$ are $g_{P_g}$ and $g_{P_s}$, respectively, with $0 < g_{P_g} = \gamma_x - \gamma_g < \gamma_x - \gamma_s = g_{P_s}$.

At the ABGP, from the Euler equation (\ref{eq:Euler}), the stock of capital follows
\begin{equation}
K_t^{*} = k^{* \frac{1}{\theta-1}} {\cal A}_t^{\frac{1}{1-\theta}}
\ \ \ \ \text{where}\ \ \ \ \
k^* = \frac{\rho + \delta + \chi g_{P_{s}} + (1-\chi)g_{E}}{\theta}
\end{equation}
is the user cost of capital divided by $\theta$.
From  (\ref{eq:K2}), consumption expenditure follows
\begin{equation}
E^*_t  =  (k^*- \delta - g_{K}) K^*_t .
\end{equation}
From (\ref{eq:output}), gross income is
\begin{equation}
Y^*_t = {\cal A}_{t} K_t^{*\,\theta} = k^* K_t^{*}.
\end{equation}
Notice that the consumption share of gross income is
\[
\frac{E^*_t}{Y^*_t} = \frac{k^*- \delta - g_{K}}{k^*} .
\]
All three variables, $K_t$, $E_t$ and $Y_t$ are measure in units of the investment good and grow at the rate $g_K=\frac{\gamma_{\cal A}}{1-\theta}$.

Since from the primal problem of the household
\[
\max V(E, P_{gt},P_{st}) + \nu_t X
\ \ \ \ \text{subject to}\ \ \ \ \ 
E + X = M_t .
\]
From the FOC for $E$ 
\[
\nu^*_t = E_t^{*\, 1-\chi} P_{st}^\chi .
\]
We have then all information required to compute the current-base equivalent variation measure in (\ref{eq:CBEV}).
 
From (\ref{eq:Roy}), real consumption on goods is
\begin{equation}
C_{gt} = \eta \left( \frac{E_t}{P_{st}} \right)^{1-\chi} \left( \frac{P_{gt}}{P_{st}} \right)^{\gamma-1} ,
\end{equation}
which grows at the constant rate $g_g = (1-\chi) g_k +(\gamma-1) g_{P_{g}} + (\chi-\gamma) g_{P_{s}}$.

%%%%%%%%%%%%%%%%%%
\paragraph{Divisia index.}
%%%%%%%%%%%%%%%%%%

The Divisia index 
\[
g^D_t = s_e \big( s_{gt} g_g + (1-s_{gt}) g_s \big) + (1-s_e) g_x ,
\]
where the shares are $s_e \equiv \frac{E_t}{M_t}= \frac{k^*- \delta - g_{K}}{k^*} $ and, from (\ref{eq:Roy}) 
$s_{gt} = \eta \left( \frac{E_t}{P_{st}} \right)^{-\chi} \left( \frac{P_{gt}}{P_{st}} \right)^\gamma$. The growth rate of investment is $g_x = g_K$.
We are only missing the growth rate of consumption services. Notice that, from (\ref{eq:Roy}), real consumption services are
\[
C_{st} = \frac{E_t}{P_{st}} - \eta \left( \frac{E_t}{P_{st}} \right)^{1-\chi} \left( \frac{P_{gt}}{P_{st}} \right)^{\gamma} ,
\]

%%%%%%%%%%%%%%%%%%
\paragraph{Fisher-Shell index.}
%%%%%%%%%%%%%%%%%%
In order to show the Fisher-Shell index is equal to the Divisia index, we must compute the total derivative of $\widehat M_{tz}$ with respect to $z$ and evaluate it at $z=t$. From the main text,
\[\tag{\ref{eq:Mhat}}
\widehat M_{tz} = e \Big( u\big(M_z, P_{gz}, P_{sz};\nu_t\big),P_{gt}, P_{st};\nu_t\Big)  ,
\]
with
\[\tag{\ref{eq:indirect}}
u(M_t, P_{gt}, P_{st};\nu_t) = V\left(\left(\nu_t P_{st}^\chi\right)^{\frac{1}{\chi-1}},P_{gt},P_{st}\right) 
+ \nu_t \left( M_t - \left(\nu_t P_{st}^\chi\right)^{\frac{1}{\chi-1}} \right) ,
\]
\[\tag{\ref{eq:expenditure}}
e(W_t, P_{gt}, P_{st};\nu_t) = \left(\nu_t P_{st}^\chi\right)^{\frac{1}{\chi-1}} + \frac{W_t}{\nu_t} - \frac{ V\left(\left(\nu_t P_{st}^\chi\right)^{\frac{1}{\chi-1}},P_{gt},P_{st}\right) }{\nu_t} ,
\]
and
\[\tag{\ref{eq:IUF}}
V(E_t, P_{gt}, P_{st}) = \frac{1}{\chi}  \left( \frac{E_t}{P_{st}} \right)^\chi - \frac{\eta}{\gamma} \left( \frac{P_{gt}}{P_{st}} \right)^\gamma  - \frac{1}{\chi} + \frac{\eta}{\gamma} .
\]
Combining them, we get
\[
\widehat M_{tz} = \frac{1-\chi}{\chi} \left(\nu_t P_{st}^\chi\right)^{\frac{1}{\chi-1}} - \frac{1}{\nu_t} \frac{\eta}{\gamma} \left( \frac{P_{gt}}{P_{st}} \right)^\gamma + M_z + \text{other terms that don't depend on $z$}
\]
In order to define the Fisher-Shell index, we take time derivatives with respect to $z$ and evaluate them at $z = t$, such that,
\[
g^{\text{FS}}_t \equiv \frac{\text{d} \widehat M_{tz}}{\text{d} z}\Bigg|_{z=t} \frac{1}{M_t} = \frac{\dot M_t}{M_t}  - \frac{E_t}{M_t} g_{P_{st}} -
\frac{\eta}{\nu_t M_t}  \left( \frac{P_{gt}}{P_{st}} \right)^\gamma
\big(g_{P_g t} - g_{P_s t}\big) .
\]

Since
\[
\frac{\dot M_t}{M_t}  = s_{et}\, g_{Et}+ (1-s_{et}) g_{Xt}  ,
\]
where $s_e = E/M$, and 
\[
 g_{Et}  = s_{gt} \big(g_{gt} + g_{P_{g}t}\big) +  s_{st} \big(g_{st} + g_{P_{s}t}\big) ,
\]
the Fisher-Shell index becomes
\[
g^{\text{FS}}_t = g^{D}_t + \frac{P_{gt}C_{gt}}{M_t} \big( g_{P_gt} -g_{P_gt}\big)  -
\frac{\eta}{\nu_t M_t}  \left( \frac{P_{gt}}{P_{st}} \right)^\gamma
\big(g_{P_gt} - g_{P_st}\big) .
\]
From (\ref{eq:Roy}), 
\[
\frac{P_{gt} C_{gt}}{E_t} = \eta \left( \frac{E_t}{P_{st}} \right)^{-\chi} \left( \frac{P_{gt}}{P_{st}} \right)^\gamma.
\]
Since at equilibrium $E_t^{\chi-1} = \nu_t P_{st}^\chi$, we can easily show that 
\[
g^{\text{FS}}_t = g^{D}_t  .
\]


%%%%%%%%%%%%%
\end{document}
%%%%%%%%%%%%%
